
\documentclass[10pt,twoside]{book}

%\usepackage[utf8]{inputenc}  %Para compilar en PdfLaTeX

%Recordatorio de relleno

%macro del capítulo
%\INEchapter{TÍTULO: SUBTÍTULO (para índice)}{TÍTULO:}{SUBTÍTULO}{Descripción}
%Hoja

%caja de media hoja
%\cajita{Título}{Descripción}{Subtítulo}{Desagregación}{Gráfica}{Fuente}{\notita{Nota}}

%caja de hoja completa
%\cajota{Título}{Descripción}{Subtítulo}{Desagregación}{Gráfica}{Fuente}{\notita{Nota}}



%Paquetes estándar
\usepackage{amsmath}
\usepackage{amsfonts}
\usepackage{amssymb}
\usepackage{graphicx}
\usepackage{pdfpages}
\usepackage{setspace} 
\usepackage{xltxtra}
\usepackage{enumitem}
\usepackage{fixltx2e}


%título
\newcommand{\titulodoc}{Nombre del documento}


%Tablas de Excel convertidas a LaTeX
\usepackage{booktabs}
\usepackage{multirow}
\newcounter{Cuadro}[chapter]
\renewcommand{\theCuadro}{\thechapter.\arabic{Cuadro}}


%Columnas definibles en ancho
\usepackage{array}
\newcolumntype{x}[1]{%
	>{\centering\arraybackslash}p{#1}}%

\newcolumntype{g}[1]{%
	>{\raggedleft\arraybackslash}p{#1}}%


\usepackage[input-decimal-markers={.}, input-ignore={,}, group-separator={,}]{siunitx}


%Para pruebas
\usepackage{lipsum}
\newcommand{\comop}{}
\newcommand{\comcl}{}
\newcommand{\guiop}{}
\newcommand{\guicl}{}
\newcommand{\apartado}[1]{\addtocounter{section}{1}
	{\noindent\Bold\Large\color{color1!80!black}\thesection $\,-$ #1}\\[2mm]}


%Para compilar en XeLaTeX con tildes
\usepackage{polyglossia}
\setmainlanguage{spanish}

%Tipo de letra
\usepackage{fontspec}
\setmainfont[
BoldFont = OpenSans-CondBold.ttf ,
ItalicFont = texgyrepagella-italic.otf ,
BoldItalicFont = OpenSans-CondLightItalic.ttf ]{OpenSans-CondLight.ttf}
\newfontfamily\Bold{Open Sans Condensed Bold}

\newfontfamily\Sans{Open Sans}
\newfontfamily\Italic{Open Sans Condensed Light Italic}
\newfontfamily\Logos{Latin Modern Roman}



%Diseño global del documento
\usepackage[paperwidth=8.5in, paperheight=6.5in, left=0.950in, right=0.8in, top=0.525in, bottom=0.675in]{geometry}
%	\setlength{\headsep}{0pt}
%	\setlength{\footskip}{46pt}
\setlength{\parindent}{2em}		%sangría
\setlength{\parskip}{2ex}		%separación entre párrafos  

%Distancias
\newlength{\cuadri} 
\setlength{\cuadri}{0.125in}




%Tabla de contenidos y vinculaciones
\usepackage{tocloft}
\usepackage[hidelinks]{hyperref}
\usepackage{url}

%Formato de  de contenidos
\setlength{\cftbeforetoctitleskip}{0em}
\AtBeginDocument{\addtocontents{toc}{\protect\thispagestyle{empty}}} 

\makeatletter
\renewcommand*\l@subsection{\@dottedtocline{2}{5.2em}{3.2em}}
\makeatother

\renewcommand{\thesection}{\thechapter.\arabic{section}}

\cftsetpnumwidth{2\cuadri}
\cftsetrmarg{8\cuadri}
\renewcommand{\cftsecnumwidth}{2.0\cuadri}
\renewcommand{\cftchapnumwidth}{2\cuadri}
\renewcommand{\cftsecindent}{2\cuadri}



% Elementos geométricos de diseño del cuerpo (cajas de colores, etc.)
\usepackage{colortbl}


%\usepackage[usenames,dvipsnames,svgnames,table]{xcolor}

%Cambios de márgenes y según paridad de hojas
\usepackage[strict]{changepage}
\strictpagecheck

%Colores base del documento
\definecolor{color1}{rgb}{0,0,0}
\definecolor{color2}{gray}{0.45}


%Para que las páginas en blanco no estén numeradas
\let\origdoublepage\cleardoublepage
\newcommand{\clearemptydoublepage}{
	\clearpage
	{\pagestyle{empty}\origdoublepage}}
\let\cleardoublepage\clearemptydoublepage

%Llamadas hacia notas
\newcommand{\llamada}{*$\ $}
\newcommand{\llamadaD}{**$\ $}
%tablas
\usepackage{pdflscape}
\usepackage{rotating}
\usepackage{bigstrut}
\usepackage{longtable}
\LTcapwidth=1.234\textwidth
\setlength{\arrayrulewidth}{0.8pt}
\arrayrulecolor{color1!80!black}

%Columnas centradas definibles en ancho
\newcolumntype{x}[1]{%
	>{\centering\arraybackslash}p{#1}}%



% Tcolorbox
\usepackage[skins, breakable, hooks]{tcolorbox}


\newtcolorbox{tocbox}{skin=enhancedmiddle, width=39\cuadri, nobeforeafter, boxrule=0pt, colframe=white, left=0\cuadri, enlarge left by = 4\cuadri, enlarge right by=2\cuadri, bottom=0pt, top=0pt, right=0\cuadri, left=0\cuadri, arc=0pt, colback= white, breakable,check odd page,toggle left and right}

\newtcolorbox{fondo}{width=6.75in, height=5.3in, skin=enhancedmiddle, nobeforeafter, enlarge left by=-0mm, enlarge top by=-0mm, boxrule=0pt, colframe=white, left=-10pt, bottom=-1pt, top=-3pt, right=-10pt, arc=0pt, colback= white}

\newtcolorbox{hoja-capitulo}{width=8.5in, height=6.5in,   skin=enhancedmiddle,  nobeforeafter,  watermark graphics=capitulo3.png, watermark opacity=1.0, watermark overzoom=1.0, enlarge left by=-7.65\cuadri, enlarge top by=-14.68\cuadri,  enlarge bottom by=-20\cuadri, boxrule=0pt, colframe=white, left=-3pt, bottom=-1pt, top=8\cuadri, right=-3pt, arc=0pt}

\newtcolorbox{columnatipoA}{width=3.19in, height=5.28in, nobeforeafter, enlarge left by=0.15pt, enlarge top by=-0mm, boxrule=0pt, colframe=white, left=-3pt, bottom=-1pt, top=0.9pt, right=-3pt, arc=0pt, colback= white}

\newtcolorbox{fondo-landscape}{width=39\cuadri, height=38\cuadri, nobeforeafter, enlarge left by=-3mm,
	enlarge bottom by=-40mm, enlarge top by=-11mm, boxrule=0pt, colframe=white, left=0pt, bottom=0pt, top=0pt, right=0pt, arc=0pt, colback= white}

\newtcolorbox{bloque-media}{width=40\cuadri, height=19\cuadri, enlarge top by=-3pt, enlarge left by=-3pt, enlarge bottom by=1.4\cuadri, nobeforeafter, colframe=white, colback=white, left=0pt, right=0pt, bottom= 0pt, top=0pt, arc=0pt, boxrule=0pt}

\newtcolorbox{bloque-una}{width=40\cuadri, height=1\cuadri, enlarge top by=-3pt, enlarge left by=-3pt, enlarge bottom by=1.4\cuadri, nobeforeafter, colframe=white, colback=white, left=0pt, right=0pt, bottom= 0pt, top=0pt, arc=0pt, boxrule=0pt}

\newtcolorbox{descripcion}{width=10.5\cuadri, height=15.9\cuadri, enlarge bottom by=0\cuadri, enlarge top by=-3pt, enlarge left by=-3pt, nobeforeafter, boxrule=3pt, colback=color1!8!white, colframe=color1!8!white, left=5pt, right=5pt, top=4pt,bottom=4pt}

\newtcolorbox{descripcion-una}{width=33.3\cuadri, height=6.2\cuadri, enlarge bottom by=1\cuadri, enlarge top by=-5pt, enlarge left by=-3pt, nobeforeafter, boxrule=3pt, colback=color1!8!white, colframe=color1!8!white, left=5pt, right=5pt, top=3pt,bottom=4pt}




\newtcolorbox{descripcion-titulin}{width=33.3\cuadri, enlarge bottom by=1\cuadri, enlarge top by=-5pt, enlarge left by=-3pt, nobeforeafter, boxrule=3pt, colback=white, colframe=white, left=5pt, right=5pt, top=3pt,bottom=4pt}

\newtcolorbox{titulin}{width=33.3\cuadri, height=3.2\cuadri, enlarge bottom by=1\cuadri, enlarge top by=0pt, enlarge left by=-3pt, nobeforeafter, boxrule=0pt, colback=white, colframe=white, left=5pt, right=5pt, top=6pt, bottom=4pt}


\newcommand{\titulito}[2]{
	\addtocontents{toc}{\protect\addvspace{0.4\baselineskip}\color{black}}
	\addcontentsline{toc}{section}{\textbf{#1}}
	\begin{bloque-media}
		$\ $\\[1.4cm]
		\begin{titulin}
			\begin{center}
				{\Huge\Bold\color{color1!90!black} #1}\\[1.5cm]
			\end{center}
		\end{titulin}
		
		\begin{tabular}{x{33.3\cuadri}}
			\hline
			$\ $\\
			\begin{descripcion-titulin}
				#2
			\end{descripcion-titulin} \\
			$\ $\\[-0.9cm] \hline 
		\end{tabular}
	\end{bloque-media} 
}

\newcommand{\titulitond}[1]{
	\addtocontents{toc}{\protect\addvspace{0.4\baselineskip}\color{black}}
	\addcontentsline{toc}{section}{\textbf{#1}}
	\begin{bloque-media}
		$\ $\\[1.4cm]
		\begin{titulin}
			\begin{center}
				{\Huge\Bold\color{color1!90!black} $\ $}\\[1.5cm]
			\end{center}
		\end{titulin}
		
		\begin{tabular}{x{33.3\cuadri}}
			\hline
			$\ $\\
			\begin{descripcion-titulin}
				\begin{center}
					{\Huge\Bold\color{color1!90!black}#1}
				\end{center}
			\end{descripcion-titulin} \\
			$\ $\\[-0.9cm] \hline 
		\end{tabular}
	\end{bloque-media} 
}



\newtcolorbox{notita-impar}{width=5.9\cuadri, enlarge bottom by=0\cuadri, enlarge top by=1\cuadri, enlarge left by=0.1\cuadri, nobeforeafter, boxrule=0pt, colback=white, colframe=color2!40!white, left=3pt, right=1pt, top=2pt,bottom=2pt, leftrule=0.6pt, rightrule=0pt,toprule=0pt,bottomrule=0pt, arc=0pt}

\newtcolorbox{notita-par}{width=5.9\cuadri, enlarge bottom by=0\cuadri, enlarge top by=1\cuadri, enlarge left by=-0.55\cuadri, nobeforeafter, boxrule=0pt, colback=white, colframe=color2!40!white, left=1pt, right=3pt, top=2pt,bottom=2pt, rightrule=0.6pt, leftrule=0pt,toprule=0pt,bottomrule=0pt, arc=0pt}

\newtcolorbox{numero-subseccion}{width=2.6\cuadri, height=1.4\cuadri, enlarge top by=-3pt, enlarge left by=-2.8\cuadri, enlarge right by= 4.613\cuadri, nobeforeafter, boxrule=0pt, colback=white, colframe=white, left=0.1\cuadri, right=0.1\cuadri, bottom= 0.455\cuadri, top=0.145\cuadri, arc=0pt}

\newtcolorbox{titulo-subseccion}{width=28.8\cuadri, enlarge top by=-3pt, enlarge left by=-0.2\cuadri, enlarge bottom by=0\cuadri, nobeforeafter, boxrule=0pt, colback=white, colframe=white, left=0.1\cuadri, right=0.1\cuadri, bottom= -0.4\cuadri, top=0\cuadri, arc=0pt}

\newtcolorbox{titulo-subseccion-blanco}{width=28\cuadri, enlarge top by=-0pt, enlarge left by=0pt, enlarge bottom by=0pt, nobeforeafter, colback=white, colframe=white, left=0\cuadri, right=0.5\cuadri, bottom= -4pt, top=-2pt, arc=0pt, bottomrule=0pt, leftrule=0pt, toprule= 0mm, rightrule= 0mm}

\newtcolorbox{titulo-subseccion-continuacion}{width=32\cuadri, enlarge top by=-3pt, enlarge left by=-0.2\cuadri, enlarge bottom by=0.4\cuadri, nobeforeafter, boxrule=0pt, colback=white, colframe=white, left=0.3\cuadri, right=0.1\cuadri, bottom= -0.4\cuadri, top=0\cuadri, arc=0pt}

\newtcolorbox{titulo}{width=34\cuadri, height=3\cuadri, enlarge top by=-3pt, enlarge left by=-3pt, enlarge bottom by=0.2\cuadri, nobeforeafter, colback=white, colframe=white, left=0.5\cuadri, right=0.5\cuadri, top=32pt,bottom=-48pt, arc=0pt, boxrule=0pt}

\newtcolorbox{centrador}{width=34\cuadri, enlarge top by=-48pt, enlarge left by=-3pt, enlarge bottom by=0pt, nobeforeafter, colback=white, colframe=white, left=-3pt, right=-3pt, bottom= -3pt, top=-3pt, arc=0pt, boxrule=0pt}

\newtcolorbox{centrador-par}{width=34\cuadri, enlarge top by=-48pt, enlarge left by=66.7pt, enlarge bottom by=0pt, nobeforeafter, colback=white, colframe=white, left=-3pt, right=-3pt, bottom= -3pt, top=-3pt, arc=0pt, boxrule=0pt}

\newtcolorbox{subtitulo}{width=22\cuadri, height=3\cuadri, enlarge top by=-3pt, enlarge left by=-3pt, enlarge bottom by=0.1\cuadri, nobeforeafter, colframe=white, colback=white, left=0\cuadri, right=0\cuadri, bottom= 0pt, top=0pt, arc=0pt, boxrule=0pt}

\newtcolorbox{subtitulo-una}{width=28\cuadri, height=3\cuadri, enlarge top by=-3pt, enlarge left by=-3pt, enlarge bottom by=0.1\cuadri, nobeforeafter, colframe=white, colback=white, left=0\cuadri, right=0\cuadri, bottom= 0pt, top=0pt, arc=0pt, boxrule=0pt}

\newtcolorbox{grafica}{width=22.1\cuadri, height=12\cuadri, enlarge top by=-3pt, enlarge left by=-5pt, enlarge bottom by=0.1\cuadri, nobeforeafter, colframe=white, colback=white, left=-2pt, right=-2pt, bottom= -2pt, top=-4pt, arc=0pt, boxrule=0pt}

\newtcolorbox{grafica-una}{width=33.4\cuadri, height=24.9\cuadri, enlarge top by=-3pt, enlarge left by=-5pt, enlarge bottom by=0.1\cuadri, nobeforeafter, colframe=white, colback=white, left=-2pt, right=-2pt, bottom= -2pt, top=-4pt, arc=0pt, boxrule=0pt}

\newtcolorbox{fuente}{width=22\cuadri, height=1\cuadri, enlarge top by=-3pt, enlarge left by=-3pt, enlarge bottom by=0\cuadri, nobeforeafter, colframe=white, colback=white, left=0pt, right=0pt, bottom= 0pt, top=2pt, arc=0pt, boxrule=0pt}

\newtcolorbox{fuente-una}{width=33.4\cuadri, height=1\cuadri, enlarge top by=-3pt, enlarge left by=-3pt, enlarge bottom by=0\cuadri, nobeforeafter, colframe=white, colback=white, left=0pt, right=0pt, bottom= 0pt, top=2pt, arc=0pt, boxrule=0pt}

\newtcolorbox{columna-central}{width=22\cuadri, height=15.9\cuadri, enlarge top by=-3pt, enlarge left by=-8pt, enlarge bottom by=0\cuadri, nobeforeafter, colframe=white, colback=white, left=0pt, right=0pt, bottom= 0pt, top=0pt, arc=0pt, boxrule=0pt}

\newtcolorbox{columna-central-una}{width=33.4\cuadri, height=28.9\cuadri, enlarge top by=-3pt, enlarge left by=-4pt, enlarge bottom by=0\cuadri, nobeforeafter, colframe=white, colback=white, left=0pt, right=0pt, bottom= 0pt, top=0pt, arc=0pt, boxrule=0pt}

\newtcolorbox{vacio1}{width=6\cuadri, height=3\cuadri, enlarge top by=-3pt, enlarge left by=-3pt, enlarge bottom by=0.1\cuadri, nobeforeafter, colframe=white, colback=white, left=0pt, right=0pt, bottom= 0pt, top=0pt, arc=0pt, boxrule=0pt}

\newtcolorbox{nota}{width=6\cuadri, height=12\cuadri, enlarge top by=-3pt, enlarge left by=-3pt, enlarge bottom by=0.1\cuadri, nobeforeafter, colframe=white, colback=white, left=0pt, right=0pt, bottom= 0pt, top=0pt, arc=0pt, boxrule=0pt}

\newtcolorbox{nota-una}{width=6\cuadri, height=24.9\cuadri, enlarge top by=-3pt, enlarge left by=-3pt, enlarge bottom by=0.1\cuadri, nobeforeafter, colframe=white, colback=white, left=0pt, right=0pt, bottom= 0pt, top=0pt, arc=0pt, boxrule=0pt}

\newtcolorbox{vacio2}{width=6\cuadri, height=1\cuadri, enlarge top by=-3pt, enlarge left by=-3pt, enlarge bottom by=0\cuadri, nobeforeafter, colframe=white, colback=white, left=0pt, right=0pt, bottom= 0pt, top=0pt, arc=0pt, boxrule=0pt}

\newtcolorbox{columna-marginal}{width=6\cuadri, height=15.9\cuadri, enlarge top by=-3pt, enlarge left by=-8pt, enlarge bottom by=0\cuadri, nobeforeafter, colframe=white, colback=white, left=0pt, right=0pt, bottom= 0pt, top=0pt, arc=0pt, boxrule=0pt}

\newtcolorbox{columna-marginal-una}{width=6\cuadri, height=28.9\cuadri, enlarge top by=-3pt, enlarge left by=-8pt, enlarge bottom by=0\cuadri, nobeforeafter, colframe=white, colback=white, left=0pt, right=0pt, bottom= 0pt, top=0pt, arc=0pt, boxrule=0pt}


% Encabezado y pie de página
\usepackage{fancyhdr}

\newlength{\nombrecapitulo}

%cajitas de encabezado y pie de página
\newtcbox{pagina}{nobeforeafter, boxrule=0pt,width= 3\cuadri, height=2\cuadri,  colback=white, left=2pt, right=2pt,   top=-1pt, bottom=-2pt, arc=0pt, enlarge left by=-10pt, enlarge right by=-10pt, width=2\cuadri, colframe = white}

\newtcbox{piecapituloderecho}{enhanced, nobeforeafter, width=1\cuadri, boxrule=0pt, colback=white, left=0pt, right=0pt, bottom= 0pt, arc=0pt, enlarge left by=-0.5\cuadri, enlarge right by=-0.5\cuadri, enlarge top by=-\nombrecapitulo, enlarge bottom by=1\cuadri}

\newtcbox{encabezado}{enhanced, nobeforeafter, height=1\cuadri, boxrule=0pt, colback=white, left=0pt, right=0pt, bottom= -0pt, arc=0pt, enlarge left by=-3pt, enlarge right by=0\cuadri, enlarge top by=0\cuadri, enlarge bottom by=-0\cuadri}

\newtcbox{piecapituloizquierdo}{enhanced, nobeforeafter, width=1\cuadri, boxrule=0pt, colback=white, left=0pt, right=-3pt, bottom= 0pt, arc=0pt, enlarge left by=-0.5\cuadri, enlarge right by=-0.5\cuadri, enlarge top by=-\nombrecapitulo, enlarge bottom by=1\cuadri}

\newtcbox{margenderecho}{nobeforeafter, height=39\cuadri, width=1\cuadri, boxrule=0pt, colback=white,enlarge left by=4\cuadri, enlarge right by= -5\cuadri, enlarge top by= -48\cuadri, enlarge bottom by=4.2\cuadri, left=-3pt, bottom= 0pt, top=-3pt, arc=0pt}


\newtcbox{margenizquierdo}{nobeforeafter, height=39\cuadri, width=1\cuadri, boxrule=0pt, enlarge left by=-5\cuadri, enlarge right by= 4\cuadri,colback=white, enlarge top by= -48\cuadri, enlarge bottom by=4.2\cuadri, right=-3pt, bottom= 0pt, top=-3pt, arc=0pt}

\newtcolorbox{artepieizquierdo}{skin=enhancedmiddle,watermark graphics=pie.png, watermark opacity=1.00, watermark overzoom=0.95, nobeforeafter, height=1\cuadri, width=63\cuadri, boxrule=0pt, enlarge left by=2\cuadri, enlarge right by= -65\cuadri, enlarge top by= -2.3\cuadri, enlarge bottom by=2.4\cuadri, right=-3pt, bottom= 0pt, top=-3pt, arc=0pt, colback=white, colframe=white}


\newtcolorbox{artepiederecho}{skin=enhancedmiddle,watermark graphics=pie.png, watermark opacity=1.00,watermark overzoom=0.95, nobeforeafter, height=1\cuadri, width=63\cuadri, boxrule=0pt, enlarge left by=-65\cuadri, enlarge right by= 2\cuadri, enlarge top by= -2.3\cuadri, enlarge bottom by=2.4\cuadri, right=-3pt, bottom= 0pt, top=-3pt, arc=0pt,colback=white, colframe=white}

%definición de estilo estándar de página
\fancypagestyle{estandar}{%
	\fancyhf{}
	\fancyfoot[RO]{\settowidth{\nombrecapitulo}{\chaptitle}\setlength{\arrayrulewidth}{0.7pt}\setlength{\tabcolsep}{3pt}\arrayrulecolor{color2}
		\margenderecho{
			\begin{tabular}{x{0.2cm}}
				$\ $\\[38.2\cuadri]
				\piecapituloderecho{\rotatebox[origin=c]{270}{\color{color2}\chaptitle$\vphantom{q}$}} \\[-1\cuadri] \hline \\[-0.6\cuadri]
				\pagina{\textbf{\thepage }}  \\ \begin{artepiederecho}\end{artepiederecho}
			\end{tabular}}
		}
		\fancyhead[LE]{$\hspace{-4\cuadri}$\encabezado{\includegraphics[width=58\cuadri]{cabezae.png}}}
		\fancyhead[RO]{\encabezado{\includegraphics[width=58\cuadri]{cabezao.png}}}
		\fancyfoot[LE]{\settowidth{\nombrecapitulo}{\titulodoc}\setlength{\arrayrulewidth}{0.7pt}\setlength{\tabcolsep}{3pt}\arrayrulecolor{color2}
			\margenizquierdo{
				\begin{tabular}{x{0.2cm}}
					$\ $\\[38.2\cuadri]
					\piecapituloizquierdo{\rotatebox{90}{\color{color2}\titulodoc$\vphantom{q}$}} \\[-1\cuadri] \hline \\[-0.6\cuadri]
					\pagina{\textbf{\thepage }}  \\ \begin{artepieizquierdo}\end{artepieizquierdo}
				\end{tabular}}
			}
			\renewcommand{\headrulewidth}{0pt}
			\renewcommand{\footrulewidth}{0pt}}
		
		\pagestyle{empty}
		
		
		%Estilo de los pies de columna
		\makeatletter
		\renewcommand\footnoterule{%
			\kern-3\p@
			\color{color2}\hrule\@width2.5cm\@height0.5pt
			\kern2.6\p@}
		\makeatother
		
		%Formato de encabezados
		\usepackage[explicit]{titlesec}
		%\usepackage{sectsty}
		
		
		\titleformat{\subsection}[runin]{}{}{0pt}{}[]
		\titlespacing{\subsection}{0pt}{-20pt}{0pt}
		
		\titleformat{\section}[runin]{}{}{0pt}{}[]
		\titlespacing{\section}{0pt}{-20pt}{0pt}
		
		\titleformat{\chapter}[runin]{}{}{0pt}{}[]
		\titlespacing{\section}{0pt}{-20pt}{0pt}
		
		%Comandos recolectores de información del pie de página
		\newcommand{\chaptitle}{}
		\newcommand{\subsectitle}{}
		\newcommand{\sectitle}{}
		
		%Cajitas de capítulo
		\newtcolorbox{numero-capitulo}{enhanced, width=4.5\cuadri, height= 5\cuadri, enlarge left by=-12\cuadri, nobeforeafter, boxrule=0pt, colback=white,  colframe=white, interior style={opacity=0.5}, left=0\cuadri, right=0\cuadri, bottom= 0.2\cuadri, top=-0.7\cuadri, arc=0pt}	
		
		\newtcolorbox{titulo-capitulo}{enhanced,width=34\cuadri, height= 5\cuadri, enlarge top by=-5\cuadri, enlarge left by=36\cuadri, nobeforeafter, boxrule=0pt, colback=white,colframe=white, left=-33\cuadri, right=-4.5\cuadri, bottom= 0\cuadri, top=-0.8\cuadri, arc=0pt}
		
		\newtcolorbox{numero-capitulo-long}{width=6\cuadri, height= 5\cuadri, enlarge top by=-3pt, enlarge left by=-3pt, nobeforeafter, boxrule=0pt, colback=color1!95!black, colframe=white, left=0\cuadri, right=0\cuadri, bottom= 0.2\cuadri, top=1.2\cuadri, arc=0pt}	
		
		\newtcolorbox{titulo-capitulo-long}{enhanced,width=34\cuadri, height= 5\cuadri,interior style={opacity=0}, enlarge top by=-3pt, enlarge left by=1.5cm, nobeforeafter, boxrule=0pt, interior style={left color=color1!10!white,
				right color=white}, left=0\cuadri, right=-6\cuadri, bottom= 0\cuadri, top=0\cuadri, arc=0pt}
		
		\newtcolorbox{capitulo-descripcion}{width=34\cuadri, enlarge top by=5\cuadri, enlarge left by=-7\cuadri, nobeforeafter, enlarge right by = 7\cuadri, interior style={color= white, opacity=0.5}, frame style={color= white, opacity=0.5} left=1.5\cuadri, right=1.5\cuadri, bottom= 1\cuadri, top= 1\cuadri, arc=0pt, bottomrule=0pt, leftrule=0pt, toprule= 0pt, rightrule= 0pt}
		
		
		%Cajitas de apéndice
		\newtcolorbox{numero-capitulo-app}{width=4.5\cuadri, height= 5\cuadri, enlarge top by=-3pt, enlarge left by=-3pt, nobeforeafter, boxrule=0pt, colback=color2!95!black, colframe=color2!95!black, left=0\cuadri, right=0\cuadri, bottom= 0.2\cuadri, top=1.2\cuadri, arc=0pt}	
		
		\newtcolorbox{titulo-capitulo-app}{enhanced,width=34\cuadri, height= 5\cuadri, enlarge top by=-3pt, enlarge left by=1.5cm, nobeforeafter, boxrule=0pt, interior style={left color=white,
				right color=white}, left=0\cuadri, right=-4\cuadri, bottom= 0\cuadri, top=0\cuadri, arc=0pt}
		
		\newtcolorbox{capitulo-descripcion-app}{width=22\cuadri, enlarge top by=-3pt, enlarge left by=-3pt, nobeforeafter, enlarge right by = 6\cuadri, colback=white, colframe=color2!20!white, left=1.5\cuadri, right=2\cuadri, bottom= 1\cuadri, top= 1\cuadri, arc=0pt, bottomrule=0pt, leftrule=2pt, toprule= 0pt, rightrule= 0pt}
		
		
		%macro de sección	
		
		\newcommand{\subsecnew}[1]{\renewcommand{\subsectitle}{#1} \subsection{#1} {\Bold\large \subsectitle\\[-0.35cm]}}
		
		\newcounter{secnumber}[chapter]
		
		\newcommand{\secnew}[1]{\stepcounter{secnumber}\renewcommand{\sectitle}{#1} \section{#1} { \raisebox{0pt}{\begin{titulo-subseccion-blanco} \Bold\large\sectitle\vphantom{p}\end{titulo-subseccion-blanco}}\\[-0.35cm]}}
		
		
		
		% Formato de números de sección, subsección, etc...
		
		\newcommand{\secnumbering}{{\large\Bold \thechapter.\thesecnumber}}
		
		
		
		%Macros de relleno de contenido
		
		\newcommand{\maco}[4]{
			\begin{landscape}
				$\ $
				
				\hojarotada{
					\begin{center}
						$\ $\\[#1]
						
						
						\begin{minipage}{32\cuadri}
							\noindent{\Bold\color{color1!80!black} Cuadro \theCuadro $\,-$   #2}\\ #3\\[-3mm]
						\end{minipage}
						
						\includegraphics{#4}
					\end{center}
				}\stepcounter{Cuadro}
			\end{landscape}}
			
			
			\newcommand{\INEchapter}[4]{
				\cleardoublepage\addtocontents{toc}{\protect\addvspace{0.6\baselineskip}\color{color1!80!black}}
				
				\chapter[\texorpdfstring{\color{color1!80!black}#1}{#1}]{#2}\renewcommand{\chaptitle}{#1}\thispagestyle{empty}\addtocontents{toc}{\protect\addvspace{0.2\baselineskip}{\color{color1!10!white}\hrule height 0.9pt} \addvspace{0.6\baselineskip} \color{black}} \stepcounter{secnumber} $\ $\\[-1cm]
				\begin{hoja-capitulo}
					{$\hspace{2.6in}$\fontsize{24mm}{1em}\selectfont \Bold \thechapter  $\quad$}
					\begin{titulo-capitulo}
						\quad 
						\raisebox{2.1\cuadri}{\begin{tabular}{l}
								\fontsize{8.5mm}{1em}\selectfont \Bold \color{black} #2\\[1mm] 
								\fontsize{8.5mm}{1em}\selectfont \Bold \color{black} #3 \vphantom{Í} 
							\end{tabular}}
						\end{titulo-capitulo}
						$\ $\\[1.5cm]
						\begin{flushright}
							\begin{capitulo-descripcion}
								\large #4
							\end{capitulo-descripcion}
						\end{flushright}
					\end{hoja-capitulo}
					\cleardoublepage
				}
				
				
				\newcommand{\INEchapterlong}[4]{\cleardoublepage\addtocontents{toc}{\protect\addvspace{0.6\baselineskip}\color{color1!80!black}}\chapter[\texorpdfstring{\color{color1!80!black}#1}{#1}]{#2}\renewcommand{\chaptitle}{#1}\thispagestyle{empty}\addtocontents{toc}{\protect\addvspace{0.3\baselineskip}{\color{color1!10!white}\hrule height 0.9pt} \addvspace{0.6\baselineskip} \color{black}} \stepcounter{secnumber} $\ $\\[-1cm]
					\begin{titulo-capitulo-long}
						\begin{numero-capitulo-long}\centering
							{\fontsize{24mm}{1em}\selectfont\color{white} \Bold \thechapter}
						\end{numero-capitulo-long}\quad 
						\raisebox{2.1\cuadri}{\begin{tabular}{l}
								\fontsize{9.5mm}{1em}\selectfont \Bold \color{color1!95!black} #2\\ 
								\fontsize{9.5mm}{1em}\selectfont \Bold \color{color1!95!black} #3 \vphantom{Í} 
							\end{tabular}}
						\end{titulo-capitulo-long}
						$\ $\\[1.5cm]
						\begin{flushright}
							\begin{capitulo-descripcion}
								\large #4
							\end{capitulo-descripcion}
						\end{flushright}
						\cleardoublepage
					}
					
					
					\newcommand{\INEchaptersin}[3]{\cleardoublepage\addtocontents{toc}{\protect\addvspace{0.6\baselineskip}\color{color1!80!black}}
						
						\chapter[\texorpdfstring{\color{color1!80!black}#1}{#1}]{#2}\renewcommand{\chaptitle}{#1}\thispagestyle{empty}\addtocontents{toc}{\protect\addvspace{0.2\baselineskip}{\color{color1!10!white}\hrule height 0.9pt} \addvspace{0.6\baselineskip} \color{black}} \stepcounter{secnumber} $\ $\\[-1cm]
						\begin{hoja-capitulo}
							{$\hspace{2.6in}$\fontsize{24mm}{1em}\selectfont \Bold \thechapter  $\quad$}
							\begin{titulo-capitulo}
								\quad 
								\raisebox{2.1\cuadri}{\begin{tabular}{l}
										\fontsize{8.5mm}{1em}\selectfont \Bold \color{black} #2\\[1mm] 
										\fontsize{8.5mm}{1em}\selectfont \Bold \color{black} #3 \vphantom{Í} 
									\end{tabular}}
								\end{titulo-capitulo}
								$\ $\\[1.5cm]
							\end{hoja-capitulo}
							\cleardoublepage
						}
						
						
						
						\newcommand{\hoja}[1]{\noindent
							\begin{fondo}
								#1 
							\end{fondo}\clearpage}
						
						
						
						\newcommand{\hojados}[2]{\noindent
							\begin{fondo}
								\begin{tabular}{>{\centering\arraybackslash}p{3.19in}>{\centering\arraybackslash}p{0.01in}>{\centering\arraybackslash}p{3.19in}}
									
									\begin{columnatipoA}
										#1
									\end{columnatipoA}  & & 
									\begin{columnatipoA}
										#2
									\end{columnatipoA}
									\\ 
									
								\end{tabular} 
								
								
							\end{fondo}\clearpage}
						
						
						
						
						\newcommand{\columna}[8]{
							$\hspace{-2.1mm}$\begin{tabular}{b{0.17in}p{2.85in}}
								\color{white}\section{#1}&\\[-5.1mm] {\color{color1} \textbf{\thesection}} & \textbf{#1.}\\[-1mm]
							\end{tabular}\\[2.5mm]
							
							
							\small $\quad$#2  
							
							\vspace{3mm}
							{\color{color2}
								\hrule }
							\vspace{-1mm}
							
							\begin{center}
								{\footnotesize{\Bold#4}} \\[-1mm]
								{\scriptsize\texttwelveudash$\,\,$#5$\,\,$\texttwelveudash}
								#6\\[0mm]
							\end{center}
							$\ $\\[-6mm]
							{\scriptsize Fuente: #7}\\[-2mm]
							
							{\color{color2}
								\hrule }
							\vspace{3mm}
							
							#3 
							
							#8
						}
						
						
						
						
						
						
						\newcommand{\columnatablaprovisional}[8]{
							$\hspace{-2.1mm}$\begin{tabular}{b{0.17in}p{2.85in}}
								\color{white}\section{#1}&\\[-5.1mm] {\color{color1} \textbf{\thesection}} & \textbf{#1.}\\[-1mm]
							\end{tabular}\\[2.5mm]
							
							
							
							\vspace{-7.5mm}
							
							\begin{center}
								#6\\[0mm]
							\end{center}
							$\ $\\[-8.5mm]
							{\scriptsize Fuente: #7}\\[-2mm]
							
							
							#8
						}
						
						
						
						
						
						
						\newcommand{\hojarotada}[1]{\noindent\begin{fondo-landscape} #1 \end{fondo-landscape}\clearpage}
						
						
						
						\newcommand{\notita}[1]{\footnotetext{\color{color2}$\hspace{-2.3mm}$\scriptsize#1\\[-1.5mm]}
							
							
						}
						
						\newcommand{\notitasin}[1]{\color{color2}$\hspace{2.6mm}$\scriptsize#1\\[-1.5mm]
							
							
						}
						
						
						\newcommand{\notitatabla}[1]{\color{color2}\scriptsize#1\\[-1.5mm]
							
							
						}
						
						
						
						
						\newcommand{\cajita}[7]{\checkoddpage\ifoddpage
							\begin{bloque-media}
								\begin{titulo}
									\begin{centrador}
										\begin{tabular}{p{2.5\cuadri}p{28.8\cuadri}}
											& \\[-3mm]	
											& {\begin{titulo-subseccion}\begin{numero-subseccion}\secnumbering \end{numero-subseccion}  \phantomsection{\secnew{#1}}  \end{titulo-subseccion}}  \\[-5mm]
											&  \\[-1.3pt]
										\end{tabular}
									\end{centrador}
								\end{titulo}
								
								\begin{tabular}{p{11\cuadri}p{21\cuadri}p{5\cuadri}}
									\begin{descripcion}
										#2
									\end{descripcion} 
									& 
									\begin{columna-central}
										\begin{subtitulo}
											\centering\footnotesize{\Bold #3} \\
											\texttwelveudash$\,\,$#4$\,\,$\texttwelveudash
										\end{subtitulo}
										
										\begin{grafica}\centering
											#5
										\end{grafica}
										
										\begin{fuente}
											\footnotesize #6 
										\end{fuente}
									\end{columna-central}
									& 
									\begin{columna-marginal}
										\begin{vacio1}
											
										\end{vacio1}
										
										\begin{nota}
											#7
										\end{nota}
										
										\begin{vacio2}
											
										\end{vacio2}
									\end{columna-marginal}
									\\
								\end{tabular}
							\end{bloque-media}
							\else
							\begin{bloque-media}
								\begin{titulo}
									\begin{centrador-par}
										\begin{tabular}{p{2.5\cuadri}p{28.8\cuadri}}
											& \\[-3mm]	
											& {\begin{titulo-subseccion}\begin{numero-subseccion}\centering\secnumbering\end{numero-subseccion}  \phantomsection{\secnew{#1}}  \end{titulo-subseccion}}  \\[-5mm]
											&  \\[-1.3pt]
										\end{tabular}
									\end{centrador-par}
								\end{titulo}
								
								\begin{tabular}{p{5\cuadri}p{21.36\cuadri}p{11\cuadri}}
									\begin{columna-marginal}
										\begin{vacio1}
											
										\end{vacio1}
										
										\begin{nota}
											#7
										\end{nota}
										
										\begin{vacio2}
											
										\end{vacio2}
									\end{columna-marginal}		
									& 
									\begin{columna-central}
										\begin{subtitulo}
											\centering\footnotesize{\Bold #3} \\
											\texttwelveudash$\,\,$#4$\,\,$\texttwelveudash
										\end{subtitulo}
										
										\begin{grafica}\centering
											#5
										\end{grafica}
										
										\begin{fuente}
											\footnotesize #6 
										\end{fuente}
									\end{columna-central}
									& 
									\begin{descripcion}
										#2
									\end{descripcion} 
									\\
								\end{tabular}
							\end{bloque-media}
							\fi
						}
						
						
						







\begin{document}

%\includepdf{portadaFALTAS_T.pdf}

$\ $
\vspace{4.5cm}

\noindent\begin{tabular}{p{0.1cm}p{6.8cm}}
	& 2014.$\,$ Guatemala, América Central \\
	&\Bold Instituto Nacional de Estadística\\[-0.4cm]
	&\color{blue!50!black}\url{www.ine.gob.gt}\\[0.9cm]
\end{tabular}\\
\noindent\begin{tabular}{p{0.1cm}p{6.8cm}}
	& Está permitida la reproducción parcial o total de los contenidos de este documento con la mención de la fuente. \\[0.5cm]
	
	& Este documento fue elaborado empleando  {\Sans R}, Inkscape, Libre Office y {\Logos \XeLaTeX}.\\
\end{tabular} 


\clearpage




$\ $
\vspace{3.5cm}

\begin{center}
	\Bold \LARGE REPÚBLICA DE GUATEMALA\\
	ESTADÍSTICAS VITALES I-2014\
\end{center}
\cleardoublepage

\hoja{
	$\ $
	\vspace{0.3cm}
	
	\begin{center}
		{\Bold \LARGE AUTORIDADES}\\[0.7cm]
		
		
		{\Bold \large \color{color1!89!black} JUNTA  DIRECTIVA} \\[0.5cm]
		
		\begin{center}
			\begin{tabular}{x{6.0cm}x{6.0cm}}
				{\Bold Ministerio de Economía}   & 		{\Bold Ministerio de Finanzas}\\
				Lic. Sergio de la Torre, Titular & Lic. Dorval Carías, Titular \\
				Lic. Jacobo Rey Sigfrido Lee, Suplente & Lic. Edwin Oswaldo Martínez, Suplente \\
				& \\
				{\Bold Ministerio de Agricultura,} & {\Bold Ministerio de Energía y Minas}\\ 
				{\Bold Ganadería y Alimentación} & Lic. Erick Archila, Titular \\
				Ing. Elmer López, Titular & Licda. Ivanova Ancheta, Suplente \\
				Ing. Carlos Alfonso Anzueto, Suplente & \\
				
				& {\Bold Banco de Guatemala} \\
				{\Bold Secretaría de Planificación y} & Lic. Julio Roberto Suárez, Titular\\
				{\Bold Programación de la Presidencia} & Lic. Sergio Francisco Recinos Rivera, Suplente\\
				Licda. Ekaterina Arbolievna Parrilla, Titular & \\
				
				Licda. Dora Marina Coc, Suplente & {\Bold Universidad de San Carlos de Guatemala} \\
				
				& Ing. Murphy Olimpo Paiz, Titular \\
				{\Bold Comité Coordinador de } & Lic. Oscar René Paniagua Carrera, Suplente \\  
				{\Bold Asociaciones  Agrícolas, Comerciales, } & \\
				{\Bold Industriales y Financieras} & {\Bold Universidades Privadas} \\
				Lic. Juan Raúl Aguilar , Titular & Dr. Oscar Guillermo Peláez, Titular \\
				Lic. Oscar Augusto Sequeira, Suplente & Lic. Ariel Rivera Irías, Suplente \\[0.3cm]
			\end{tabular} 	
		\end{center}	
		
		
		
		
		
		{\Bold \large \color{color1!89!black} GERENCIA}\\[0.2cm]
		Lic. Rubén Narciso, Gerente\\
		Lic. Jaime Mejía Salguero, Subgerente Técnico\\
		Ing. Orlando Monzón, Subgerente Administrativo Financiero\\
		
		
	\end{center}
}{}
\clearpage

$\ $
\vspace{0.5cm}

\begin{center}
	{\Bold \LARGE EQUIPO RESPONSABLE}\\[1.5cm]
	
	{\Bold \large \color{color1!89!black} REVISIÓN GENERAL}\\[0.2cm]
	Rubén Narciso\\[0.8cm]
	
	
	{\Bold \large \color{color1!89!black} EQUIPO TÉCNICO}\\[0.2cm]
	Jaime Roberto Mejía Salguero\\
	Flor de María Hernández Soto\\
	Cristian Miguel Cabrera Ayala\\
	Blanca Angelica Ramirez González\\
	Marlon Humberto Pirir Garcia\\[0.8cm]
	
	{\Bold \large \color{color1!89!black} DIAGRAMACIÓN Y DISEÑO}\\[0.2cm]
	Hugo Allan García Monterrosa\\
	Fabiola Beatriz Ramírez Pinto\\
	José Carlos Bonilla Aldana\\[0.8cm]
	
	
	
\end{center}\setcounter{page}{0}\cleardoublepage



$\ $\\[0.7cm]

\tableofcontents

\cleardoublepage
\pagestyle{estandar}
\setcounter{page}{1}
\setlength{\arrayrulewidth}{1.0pt}


\cleardoublepage





$\ $\\[0.5cm]
\thispagestyle{empty}
\noindent {\Bold \LARGE Presentación}




$\ $\\



El Instituto Nacional de Estadística -INE-, consciente de la demanda de información demográfica y siendo el ente rector de la política estadística nacional en Guatemala, en cumplimiento a su Ley Orgánica, Decreto Ley 3-85, se complace en presentar el siguiente informe, que contiene las {\Bold Estadísticas Vitales}, con información correspondiente al {\Bold primer trimestre del 2014}, información esencial para la planificación del desarrollo humano.

La información presentada a continuación fue recolectada a través del Registro Nacional de las Personas  -RENAP- y consiste en los hechos ocurridos sobre nacimientos, defunciones, defunciones fetales, matrimonios y divorcios registrados en el primer trimestre del 2014. 

Sin embargo, los datos para el período son preliminares, sujetos a la adición de registros ingresados tardíamente.

Por lo tanto, el INE se complace en presentar este informe, con el propósito de brindar una herramienta más de análisis a la población guatemalteca, y a la vez agradece el aporte y colaboración del Registro Nacional de las Personas, a quien se insta  a continuar con el apoyo a este proceso.


\thispagestyle{empty}


\cleardoublepage

%---------------------------------presentación
%quité la presentación, para fines de trabajar un documento compacto



%--------------------------CONTENIDOS---------------------------





\INEchapter{Comercio Total}{Comercio Total}{}{\quad ¿Qué es comercio total?}

\hojados{\columna{Exportaciones totales}{Las exportaciones del primer trimestre 2014  revelan un crecimiento de 4\% comparado con el primer trimestre 2013.  Según la serie histórica el primer trimestre del 2013 tuvo una disminución de 1\% comparado al 2012 y este a su vez manifiesta tambien una diferencia de -3\%  al 2011.  No así el 2011 manifiesta un crecimiento de 26\% comparado a 2010}{}{Exportaciones Trimestrales Comercio Total    2011-2014}{(cifras preliminares en US\$)}{\ \\[6mm]\begin{tikzpicture}[x=1pt,y=1pt,scale=1]  % Created by tikzDevice version 0.7.0 on 2014-12-15 22:10:25
% !TEX encoding = UTF-8 Unicode
\definecolor[named]{fillColor}{rgb}{1.00,1.00,1.00}
\path[use as bounding box,fill=fillColor,fill opacity=0.00] (0,0) rectangle (230.54,138.04);
\begin{scope}
\path[clip] (  0.00,  0.00) rectangle (230.54,138.04);
\definecolor[named]{drawColor}{rgb}{1.00,1.00,1.00}

\path[draw=drawColor,line width= 0.6pt,line join=round,line cap=round] ( -0.00,  0.00) rectangle (230.54,138.04);
\end{scope}
\begin{scope}
\path[clip] (  0.00,  0.00) rectangle (230.54,138.04);

\path[] (  1.50, 42.66) rectangle (222.01,130.92);

\path[] (  1.50, 56.19) --
	(222.01, 56.19);

\path[] (  1.50, 71.30) --
	(222.01, 71.30);

\path[] (  1.50, 86.41) --
	(222.01, 86.41);

\path[] (  1.50,101.52) --
	(222.01,101.52);

\path[] (  1.50,116.64) --
	(222.01,116.64);

\path[] (  1.50, 48.64) --
	(222.01, 48.64);

\path[] (  1.50, 63.75) --
	(222.01, 63.75);

\path[] (  1.50, 78.86) --
	(222.01, 78.86);

\path[] (  1.50, 93.97) --
	(222.01, 93.97);

\path[] (  1.50,109.08) --
	(222.01,109.08);

\path[] (  1.50,124.19) --
	(222.01,124.19);

\path[] ( 11.52, 42.66) --
	( 11.52,130.92);

\path[] ( 28.23, 42.66) --
	( 28.23,130.92);

\path[] ( 44.93, 42.66) --
	( 44.93,130.92);

\path[] ( 61.64, 42.66) --
	( 61.64,130.92);

\path[] ( 78.34, 42.66) --
	( 78.34,130.92);

\path[] ( 95.05, 42.66) --
	( 95.05,130.92);

\path[] (111.75, 42.66) --
	(111.75,130.92);

\path[] (128.46, 42.66) --
	(128.46,130.92);

\path[] (145.16, 42.66) --
	(145.16,130.92);

\path[] (161.87, 42.66) --
	(161.87,130.92);

\path[] (178.57, 42.66) --
	(178.57,130.92);

\path[] (195.28, 42.66) --
	(195.28,130.92);

\path[] (211.98, 42.66) --
	(211.98,130.92);
\definecolor[named]{drawColor}{rgb}{0.00,0.00,0.00}

\path[draw=drawColor,line width= 1.9pt,line join=round] ( 11.52,126.91) --
	( 28.23,124.49) --
	( 44.93, 93.06) --
	( 61.64, 92.00) --
	( 78.34,114.82) --
	( 95.05,105.60) --
	(111.75, 78.71) --
	(128.46, 73.42) --
	(145.16,109.99) --
	(161.87,118.45) --
	(178.57, 73.57) --
	(195.28, 78.40) --
	(211.98,124.04);

\node[text=drawColor,anchor=base,inner sep=0pt, outer sep=0pt, scale=  0.91] at ( 11.52,130.03) {2,718};

\node[text=drawColor,anchor=base west,inner sep=0pt, outer sep=0pt, scale=  0.91] at ( 28.23,127.62) {2,702};

\node[text=drawColor,anchor=base west,inner sep=0pt, outer sep=0pt, scale=  0.91] at ( 44.93, 96.19) {2,494};

\node[text=drawColor,anchor=base,inner sep=0pt, outer sep=0pt, scale=  0.91] at ( 61.64, 82.63) {2,487};

\node[text=drawColor,anchor=base,inner sep=0pt, outer sep=0pt, scale=  0.91] at ( 78.34,117.95) {2,638};

\node[text=drawColor,anchor=base west,inner sep=0pt, outer sep=0pt, scale=  0.91] at ( 95.05,108.73) {2,577};

\node[text=drawColor,anchor=base west,inner sep=0pt, outer sep=0pt, scale=  0.91] at (111.75, 81.83) {2,399};

\node[text=drawColor,anchor=base,inner sep=0pt, outer sep=0pt, scale=  0.91] at (128.46, 64.05) {2,364};

\node[text=drawColor,anchor=base east,inner sep=0pt, outer sep=0pt, scale=  0.91] at (141.03,109.99) {2,606};

\node[text=drawColor,anchor=base,inner sep=0pt, outer sep=0pt, scale=  0.91] at (161.87,121.57) {2,662};

\node[text=drawColor,anchor=base,inner sep=0pt, outer sep=0pt, scale=  0.91] at (178.57, 64.20) {2,365};

\node[text=drawColor,anchor=base east,inner sep=0pt, outer sep=0pt, scale=  0.91] at (191.15, 78.40) {2,397};

\node[text=drawColor,anchor=base,inner sep=0pt, outer sep=0pt, scale=  0.91] at (211.98,127.16) {2,699};
\end{scope}
\begin{scope}
\path[clip] (  0.00,  0.00) rectangle (230.54,138.04);

\path[] (  1.50, 42.66) --
	(  1.50,130.92);
\end{scope}
\begin{scope}
\path[clip] (  0.00,  0.00) rectangle (230.54,138.04);

\path[] (  0.00, 48.64) --
	(  1.50, 48.64);

\path[] (  0.00, 63.75) --
	(  1.50, 63.75);

\path[] (  0.00, 78.86) --
	(  1.50, 78.86);

\path[] (  0.00, 93.97) --
	(  1.50, 93.97);

\path[] (  0.00,109.08) --
	(  1.50,109.08);

\path[] (  0.00,124.19) --
	(  1.50,124.19);
\end{scope}
\begin{scope}
\path[clip] (  0.00,  0.00) rectangle (230.54,138.04);
\definecolor[named]{drawColor}{rgb}{0.00,0.00,0.00}

\path[draw=drawColor,line width= 0.6pt,line join=round] (  1.50, 42.66) --
	(222.01, 42.66);
\end{scope}
\begin{scope}
\path[clip] (  0.00,  0.00) rectangle (230.54,138.04);

\path[] ( 11.52, 38.39) --
	( 11.52, 42.66);

\path[] ( 28.23, 38.39) --
	( 28.23, 42.66);

\path[] ( 44.93, 38.39) --
	( 44.93, 42.66);

\path[] ( 61.64, 38.39) --
	( 61.64, 42.66);

\path[] ( 78.34, 38.39) --
	( 78.34, 42.66);

\path[] ( 95.05, 38.39) --
	( 95.05, 42.66);

\path[] (111.75, 38.39) --
	(111.75, 42.66);

\path[] (128.46, 38.39) --
	(128.46, 42.66);

\path[] (145.16, 38.39) --
	(145.16, 42.66);

\path[] (161.87, 38.39) --
	(161.87, 42.66);

\path[] (178.57, 38.39) --
	(178.57, 42.66);

\path[] (195.28, 38.39) --
	(195.28, 42.66);

\path[] (211.98, 38.39) --
	(211.98, 42.66);
\end{scope}
\begin{scope}
\path[clip] (  0.00,  0.00) rectangle (230.54,138.04);
\definecolor[named]{drawColor}{rgb}{0.00,0.00,0.00}

\node[text=drawColor,rotate= 90.00,anchor=base east,inner sep=0pt, outer sep=0pt, scale=  1.00] at ( 14.97, 35.55) {T1-2011};

\node[text=drawColor,rotate= 90.00,anchor=base east,inner sep=0pt, outer sep=0pt, scale=  1.00] at ( 31.67, 35.55) {T2-2011};

\node[text=drawColor,rotate= 90.00,anchor=base east,inner sep=0pt, outer sep=0pt, scale=  1.00] at ( 48.38, 35.55) {T3-2011};

\node[text=drawColor,rotate= 90.00,anchor=base east,inner sep=0pt, outer sep=0pt, scale=  1.00] at ( 65.08, 35.55) {T4-2011};

\node[text=drawColor,rotate= 90.00,anchor=base east,inner sep=0pt, outer sep=0pt, scale=  1.00] at ( 81.79, 35.55) {T1-2012};

\node[text=drawColor,rotate= 90.00,anchor=base east,inner sep=0pt, outer sep=0pt, scale=  1.00] at ( 98.49, 35.55) {T2-2012};

\node[text=drawColor,rotate= 90.00,anchor=base east,inner sep=0pt, outer sep=0pt, scale=  1.00] at (115.20, 35.55) {T3-2012};

\node[text=drawColor,rotate= 90.00,anchor=base east,inner sep=0pt, outer sep=0pt, scale=  1.00] at (131.90, 35.55) {T4-2012};

\node[text=drawColor,rotate= 90.00,anchor=base east,inner sep=0pt, outer sep=0pt, scale=  1.00] at (148.61, 35.55) {T1-2013};

\node[text=drawColor,rotate= 90.00,anchor=base east,inner sep=0pt, outer sep=0pt, scale=  1.00] at (165.31, 35.55) {T2-2013};

\node[text=drawColor,rotate= 90.00,anchor=base east,inner sep=0pt, outer sep=0pt, scale=  1.00] at (182.02, 35.55) {T3-2013};

\node[text=drawColor,rotate= 90.00,anchor=base east,inner sep=0pt, outer sep=0pt, scale=  1.00] at (198.72, 35.55) {T4-2013};

\node[text=drawColor,rotate= 90.00,anchor=base east,inner sep=0pt, outer sep=0pt, scale=  1.00] at (215.43, 35.55) {T1-2014};
\end{scope}
  \end{tikzpicture}}{INE, con datos del BANGUAT.}{\notitasin{Los datos del año 2014 se presentan como preliminares y serán ajustados por el registro tardío de los mismos.}}}
{\columna{Variación porcentual de las exportaciones, mismo trimestre año anterior}{La gráfica muestra la variación porcentual en las exportaciones de la serie histórica  2010 a 2014.  se observa que a partir del primer trimestre 2012 una baja en las exportaciones de Guatemala con el resto del mundo, teniendo una leve recuperación el el  segundo y cuarto trimestre 2013, seguido del primer trimestre 2014.}{}{Exportaciones Comercio Total (Variación porcentual) 2010-2014}{( mismo trimestre años anteriores)}{\ \\[6mm]\begin{tikzpicture}[x=1pt,y=1pt,scale=1]  % Created by tikzDevice version 0.7.0 on 2014-12-18 22:38:43
% !TEX encoding = UTF-8 Unicode
\definecolor[named]{fillColor}{rgb}{1.00,1.00,1.00}
\path[use as bounding box,fill=fillColor,fill opacity=0.00] (0,0) rectangle (230.54,138.04);
\begin{scope}
\path[clip] (  0.00,  0.00) rectangle (230.54,138.04);
\definecolor[named]{drawColor}{rgb}{1.00,1.00,1.00}

\path[draw=drawColor,line width= 0.6pt,line join=round,line cap=round] (  0.00,  0.00) rectangle (230.54,138.04);
\end{scope}
\begin{scope}
\path[clip] (  0.00,  0.00) rectangle (230.54,138.04);

\path[] (  6.05, 49.31) rectangle (230.54,113.00);

\path[] ( 12.12, 49.31) --
	( 12.12,113.00);

\path[] ( 22.23, 49.31) --
	( 22.23,113.00);

\path[] ( 32.34, 49.31) --
	( 32.34,113.00);

\path[] ( 42.45, 49.31) --
	( 42.45,113.00);

\path[] ( 52.56, 49.31) --
	( 52.56,113.00);

\path[] ( 62.68, 49.31) --
	( 62.68,113.00);

\path[] ( 72.79, 49.31) --
	( 72.79,113.00);

\path[] ( 82.90, 49.31) --
	( 82.90,113.00);

\path[] ( 93.01, 49.31) --
	( 93.01,113.00);

\path[] (103.13, 49.31) --
	(103.13,113.00);

\path[] (113.24, 49.31) --
	(113.24,113.00);

\path[] (123.35, 49.31) --
	(123.35,113.00);

\path[] (133.46, 49.31) --
	(133.46,113.00);

\path[] (143.58, 49.31) --
	(143.58,113.00);

\path[] (153.69, 49.31) --
	(153.69,113.00);

\path[] (163.80, 49.31) --
	(163.80,113.00);

\path[] (173.91, 49.31) --
	(173.91,113.00);

\path[] (184.02, 49.31) --
	(184.02,113.00);

\path[] (194.14, 49.31) --
	(194.14,113.00);

\path[] (204.25, 49.31) --
	(204.25,113.00);

\path[] (214.36, 49.31) --
	(214.36,113.00);

\path[] (224.47, 49.31) --
	(224.47,113.00);
\definecolor[named]{drawColor}{rgb}{0.00,0.00,0.00}

\path[draw=drawColor,line width= 0.6pt,line join=round] (  9.08, 49.31) rectangle ( 15.15,113.00);

\path[draw=drawColor,line width= 0.6pt,line join=round] ( 19.19, 49.31) rectangle ( 25.26, 85.13);

\path[draw=drawColor,line width= 0.6pt,line join=round] ( 29.31, 49.31) rectangle ( 35.37, 79.53);

\path[draw=drawColor,line width= 0.6pt,line join=round] ( 39.42, 49.31) rectangle ( 45.49, 78.70);

\path[draw=drawColor,line width= 0.6pt,line join=round] ( 49.53, 49.31) rectangle ( 55.60, 78.23);

\path[draw=drawColor,line width= 0.6pt,line join=round] ( 59.64, 49.31) rectangle ( 65.71, 68.29);

\path[draw=drawColor,line width= 0.6pt,line join=round] ( 69.76, 49.31) rectangle ( 75.82, 64.72);

\path[draw=drawColor,line width= 0.6pt,line join=round] ( 79.87, 49.31) rectangle ( 85.94, 64.33);

\path[draw=drawColor,line width= 0.6pt,line join=round] ( 89.98, 49.31) rectangle ( 96.05, 63.02);

\path[draw=drawColor,line width= 0.6pt,line join=round] (100.09, 49.31) rectangle (106.16, 62.74);

\path[draw=drawColor,line width= 0.6pt,line join=round] (110.20, 49.31) rectangle (116.27, 61.06);

\path[draw=drawColor,line width= 0.6pt,line join=round] (120.32, 49.31) rectangle (126.38, 60.04);

\path[draw=drawColor,line width= 0.6pt,line join=round] (130.43, 49.31) rectangle (136.50, 59.83);

\path[draw=drawColor,line width= 0.6pt,line join=round] (140.54, 49.31) rectangle (146.61, 59.35);

\path[draw=drawColor,line width= 0.6pt,line join=round] (150.65, 49.31) rectangle (156.72, 58.41);

\path[draw=drawColor,line width= 0.6pt,line join=round] (160.77, 49.31) rectangle (166.83, 57.94);

\path[draw=drawColor,line width= 0.6pt,line join=round] (170.88, 49.31) rectangle (176.95, 57.81);

\path[draw=drawColor,line width= 0.6pt,line join=round] (180.99, 49.31) rectangle (187.06, 57.41);

\path[draw=drawColor,line width= 0.6pt,line join=round] (191.10, 49.31) rectangle (197.17, 57.35);

\path[draw=drawColor,line width= 0.6pt,line join=round] (201.22, 49.31) rectangle (207.28, 55.88);

\path[draw=drawColor,line width= 0.6pt,line join=round] (211.33, 49.31) rectangle (217.40, 54.50);

\path[draw=drawColor,line width= 0.6pt,line join=round] (221.44, 49.31) rectangle (227.51, 53.18);

\node[text=drawColor,rotate= 90.00,anchor=base west,inner sep=0pt, outer sep=0pt, scale=  0.85] at ( 15.15,114.87) {16,065};

\node[text=drawColor,rotate= 90.00,anchor=base west,inner sep=0pt, outer sep=0pt, scale=  0.85] at ( 25.26, 86.67) {9,035};

\node[text=drawColor,rotate= 90.00,anchor=base west,inner sep=0pt, outer sep=0pt, scale=  0.85] at ( 35.38, 81.06) {7,622};

\node[text=drawColor,rotate= 90.00,anchor=base west,inner sep=0pt, outer sep=0pt, scale=  0.85] at ( 45.49, 80.23) {7,412};

\node[text=drawColor,rotate= 90.00,anchor=base west,inner sep=0pt, outer sep=0pt, scale=  0.85] at ( 55.60, 79.76) {7,294};

\node[text=drawColor,rotate= 90.00,anchor=base west,inner sep=0pt, outer sep=0pt, scale=  0.85] at ( 65.71, 69.83) {4,788};

\node[text=drawColor,rotate= 90.00,anchor=base west,inner sep=0pt, outer sep=0pt, scale=  0.85] at ( 75.82, 66.26) {3,887};

\node[text=drawColor,rotate= 90.00,anchor=base west,inner sep=0pt, outer sep=0pt, scale=  0.85] at ( 85.94, 65.86) {3,787};

\node[text=drawColor,rotate= 90.00,anchor=base west,inner sep=0pt, outer sep=0pt, scale=  0.85] at ( 96.05, 64.56) {3,458};

\node[text=drawColor,rotate= 90.00,anchor=base west,inner sep=0pt, outer sep=0pt, scale=  0.85] at (106.16, 64.28) {3,387};

\node[text=drawColor,rotate= 90.00,anchor=base west,inner sep=0pt, outer sep=0pt, scale=  0.85] at (116.27, 62.60) {2,963};

\node[text=drawColor,rotate= 90.00,anchor=base west,inner sep=0pt, outer sep=0pt, scale=  0.85] at (126.39, 61.58) {2,706};

\node[text=drawColor,rotate= 90.00,anchor=base west,inner sep=0pt, outer sep=0pt, scale=  0.85] at (136.50, 61.36) {2,652};

\node[text=drawColor,rotate= 90.00,anchor=base west,inner sep=0pt, outer sep=0pt, scale=  0.85] at (146.61, 60.88) {2,531};

\node[text=drawColor,rotate= 90.00,anchor=base west,inner sep=0pt, outer sep=0pt, scale=  0.85] at (156.72, 59.94) {2,294};

\node[text=drawColor,rotate= 90.00,anchor=base west,inner sep=0pt, outer sep=0pt, scale=  0.85] at (166.84, 59.48) {2,176};

\node[text=drawColor,rotate= 90.00,anchor=base west,inner sep=0pt, outer sep=0pt, scale=  0.85] at (176.95, 59.34) {2,143};

\node[text=drawColor,rotate= 90.00,anchor=base west,inner sep=0pt, outer sep=0pt, scale=  0.85] at (187.06, 58.95) {2,043};

\node[text=drawColor,rotate= 90.00,anchor=base west,inner sep=0pt, outer sep=0pt, scale=  0.85] at (197.17, 58.88) {2,026};

\node[text=drawColor,rotate= 90.00,anchor=base west,inner sep=0pt, outer sep=0pt, scale=  0.85] at (207.28, 57.42) {1,657};

\node[text=drawColor,rotate= 90.00,anchor=base west,inner sep=0pt, outer sep=0pt, scale=  0.85] at (217.40, 56.03) {1,307};

\node[text=drawColor,rotate= 90.00,anchor=base west,inner sep=0pt, outer sep=0pt, scale=  0.85] at (227.51, 54.20) {976};
\end{scope}
\begin{scope}
\path[clip] (  0.00,  0.00) rectangle (230.54,138.04);

\path[] (  6.05, 49.31) --
	(  6.05,113.00);
\end{scope}
\begin{scope}
\path[clip] (  0.00,  0.00) rectangle (230.54,138.04);
\definecolor[named]{drawColor}{rgb}{0.00,0.00,0.00}

\path[draw=drawColor,line width= 0.6pt,line join=round] (  6.05, 49.31) --
	(230.54, 49.31);
\end{scope}
\begin{scope}
\path[clip] (  0.00,  0.00) rectangle (230.54,138.04);

\path[] ( 12.12, 45.05) --
	( 12.12, 49.31);

\path[] ( 22.23, 45.05) --
	( 22.23, 49.31);

\path[] ( 32.34, 45.05) --
	( 32.34, 49.31);

\path[] ( 42.45, 45.05) --
	( 42.45, 49.31);

\path[] ( 52.56, 45.05) --
	( 52.56, 49.31);

\path[] ( 62.68, 45.05) --
	( 62.68, 49.31);

\path[] ( 72.79, 45.05) --
	( 72.79, 49.31);

\path[] ( 82.90, 45.05) --
	( 82.90, 49.31);

\path[] ( 93.01, 45.05) --
	( 93.01, 49.31);

\path[] (103.13, 45.05) --
	(103.13, 49.31);

\path[] (113.24, 45.05) --
	(113.24, 49.31);

\path[] (123.35, 45.05) --
	(123.35, 49.31);

\path[] (133.46, 45.05) --
	(133.46, 49.31);

\path[] (143.58, 45.05) --
	(143.58, 49.31);

\path[] (153.69, 45.05) --
	(153.69, 49.31);

\path[] (163.80, 45.05) --
	(163.80, 49.31);

\path[] (173.91, 45.05) --
	(173.91, 49.31);

\path[] (184.02, 45.05) --
	(184.02, 49.31);

\path[] (194.14, 45.05) --
	(194.14, 49.31);

\path[] (204.25, 45.05) --
	(204.25, 49.31);

\path[] (214.36, 45.05) --
	(214.36, 49.31);

\path[] (224.47, 45.05) --
	(224.47, 49.31);
\end{scope}
\begin{scope}
\path[clip] (  0.00,  0.00) rectangle (230.54,138.04);
\definecolor[named]{drawColor}{rgb}{0.00,0.00,0.00}

\node[text=drawColor,rotate= 90.00,anchor=base east,inner sep=0pt, outer sep=0pt, scale=  1.00] at ( 15.68, 42.20) {Guatemala};

\node[text=drawColor,rotate= 90.00,anchor=base east,inner sep=0pt, outer sep=0pt, scale=  1.00] at ( 25.80, 42.20) {Huehuetenango};

\node[text=drawColor,rotate= 90.00,anchor=base east,inner sep=0pt, outer sep=0pt, scale=  1.00] at ( 35.91, 42.20) {Alta Verapaz};

\node[text=drawColor,rotate= 90.00,anchor=base east,inner sep=0pt, outer sep=0pt, scale=  1.00] at ( 46.02, 42.20) {Quiche};

\node[text=drawColor,rotate= 90.00,anchor=base east,inner sep=0pt, outer sep=0pt, scale=  1.00] at ( 56.13, 42.20) {San Marcos};

\node[text=drawColor,rotate= 90.00,anchor=base east,inner sep=0pt, outer sep=0pt, scale=  1.00] at ( 66.25, 42.20) {Quetzaltenango};

\node[text=drawColor,rotate= 90.00,anchor=base east,inner sep=0pt, outer sep=0pt, scale=  1.00] at ( 76.36, 42.20) {Chimaltenango};

\node[text=drawColor,rotate= 90.00,anchor=base east,inner sep=0pt, outer sep=0pt, scale=  1.00] at ( 86.47, 42.20) {Escuintla};

\node[text=drawColor,rotate= 90.00,anchor=base east,inner sep=0pt, outer sep=0pt, scale=  1.00] at ( 96.58, 42.20) {Peten};

\node[text=drawColor,rotate= 90.00,anchor=base east,inner sep=0pt, outer sep=0pt, scale=  1.00] at (106.70, 42.20) {Suchitepequez};

\node[text=drawColor,rotate= 90.00,anchor=base east,inner sep=0pt, outer sep=0pt, scale=  1.00] at (116.81, 42.20) {Totonicapan};

\node[text=drawColor,rotate= 90.00,anchor=base east,inner sep=0pt, outer sep=0pt, scale=  1.00] at (126.92, 42.20) {Chiquimula};

\node[text=drawColor,rotate= 90.00,anchor=base east,inner sep=0pt, outer sep=0pt, scale=  1.00] at (137.03, 42.20) {Jutiapa};

\node[text=drawColor,rotate= 90.00,anchor=base east,inner sep=0pt, outer sep=0pt, scale=  1.00] at (147.14, 42.20) {Solola};

\node[text=drawColor,rotate= 90.00,anchor=base east,inner sep=0pt, outer sep=0pt, scale=  1.00] at (157.26, 42.20) {Jalapa};

\node[text=drawColor,rotate= 90.00,anchor=base east,inner sep=0pt, outer sep=0pt, scale=  1.00] at (167.37, 42.20) {Izabal};

\node[text=drawColor,rotate= 90.00,anchor=base east,inner sep=0pt, outer sep=0pt, scale=  1.00] at (177.48, 42.20) {Santa Rosa};

\node[text=drawColor,rotate= 90.00,anchor=base east,inner sep=0pt, outer sep=0pt, scale=  1.00] at (187.59, 42.20) {Baja Verapaz};

\node[text=drawColor,rotate= 90.00,anchor=base east,inner sep=0pt, outer sep=0pt, scale=  1.00] at (197.71, 42.20) {Retalhuleu};

\node[text=drawColor,rotate= 90.00,anchor=base east,inner sep=0pt, outer sep=0pt, scale=  1.00] at (207.82, 42.20) {Sacatepequez};

\node[text=drawColor,rotate= 90.00,anchor=base east,inner sep=0pt, outer sep=0pt, scale=  1.00] at (217.93, 42.20) {Zacapa};

\node[text=drawColor,rotate= 90.00,anchor=base east,inner sep=0pt, outer sep=0pt, scale=  1.00] at (228.04, 42.20) {El Progreso};
\end{scope}
  \end{tikzpicture}}{INE, con datos del BANGUAT.}{\notitasin{Los datos del año 2014 se presentan como preliminares y serán ajustados por el registro tardío de los mismos.}}}
\hojados{\columna{Importaciones totales}{Las importaciones muestran un comportamiento de crecimiento a partir del primer trimestre 2011 con un 26\%,  se observan valores negativos en el segundo y tercer trimestre del 2012,  En el primer trimestre 2013 se advierte nuevamente una recuperación de 1\%   y el 2014 este porcentaje  se eleva a 5\%.}{}{Importaciones Trimestrales Comercio Total    2011-2014}{(cifras preliminares en US\$)}{\ \\[6mm]\begin{tikzpicture}[x=1pt,y=1pt,scale=1]  % Created by tikzDevice version 0.7.0 on 2014-12-18 20:42:30
% !TEX encoding = UTF-8 Unicode
\definecolor[named]{fillColor}{rgb}{1.00,1.00,1.00}
\path[use as bounding box,fill=fillColor,fill opacity=0.00] (0,0) rectangle (230.54,138.04);
\begin{scope}
\path[clip] (  0.00,  0.00) rectangle (230.54,138.04);
\definecolor[named]{drawColor}{rgb}{1.00,1.00,1.00}

\path[draw=drawColor,line width= 0.6pt,line join=round,line cap=round] (  0.00,  0.00) rectangle (230.54,138.04);
\end{scope}
\begin{scope}
\path[clip] (  0.00,  0.00) rectangle (230.54,138.04);

\path[] (  6.05, 26.76) rectangle (230.54,123.81);

\path[] ( 20.69, 26.76) --
	( 20.69,123.81);

\path[] ( 45.09, 26.76) --
	( 45.09,123.81);

\path[] ( 69.49, 26.76) --
	( 69.49,123.81);

\path[] ( 93.89, 26.76) --
	( 93.89,123.81);

\path[] (118.29, 26.76) --
	(118.29,123.81);

\path[] (142.70, 26.76) --
	(142.70,123.81);

\path[] (167.10, 26.76) --
	(167.10,123.81);

\path[] (191.50, 26.76) --
	(191.50,123.81);

\path[] (215.90, 26.76) --
	(215.90,123.81);
\definecolor[named]{drawColor}{rgb}{0.00,0.00,0.00}

\path[draw=drawColor,line width= 0.6pt,line join=round] ( 13.37, 26.76) rectangle ( 28.01, 88.39);

\path[draw=drawColor,line width= 0.6pt,line join=round] ( 37.77, 26.76) rectangle ( 52.41,123.81);

\path[draw=drawColor,line width= 0.6pt,line join=round] ( 62.17, 26.76) rectangle ( 76.81,102.00);

\path[draw=drawColor,line width= 0.6pt,line join=round] ( 86.57, 26.76) rectangle (101.21, 78.23);

\path[draw=drawColor,line width= 0.6pt,line join=round] (110.97, 26.76) rectangle (125.62, 55.89);

\path[draw=drawColor,line width= 0.6pt,line join=round] (135.38, 26.76) rectangle (150.02, 37.00);

\path[draw=drawColor,line width= 0.6pt,line join=round] (159.78, 26.76) rectangle (174.42, 27.70);

\path[draw=drawColor,line width= 0.6pt,line join=round] (184.18, 26.76) rectangle (198.82, 26.88);
\definecolor[named]{drawColor}{rgb}{0.78,0.78,0.78}
\definecolor[named]{fillColor}{rgb}{0.78,0.78,0.78}

\path[draw=drawColor,line width= 0.6pt,line join=round,fill=fillColor] (208.58, 26.76) rectangle (223.22, 26.86);
\definecolor[named]{drawColor}{rgb}{0.00,0.00,0.00}

\node[text=drawColor,anchor=base,inner sep=0pt, outer sep=0pt, scale=  0.85] at ( 20.69, 91.43) {17,363};

\node[text=drawColor,anchor=base,inner sep=0pt, outer sep=0pt, scale=  0.85] at ( 45.09,126.84) {27,340};

\node[text=drawColor,anchor=base,inner sep=0pt, outer sep=0pt, scale=  0.85] at ( 69.49,105.04) {21,196};

\node[text=drawColor,anchor=base,inner sep=0pt, outer sep=0pt, scale=  0.85] at ( 93.89, 81.26) {14,499};

\node[text=drawColor,anchor=base,inner sep=0pt, outer sep=0pt, scale=  0.85] at (118.29, 58.92) {8,205};

\node[text=drawColor,anchor=base,inner sep=0pt, outer sep=0pt, scale=  0.85] at (142.70, 40.04) {2,886};

\node[text=drawColor,anchor=base,inner sep=0pt, outer sep=0pt, scale=  0.85] at (167.10, 30.74) {265};

\node[text=drawColor,anchor=base,inner sep=0pt, outer sep=0pt, scale=  0.85] at (191.50, 29.91) {33};

\node[text=drawColor,anchor=base,inner sep=0pt, outer sep=0pt, scale=  0.85] at (215.90, 29.89) {27};
\end{scope}
\begin{scope}
\path[clip] (  0.00,  0.00) rectangle (230.54,138.04);

\path[] (  6.05, 26.76) --
	(  6.05,123.81);
\end{scope}
\begin{scope}
\path[clip] (  0.00,  0.00) rectangle (230.54,138.04);
\definecolor[named]{drawColor}{rgb}{0.00,0.00,0.00}

\path[draw=drawColor,line width= 0.6pt,line join=round] (  6.05, 26.76) --
	(230.54, 26.76);
\end{scope}
\begin{scope}
\path[clip] (  0.00,  0.00) rectangle (230.54,138.04);

\path[] ( 20.69, 22.49) --
	( 20.69, 26.76);

\path[] ( 45.09, 22.49) --
	( 45.09, 26.76);

\path[] ( 69.49, 22.49) --
	( 69.49, 26.76);

\path[] ( 93.89, 22.49) --
	( 93.89, 26.76);

\path[] (118.29, 22.49) --
	(118.29, 26.76);

\path[] (142.70, 22.49) --
	(142.70, 26.76);

\path[] (167.10, 22.49) --
	(167.10, 26.76);

\path[] (191.50, 22.49) --
	(191.50, 26.76);

\path[] (215.90, 22.49) --
	(215.90, 26.76);
\end{scope}
\begin{scope}
\path[clip] (  0.00,  0.00) rectangle (230.54,138.04);
\definecolor[named]{drawColor}{rgb}{0.00,0.00,0.00}

\node[text=drawColor,rotate= 90.00,anchor=base east,inner sep=0pt, outer sep=0pt, scale=  1.00] at ( 24.26, 19.65) {15 a 19};

\node[text=drawColor,rotate= 90.00,anchor=base east,inner sep=0pt, outer sep=0pt, scale=  1.00] at ( 48.66, 19.65) {20 a 24};

\node[text=drawColor,rotate= 90.00,anchor=base east,inner sep=0pt, outer sep=0pt, scale=  1.00] at ( 73.06, 19.65) {25 a 29};

\node[text=drawColor,rotate= 90.00,anchor=base east,inner sep=0pt, outer sep=0pt, scale=  1.00] at ( 97.46, 19.65) {30 a 34};

\node[text=drawColor,rotate= 90.00,anchor=base east,inner sep=0pt, outer sep=0pt, scale=  1.00] at (121.86, 19.65) {35 a 39};

\node[text=drawColor,rotate= 90.00,anchor=base east,inner sep=0pt, outer sep=0pt, scale=  1.00] at (146.27, 19.65) {40 a 44};

\node[text=drawColor,rotate= 90.00,anchor=base east,inner sep=0pt, outer sep=0pt, scale=  1.00] at (170.67, 19.65) {45 a 49};

\node[text=drawColor,rotate= 90.00,anchor=base east,inner sep=0pt, outer sep=0pt, scale=  1.00] at (195.07, 19.65) {50 ó más};

\node[text=drawColor,rotate= 90.00,anchor=base east,inner sep=0pt, outer sep=0pt, scale=  1.00] at (219.47, 19.65) {Ignorado};
\end{scope}
  \end{tikzpicture}}{INE, con datos del BANGUAT.}{\notitasin{Los datos del año 2014 se presentan como preliminares y serán ajustados por el registro tardío de los mismos.}}}
{\columna{Variación porcentual de las importaciones, mismo trimestre año anterior}{La gráfica muestra la variación porcentual en las Importaciones de la serie histórica  2011 a 2014.  Se observa que a partir del primer trimestre 2012 una baja en las Importaciones a Guatemala del resto del mundo, la serie se recupera nuevamente a partir del tercer trimestre del 2012,  trimestre 2013,  mostrando trimestres con intervalos de baja y  crecimiento intercalados, primer trimestre  2014 reporta un porcentaje de 5\%.}{}{Importaciones Comercio Total (Variación porcentual) 2011-2014}{( mismo trimestre años anteriores)}{\ \\[6mm]\begin{tikzpicture}[x=1pt,y=1pt,scale=1]  % Created by tikzDevice version 0.7.0 on 2014-12-16 17:45:30
% !TEX encoding = UTF-8 Unicode
\definecolor[named]{fillColor}{rgb}{1.00,1.00,1.00}
\path[use as bounding box,fill=fillColor,fill opacity=0.00] (0,0) rectangle (230.54,138.04);
\begin{scope}
\path[clip] (  0.00,  0.00) rectangle (230.54,138.04);
\definecolor[named]{drawColor}{rgb}{1.00,1.00,1.00}

\path[draw=drawColor,line width= 0.6pt,line join=round,line cap=round] (  0.00,  0.00) rectangle (230.54,138.04);
\end{scope}
\begin{scope}
\path[clip] (  0.00,  0.00) rectangle (230.54,138.04);

\path[] (  5.80, 26.91) rectangle (230.54,123.81);

\path[] ( 19.02, 26.91) --
	( 19.02,123.81);

\path[] ( 41.05, 26.91) --
	( 41.05,123.81);

\path[] ( 63.08, 26.91) --
	( 63.08,123.81);

\path[] ( 85.12, 26.91) --
	( 85.12,123.81);

\path[] (107.15, 26.91) --
	(107.15,123.81);

\path[] (129.19, 26.91) --
	(129.19,123.81);

\path[] (151.22, 26.91) --
	(151.22,123.81);

\path[] (173.25, 26.91) --
	(173.25,123.81);

\path[] (195.29, 26.91) --
	(195.29,123.81);

\path[] (217.32, 26.91) --
	(217.32,123.81);
\definecolor[named]{drawColor}{rgb}{0.00,0.00,0.00}

\path[draw=drawColor,line width= 0.6pt,line join=round] ( 12.41, 26.91) rectangle ( 25.63, 26.95);

\path[draw=drawColor,line width= 0.6pt,line join=round] ( 34.44, 26.91) rectangle ( 47.66, 26.95);

\path[draw=drawColor,line width= 0.6pt,line join=round] ( 56.47, 26.91) rectangle ( 69.69, 27.00);

\path[draw=drawColor,line width= 0.6pt,line join=round] ( 78.51, 26.91) rectangle ( 91.73, 28.08);

\path[draw=drawColor,line width= 0.6pt,line join=round] (100.54, 26.91) rectangle (113.76, 33.66);

\path[draw=drawColor,line width= 0.6pt,line join=round] (122.58, 26.91) rectangle (135.80, 46.94);

\path[draw=drawColor,line width= 0.6pt,line join=round] (144.61, 26.91) rectangle (157.83, 71.20);

\path[draw=drawColor,line width= 0.6pt,line join=round] (166.64, 26.91) rectangle (179.86, 93.43);

\path[draw=drawColor,line width= 0.6pt,line join=round] (188.68, 26.91) rectangle (201.90,106.92);

\path[draw=drawColor,line width= 0.6pt,line join=round] (210.71, 26.91) rectangle (223.93,123.81);

\node[text=drawColor,anchor=base,inner sep=0pt, outer sep=0pt, scale=  0.85] at ( 19.02, 29.87) {2};

\node[text=drawColor,anchor=base,inner sep=0pt, outer sep=0pt, scale=  0.85] at ( 41.05, 29.87) {2};

\node[text=drawColor,anchor=base,inner sep=0pt, outer sep=0pt, scale=  0.85] at ( 63.08, 29.93) {5};

\node[text=drawColor,anchor=base,inner sep=0pt, outer sep=0pt, scale=  0.85] at ( 85.12, 31.01) {66};

\node[text=drawColor,anchor=base,inner sep=0pt, outer sep=0pt, scale=  0.85] at (107.15, 36.59) {381};

\node[text=drawColor,anchor=base,inner sep=0pt, outer sep=0pt, scale=  0.85] at (129.19, 49.87) {1,130};

\node[text=drawColor,anchor=base,inner sep=0pt, outer sep=0pt, scale=  0.85] at (151.22, 74.13) {2,499};

\node[text=drawColor,anchor=base,inner sep=0pt, outer sep=0pt, scale=  0.85] at (173.25, 96.36) {3,753};

\node[text=drawColor,anchor=base,inner sep=0pt, outer sep=0pt, scale=  0.85] at (195.29,109.85) {4,514};

\node[text=drawColor,anchor=base,inner sep=0pt, outer sep=0pt, scale=  0.85] at (217.32,126.74) {5,467};
\end{scope}
\begin{scope}
\path[clip] (  0.00,  0.00) rectangle (230.54,138.04);

\path[] (  5.80, 26.91) --
	(  5.80,123.81);
\end{scope}
\begin{scope}
\path[clip] (  0.00,  0.00) rectangle (230.54,138.04);
\definecolor[named]{drawColor}{rgb}{0.00,0.00,0.00}

\path[draw=drawColor,line width= 0.6pt,line join=round] (  5.80, 26.91) --
	(230.54, 26.91);
\end{scope}
\begin{scope}
\path[clip] (  0.00,  0.00) rectangle (230.54,138.04);

\path[] ( 19.02, 22.64) --
	( 19.02, 26.91);

\path[] ( 41.05, 22.64) --
	( 41.05, 26.91);

\path[] ( 63.08, 22.64) --
	( 63.08, 26.91);

\path[] ( 85.12, 22.64) --
	( 85.12, 26.91);

\path[] (107.15, 22.64) --
	(107.15, 26.91);

\path[] (129.19, 22.64) --
	(129.19, 26.91);

\path[] (151.22, 22.64) --
	(151.22, 26.91);

\path[] (173.25, 22.64) --
	(173.25, 26.91);

\path[] (195.29, 22.64) --
	(195.29, 26.91);

\path[] (217.32, 22.64) --
	(217.32, 26.91);
\end{scope}
\begin{scope}
\path[clip] (  0.00,  0.00) rectangle (230.54,138.04);
\definecolor[named]{drawColor}{rgb}{0.00,0.00,0.00}

\node[text=drawColor,anchor=base,inner sep=0pt, outer sep=0pt, scale=  1.00] at ( 19.02, 12.91) {10};

\node[text=drawColor,anchor=base,inner sep=0pt, outer sep=0pt, scale=  1.00] at ( 41.05, 12.91) {11};

\node[text=drawColor,anchor=base,inner sep=0pt, outer sep=0pt, scale=  1.00] at ( 63.08, 12.91) {12};

\node[text=drawColor,anchor=base,inner sep=0pt, outer sep=0pt, scale=  1.00] at ( 85.12, 12.91) {13};

\node[text=drawColor,anchor=base,inner sep=0pt, outer sep=0pt, scale=  1.00] at (107.15, 12.91) {14};

\node[text=drawColor,anchor=base,inner sep=0pt, outer sep=0pt, scale=  1.00] at (129.19, 12.91) {15};

\node[text=drawColor,anchor=base,inner sep=0pt, outer sep=0pt, scale=  1.00] at (151.22, 12.91) {16};

\node[text=drawColor,anchor=base,inner sep=0pt, outer sep=0pt, scale=  1.00] at (173.25, 12.91) {17};

\node[text=drawColor,anchor=base,inner sep=0pt, outer sep=0pt, scale=  1.00] at (195.29, 12.91) {18};

\node[text=drawColor,anchor=base,inner sep=0pt, outer sep=0pt, scale=  1.00] at (217.32, 12.91) {19};
\end{scope}
  \end{tikzpicture}}{INE, con datos del BANGUAT.}{\notitasin{Los datos del año 2014 se presentan como preliminares y serán ajustados por el registro tardío de los mismos.}}}
\hojados { \columna{Balanza Comercial General}{En el primer trimestre 2013 las exportaciones de Guatemala disminuyeron en 1.2\%  alcanzando US\$ 2,606,486,767,  comparando con el primer trimestre 2014 que crecieron en 3.5\% alcanzando un valor de US\$2,698,912,994.                                      Por su parte las importaciones en el mismo periodo se elevaron den un 51.6\% alcanzando el monto de US\$4,148,760,077  y el 2014  el crecimiento fue de 5.5\% con un monto de US\$4,376,926,898}{}{Balanza Comercial General trimestral  2011-2014}{(cifras preliminares en US\$)}{\ \\[6mm]\begin{tikzpicture}[x=1pt,y=1pt,scale=1]  % Created by tikzDevice version 0.7.0 on 2014-12-15 22:10:38
% !TEX encoding = UTF-8 Unicode
\definecolor[named]{fillColor}{rgb}{1.00,1.00,1.00}
\path[use as bounding box,fill=fillColor,fill opacity=0.00] (0,0) rectangle (230.54,138.04);
\begin{scope}
\path[clip] (  0.00,  0.00) rectangle (230.54,138.04);
\definecolor[named]{drawColor}{rgb}{1.00,1.00,1.00}

\path[draw=drawColor,line width= 0.6pt,line join=round,line cap=round] ( -0.00,  0.00) rectangle (230.54,138.04);
\end{scope}
\begin{scope}
\path[clip] (  0.00,  0.00) rectangle (230.54,138.04);

\path[] ( -5.16, 42.66) rectangle (222.01,130.92);

\path[] (  0.00, 59.54) --
	(222.01, 59.54);

\path[] (  0.00, 90.38) --
	(222.01, 90.38);

\path[] (  0.00,121.21) --
	(222.01,121.21);

\path[] (  0.00, 44.13) --
	(222.01, 44.13);

\path[] (  0.00, 74.96) --
	(222.01, 74.96);

\path[] (  0.00,105.79) --
	(222.01,105.79);

\path[] (  5.16, 42.66) --
	(  5.16,130.92);

\path[] ( 22.37, 42.66) --
	( 22.37,130.92);

\path[] ( 39.58, 42.66) --
	( 39.58,130.92);

\path[] ( 56.79, 42.66) --
	( 56.79,130.92);

\path[] ( 74.00, 42.66) --
	( 74.00,130.92);

\path[] ( 91.21, 42.66) --
	( 91.21,130.92);

\path[] (108.42, 42.66) --
	(108.42,130.92);

\path[] (125.63, 42.66) --
	(125.63,130.92);

\path[] (142.84, 42.66) --
	(142.84,130.92);

\path[] (160.05, 42.66) --
	(160.05,130.92);

\path[] (177.26, 42.66) --
	(177.26,130.92);

\path[] (194.47, 42.66) --
	(194.47,130.92);

\path[] (211.68, 42.66) --
	(211.68,130.92);
\definecolor[named]{drawColor}{rgb}{0.00,0.00,0.00}

\path[draw=drawColor,line width= 1.9pt,line join=round] (  5.16,115.19) --
	( 22.37,126.91) --
	( 39.58, 85.44) --
	( 56.79, 81.59) --
	( 74.00,122.90) --
	( 91.21, 82.05) --
	(108.42, 74.19) --
	(125.63,111.03) --
	(142.84, 80.36) --
	(160.05, 88.53) --
	(177.26, 99.62) --
	(194.47, 73.42) --
	(211.68, 88.53);

\node[text=drawColor,anchor=base,inner sep=0pt, outer sep=0pt, scale=  0.91] at (  5.16,105.82) {26.1};

\node[text=drawColor,anchor=base,inner sep=0pt, outer sep=0pt, scale=  0.91] at ( 22.37,130.03) {33.7};

\node[text=drawColor,anchor=base west,inner sep=0pt, outer sep=0pt, scale=  0.91] at ( 39.58, 88.57) {6.8};

\node[text=drawColor,anchor=base,inner sep=0pt, outer sep=0pt, scale=  0.91] at ( 56.79, 72.22) {4.3};

\node[text=drawColor,anchor=base,inner sep=0pt, outer sep=0pt, scale=  0.91] at ( 74.00,126.03) {31.1};

\node[text=drawColor,anchor=base west,inner sep=0pt, outer sep=0pt, scale=  0.91] at ( 91.21, 85.17) {4.6};

\node[text=drawColor,anchor=base,inner sep=0pt, outer sep=0pt, scale=  0.91] at (108.42, 64.82) {-0.5};

\node[text=drawColor,anchor=base,inner sep=0pt, outer sep=0pt, scale=  0.91] at (125.63,114.16) {23.4};

\node[text=drawColor,anchor=base,inner sep=0pt, outer sep=0pt, scale=  0.91] at (142.84, 70.98) {3.5};

\node[text=drawColor,anchor=base east,inner sep=0pt, outer sep=0pt, scale=  0.91] at (157.73, 88.53) {8.8};

\node[text=drawColor,anchor=base,inner sep=0pt, outer sep=0pt, scale=  0.91] at (177.26,102.75) {16.0};

\node[text=drawColor,anchor=base,inner sep=0pt, outer sep=0pt, scale=  0.91] at (194.47, 64.05) {-1.0};

\node[text=drawColor,anchor=base,inner sep=0pt, outer sep=0pt, scale=  0.91] at (211.68, 91.65) {8.8};
\end{scope}
\begin{scope}
\path[clip] (  0.00,  0.00) rectangle (230.54,138.04);
\definecolor[named]{drawColor}{rgb}{0.00,0.00,0.00}

\path[draw=drawColor,line width= 0.6pt,line join=round] (  0.00, 42.66) --
	(222.01, 42.66);
\end{scope}
\begin{scope}
\path[clip] (  0.00,  0.00) rectangle (230.54,138.04);

\path[] (  5.16, 38.39) --
	(  5.16, 42.66);

\path[] ( 22.37, 38.39) --
	( 22.37, 42.66);

\path[] ( 39.58, 38.39) --
	( 39.58, 42.66);

\path[] ( 56.79, 38.39) --
	( 56.79, 42.66);

\path[] ( 74.00, 38.39) --
	( 74.00, 42.66);

\path[] ( 91.21, 38.39) --
	( 91.21, 42.66);

\path[] (108.42, 38.39) --
	(108.42, 42.66);

\path[] (125.63, 38.39) --
	(125.63, 42.66);

\path[] (142.84, 38.39) --
	(142.84, 42.66);

\path[] (160.05, 38.39) --
	(160.05, 42.66);

\path[] (177.26, 38.39) --
	(177.26, 42.66);

\path[] (194.47, 38.39) --
	(194.47, 42.66);

\path[] (211.68, 38.39) --
	(211.68, 42.66);
\end{scope}
\begin{scope}
\path[clip] (  0.00,  0.00) rectangle (230.54,138.04);
\definecolor[named]{drawColor}{rgb}{0.00,0.00,0.00}

\node[text=drawColor,rotate= 90.00,anchor=base east,inner sep=0pt, outer sep=0pt, scale=  1.00] at (  8.61, 35.55) {T1-2011};

\node[text=drawColor,rotate= 90.00,anchor=base east,inner sep=0pt, outer sep=0pt, scale=  1.00] at ( 25.82, 35.55) {T2-2011};

\node[text=drawColor,rotate= 90.00,anchor=base east,inner sep=0pt, outer sep=0pt, scale=  1.00] at ( 43.02, 35.55) {T3-2011};

\node[text=drawColor,rotate= 90.00,anchor=base east,inner sep=0pt, outer sep=0pt, scale=  1.00] at ( 60.23, 35.55) {T4-2011};

\node[text=drawColor,rotate= 90.00,anchor=base east,inner sep=0pt, outer sep=0pt, scale=  1.00] at ( 77.44, 35.55) {T1-2012};

\node[text=drawColor,rotate= 90.00,anchor=base east,inner sep=0pt, outer sep=0pt, scale=  1.00] at ( 94.65, 35.55) {T2-2012};

\node[text=drawColor,rotate= 90.00,anchor=base east,inner sep=0pt, outer sep=0pt, scale=  1.00] at (111.86, 35.55) {T3-2012};

\node[text=drawColor,rotate= 90.00,anchor=base east,inner sep=0pt, outer sep=0pt, scale=  1.00] at (129.07, 35.55) {T4-2012};

\node[text=drawColor,rotate= 90.00,anchor=base east,inner sep=0pt, outer sep=0pt, scale=  1.00] at (146.28, 35.55) {T1-2013};

\node[text=drawColor,rotate= 90.00,anchor=base east,inner sep=0pt, outer sep=0pt, scale=  1.00] at (163.49, 35.55) {T2-2013};

\node[text=drawColor,rotate= 90.00,anchor=base east,inner sep=0pt, outer sep=0pt, scale=  1.00] at (180.70, 35.55) {T3-2013};

\node[text=drawColor,rotate= 90.00,anchor=base east,inner sep=0pt, outer sep=0pt, scale=  1.00] at (197.91, 35.55) {T4-2013};

\node[text=drawColor,rotate= 90.00,anchor=base east,inner sep=0pt, outer sep=0pt, scale=  1.00] at (215.12, 35.55) {T1-2014};
\end{scope}
  \end{tikzpicture}}{INE, con datos del BANGUAT.}{\notitasin{Los datos del año 2014 se presentan como preliminares y serán ajustados por el registro tardío de los mismos.}}}
{\columna{Variación}{La gráfica muestra la variación porcentual del saldo de la balanza comercial de  2011 a 2014.  Se observa que el saldo siendo negativo  muestra periodos en los cuales es más bajo,  comparando el primer trimestre 2012 el porcentaje es de -31.1\% al  primer trimestre 2013 el cual es  de -3.5\%  y el 2014  el porcentaje se eleva -8.8\%, siendo menos desfavorable para Guatemala.}{}{Variación porcentual del saldo de la Balanza comercial 2011-2014 }{( mismo trimestre años anteriores)}{\ \\[6mm]\begin{tikzpicture}[x=1pt,y=1pt,scale=1]  % Created by tikzDevice version 0.7.0 on 2014-12-15 22:10:38
% !TEX encoding = UTF-8 Unicode
\definecolor[named]{fillColor}{rgb}{1.00,1.00,1.00}
\path[use as bounding box,fill=fillColor,fill opacity=0.00] (0,0) rectangle (230.54,138.04);
\begin{scope}
\path[clip] (  0.00,  0.00) rectangle (230.54,138.04);
\definecolor[named]{drawColor}{rgb}{1.00,1.00,1.00}

\path[draw=drawColor,line width= 0.6pt,line join=round,line cap=round] ( -0.00,  0.00) rectangle (230.54,138.04);
\end{scope}
\begin{scope}
\path[clip] (  0.00,  0.00) rectangle (230.54,138.04);

\path[] ( -5.16, 42.66) rectangle (222.01,130.92);

\path[] (  0.00, 59.54) --
	(222.01, 59.54);

\path[] (  0.00, 90.38) --
	(222.01, 90.38);

\path[] (  0.00,121.21) --
	(222.01,121.21);

\path[] (  0.00, 44.13) --
	(222.01, 44.13);

\path[] (  0.00, 74.96) --
	(222.01, 74.96);

\path[] (  0.00,105.79) --
	(222.01,105.79);

\path[] (  5.16, 42.66) --
	(  5.16,130.92);

\path[] ( 22.37, 42.66) --
	( 22.37,130.92);

\path[] ( 39.58, 42.66) --
	( 39.58,130.92);

\path[] ( 56.79, 42.66) --
	( 56.79,130.92);

\path[] ( 74.00, 42.66) --
	( 74.00,130.92);

\path[] ( 91.21, 42.66) --
	( 91.21,130.92);

\path[] (108.42, 42.66) --
	(108.42,130.92);

\path[] (125.63, 42.66) --
	(125.63,130.92);

\path[] (142.84, 42.66) --
	(142.84,130.92);

\path[] (160.05, 42.66) --
	(160.05,130.92);

\path[] (177.26, 42.66) --
	(177.26,130.92);

\path[] (194.47, 42.66) --
	(194.47,130.92);

\path[] (211.68, 42.66) --
	(211.68,130.92);
\definecolor[named]{drawColor}{rgb}{0.00,0.00,0.00}

\path[draw=drawColor,line width= 1.9pt,line join=round] (  5.16,115.19) --
	( 22.37,126.91) --
	( 39.58, 85.44) --
	( 56.79, 81.59) --
	( 74.00,122.90) --
	( 91.21, 82.05) --
	(108.42, 74.19) --
	(125.63,111.03) --
	(142.84, 80.36) --
	(160.05, 88.53) --
	(177.26, 99.62) --
	(194.47, 73.42) --
	(211.68, 88.53);

\node[text=drawColor,anchor=base,inner sep=0pt, outer sep=0pt, scale=  0.91] at (  5.16,105.82) {26.1};

\node[text=drawColor,anchor=base,inner sep=0pt, outer sep=0pt, scale=  0.91] at ( 22.37,130.03) {33.7};

\node[text=drawColor,anchor=base west,inner sep=0pt, outer sep=0pt, scale=  0.91] at ( 39.58, 88.57) {6.8};

\node[text=drawColor,anchor=base,inner sep=0pt, outer sep=0pt, scale=  0.91] at ( 56.79, 72.22) {4.3};

\node[text=drawColor,anchor=base,inner sep=0pt, outer sep=0pt, scale=  0.91] at ( 74.00,126.03) {31.1};

\node[text=drawColor,anchor=base west,inner sep=0pt, outer sep=0pt, scale=  0.91] at ( 91.21, 85.17) {4.6};

\node[text=drawColor,anchor=base,inner sep=0pt, outer sep=0pt, scale=  0.91] at (108.42, 64.82) {-0.5};

\node[text=drawColor,anchor=base,inner sep=0pt, outer sep=0pt, scale=  0.91] at (125.63,114.16) {23.4};

\node[text=drawColor,anchor=base,inner sep=0pt, outer sep=0pt, scale=  0.91] at (142.84, 70.98) {3.5};

\node[text=drawColor,anchor=base east,inner sep=0pt, outer sep=0pt, scale=  0.91] at (157.73, 88.53) {8.8};

\node[text=drawColor,anchor=base,inner sep=0pt, outer sep=0pt, scale=  0.91] at (177.26,102.75) {16.0};

\node[text=drawColor,anchor=base,inner sep=0pt, outer sep=0pt, scale=  0.91] at (194.47, 64.05) {-1.0};

\node[text=drawColor,anchor=base,inner sep=0pt, outer sep=0pt, scale=  0.91] at (211.68, 91.65) {8.8};
\end{scope}
\begin{scope}
\path[clip] (  0.00,  0.00) rectangle (230.54,138.04);
\definecolor[named]{drawColor}{rgb}{0.00,0.00,0.00}

\path[draw=drawColor,line width= 0.6pt,line join=round] (  0.00, 42.66) --
	(222.01, 42.66);
\end{scope}
\begin{scope}
\path[clip] (  0.00,  0.00) rectangle (230.54,138.04);

\path[] (  5.16, 38.39) --
	(  5.16, 42.66);

\path[] ( 22.37, 38.39) --
	( 22.37, 42.66);

\path[] ( 39.58, 38.39) --
	( 39.58, 42.66);

\path[] ( 56.79, 38.39) --
	( 56.79, 42.66);

\path[] ( 74.00, 38.39) --
	( 74.00, 42.66);

\path[] ( 91.21, 38.39) --
	( 91.21, 42.66);

\path[] (108.42, 38.39) --
	(108.42, 42.66);

\path[] (125.63, 38.39) --
	(125.63, 42.66);

\path[] (142.84, 38.39) --
	(142.84, 42.66);

\path[] (160.05, 38.39) --
	(160.05, 42.66);

\path[] (177.26, 38.39) --
	(177.26, 42.66);

\path[] (194.47, 38.39) --
	(194.47, 42.66);

\path[] (211.68, 38.39) --
	(211.68, 42.66);
\end{scope}
\begin{scope}
\path[clip] (  0.00,  0.00) rectangle (230.54,138.04);
\definecolor[named]{drawColor}{rgb}{0.00,0.00,0.00}

\node[text=drawColor,rotate= 90.00,anchor=base east,inner sep=0pt, outer sep=0pt, scale=  1.00] at (  8.61, 35.55) {T1-2011};

\node[text=drawColor,rotate= 90.00,anchor=base east,inner sep=0pt, outer sep=0pt, scale=  1.00] at ( 25.82, 35.55) {T2-2011};

\node[text=drawColor,rotate= 90.00,anchor=base east,inner sep=0pt, outer sep=0pt, scale=  1.00] at ( 43.02, 35.55) {T3-2011};

\node[text=drawColor,rotate= 90.00,anchor=base east,inner sep=0pt, outer sep=0pt, scale=  1.00] at ( 60.23, 35.55) {T4-2011};

\node[text=drawColor,rotate= 90.00,anchor=base east,inner sep=0pt, outer sep=0pt, scale=  1.00] at ( 77.44, 35.55) {T1-2012};

\node[text=drawColor,rotate= 90.00,anchor=base east,inner sep=0pt, outer sep=0pt, scale=  1.00] at ( 94.65, 35.55) {T2-2012};

\node[text=drawColor,rotate= 90.00,anchor=base east,inner sep=0pt, outer sep=0pt, scale=  1.00] at (111.86, 35.55) {T3-2012};

\node[text=drawColor,rotate= 90.00,anchor=base east,inner sep=0pt, outer sep=0pt, scale=  1.00] at (129.07, 35.55) {T4-2012};

\node[text=drawColor,rotate= 90.00,anchor=base east,inner sep=0pt, outer sep=0pt, scale=  1.00] at (146.28, 35.55) {T1-2013};

\node[text=drawColor,rotate= 90.00,anchor=base east,inner sep=0pt, outer sep=0pt, scale=  1.00] at (163.49, 35.55) {T2-2013};

\node[text=drawColor,rotate= 90.00,anchor=base east,inner sep=0pt, outer sep=0pt, scale=  1.00] at (180.70, 35.55) {T3-2013};

\node[text=drawColor,rotate= 90.00,anchor=base east,inner sep=0pt, outer sep=0pt, scale=  1.00] at (197.91, 35.55) {T4-2013};

\node[text=drawColor,rotate= 90.00,anchor=base east,inner sep=0pt, outer sep=0pt, scale=  1.00] at (215.12, 35.55) {T1-2014};
\end{scope}
  \end{tikzpicture}}{INE, con datos del BANGUAT.}{\notitasin{Los datos del año 2014 se presentan como preliminares y serán ajustados por el registro tardío de los mismos.}}}
%\hojados{ \columna{Balanza Comercial por continente}{La gráfica muestra la variación porcentual del saldo de la balanza comercial de  2014.  El continente con un mayor volumen de operaciones es con el de América con  un 79\% en las exportaciones y  67\% en las importaciones, seguido  por el continente  de Asia con 13\% en exportaciones y   24\% en las importaciones y en tercer lugar el continente  con un 6\% en las exportaciones y  9\% en las importaciones.}{}{Balanza Comercial por Continente  primer trimestre 2014}{(cifras preliminares en US\$)}{\ \\[6mm]\begin{tikzpicture}[x=1pt,y=1pt,scale=1]  % Created by tikzDevice version 0.7.0 on 2014-12-11 15:14:01
% !TEX encoding = UTF-8 Unicode
\definecolor[named]{fillColor}{rgb}{1.00,1.00,1.00}
\path[use as bounding box,fill=fillColor,fill opacity=0.00] (0,0) rectangle (230.54,138.04);
\begin{scope}
\path[clip] (  0.00,  0.00) rectangle (230.54,138.04);
\definecolor[named]{drawColor}{rgb}{1.00,1.00,1.00}

\path[draw=drawColor,line width= 0.6pt,line join=round,line cap=round] (  0.00,  0.00) rectangle (230.54,138.04);
\end{scope}
\begin{scope}
\path[clip] (  0.00,  0.00) rectangle (230.54,138.04);

\path[] (  7.00, 28.11) rectangle (230.54,123.81);

\path[] ( 67.97, 28.11) --
	( 67.97,123.81);

\path[] (169.58, 28.11) --
	(169.58,123.81);
\definecolor[named]{drawColor}{rgb}{0.00,0.00,0.00}

\path[draw=drawColor,line width= 0.6pt,line join=round] ( 37.48, 28.11) rectangle ( 98.45,123.81);

\path[draw=drawColor,line width= 0.6pt,line join=round] (139.09, 28.11) rectangle (200.06,120.45);

\node[text=drawColor,anchor=base,inner sep=0pt, outer sep=0pt, scale=  0.71] at ( 67.97,126.74) {50.9};

\node[text=drawColor,anchor=base,inner sep=0pt, outer sep=0pt, scale=  0.71] at (169.58,123.38) {49.1};
\end{scope}
\begin{scope}
\path[clip] (  0.00,  0.00) rectangle (230.54,138.04);

\path[] (  7.00, 28.11) --
	(  7.00,123.81);
\end{scope}
\begin{scope}
\path[clip] (  0.00,  0.00) rectangle (230.54,138.04);
\definecolor[named]{drawColor}{rgb}{0.00,0.00,0.00}

\path[draw=drawColor,line width= 0.6pt,line join=round] (  7.00, 28.11) --
	(230.54, 28.11);
\end{scope}
\begin{scope}
\path[clip] (  0.00,  0.00) rectangle (230.54,138.04);

\path[] ( 67.97, 23.85) --
	( 67.97, 28.11);

\path[] (169.58, 23.85) --
	(169.58, 28.11);
\end{scope}
\begin{scope}
\path[clip] (  0.00,  0.00) rectangle (230.54,138.04);
\definecolor[named]{drawColor}{rgb}{0.00,0.00,0.00}

\node[text=drawColor,anchor=base,inner sep=0pt, outer sep=0pt, scale=  0.83] at ( 67.97, 14.11) {Hombre};

\node[text=drawColor,anchor=base,inner sep=0pt, outer sep=0pt, scale=  0.83] at (169.58, 14.11) {Mujer};
\end{scope}
  \end{tikzpicture}}{INE, con datos del BANGUAT.}{\notitasin{Los datos del año 2014 se presentan como preliminares y serán ajustados por el registro tardío de los mismos.}}}
%{\columna{Balanza Comercial con Centroamérica}{Los países centroaméricanos como principales socios comerciales se  el Salvador con un porcentaje de 38.2\% en las exportaciones y 42.9\% en las importaciones seguido de Honduras con 25.7\% en las Exportaciones y 16\% en las importaciones, en tercer lugar es Costa Rica con 12.5\%  en las exportaciones y 30.6\% en las importaciones,  con los otros países se sostiene un porcentaje más bajo en ambas vías.}{}{    Balanza  Comercial con el Mercado Común  Centroaméricano,}{( Cifras expresadas en US dólares )}{\ \\[6mm]\begin{tikzpicture}[x=1pt,y=1pt,scale=1]  % Created by tikzDevice version 0.7.0 on 2014-12-18 20:47:29
% !TEX encoding = UTF-8 Unicode
\definecolor[named]{fillColor}{rgb}{1.00,1.00,1.00}
\path[use as bounding box,fill=fillColor,fill opacity=0.00] (0,0) rectangle (230.54,138.04);
\begin{scope}
\path[clip] (  0.00,  0.00) rectangle (230.54,138.04);
\definecolor[named]{drawColor}{rgb}{1.00,1.00,1.00}

\path[draw=drawColor,line width= 0.6pt,line join=round,line cap=round] (  0.00,  0.00) rectangle (230.54,138.04);
\end{scope}
\begin{scope}
\path[clip] (  0.00,  0.00) rectangle (230.54,138.04);

\path[] ( 60.29, -2.49) rectangle (203.23,138.04);

\path[] ( 60.29, 13.73) --
	(203.23, 13.73);

\path[] ( 60.29, 40.75) --
	(203.23, 40.75);

\path[] ( 60.29, 67.77) --
	(203.23, 67.77);

\path[] ( 60.29, 94.80) --
	(203.23, 94.80);

\path[] ( 60.29,121.82) --
	(203.23,121.82);
\definecolor[named]{drawColor}{rgb}{0.78,0.78,0.78}
\definecolor[named]{fillColor}{rgb}{0.78,0.78,0.78}

\path[draw=drawColor,line width= 0.6pt,line join=round,fill=fillColor] ( 60.29,  5.62) rectangle ( 60.45, 21.83);
\definecolor[named]{drawColor}{rgb}{0.00,0.00,0.00}

\path[draw=drawColor,line width= 0.6pt,line join=round] ( 60.29, 32.64) rectangle ( 60.55, 48.86);

\path[draw=drawColor,line width= 0.6pt,line join=round] ( 60.29, 59.67) rectangle ( 61.02, 75.88);

\path[draw=drawColor,line width= 0.6pt,line join=round] ( 60.29, 86.69) rectangle ( 79.06,102.90);

\path[draw=drawColor,line width= 0.6pt,line join=round] ( 60.29,113.71) rectangle (203.23,129.93);

\node[text=drawColor,anchor=base west,inner sep=0pt, outer sep=0pt, scale=  0.85] at ( 64.73, 10.69) {0.1};

\node[text=drawColor,anchor=base west,inner sep=0pt, outer sep=0pt, scale=  0.85] at ( 64.83, 37.71) {0.2};

\node[text=drawColor,anchor=base west,inner sep=0pt, outer sep=0pt, scale=  0.85] at ( 65.30, 64.74) {0.4};

\node[text=drawColor,anchor=base west,inner sep=0pt, outer sep=0pt, scale=  0.85] at ( 85.04, 91.76) {11.5};

\node[text=drawColor,anchor=base west,inner sep=0pt, outer sep=0pt, scale=  0.85] at (209.20,118.79) {87.8};
\end{scope}
\begin{scope}
\path[clip] (  0.00,  0.00) rectangle (230.54,138.04);
\definecolor[named]{drawColor}{rgb}{0.00,0.00,0.00}

\path[draw=drawColor,line width= 0.6pt,line join=round] ( 60.29,  0.00) --
	( 60.29,138.04);
\end{scope}
\begin{scope}
\path[clip] (  0.00,  0.00) rectangle (230.54,138.04);
\definecolor[named]{drawColor}{rgb}{0.00,0.00,0.00}

\node[text=drawColor,anchor=base east,inner sep=0pt, outer sep=0pt, scale=  1.00] at ( 53.18, 10.16) {Ignorado};

\node[text=drawColor,anchor=base east,inner sep=0pt, outer sep=0pt, scale=  1.00] at ( 53.18, 37.18) {Peso extremadamente bajo};

\node[text=drawColor,anchor=base east,inner sep=0pt, outer sep=0pt, scale=  1.00] at ( 53.18, 64.20) {Peso muy bajo};

\node[text=drawColor,anchor=base east,inner sep=0pt, outer sep=0pt, scale=  1.00] at ( 53.18, 91.23) {Peso bajo};

\node[text=drawColor,anchor=base east,inner sep=0pt, outer sep=0pt, scale=  1.00] at ( 53.18,118.25) {Peso adecuado};
\end{scope}
\begin{scope}
\path[clip] (  0.00,  0.00) rectangle (230.54,138.04);

\path[] ( 56.03, 13.73) --
	( 60.29, 13.73);

\path[] ( 56.03, 40.75) --
	( 60.29, 40.75);

\path[] ( 56.03, 67.77) --
	( 60.29, 67.77);

\path[] ( 56.03, 94.80) --
	( 60.29, 94.80);

\path[] ( 56.03,121.82) --
	( 60.29,121.82);
\end{scope}
  \end{tikzpicture}}{INE, con datos del BANGUAT.}{\notitasin{Los datos del año 2014 se presentan como preliminares y serán ajustados por el registro tardío de los mismos.}}}
\hojados{\columna{Exportaciones 10 principales productos trimestre 2014}{El principal producto de exportación de Guatemala al resto del mundo, constituyó en el primer trimestre 2014, en el azucar con 25.9\%  de porcentaje en esta serie y 292,652,996 US\$, seguido de  café oro con un 16.1\%  182,169,782 y banano con 13.8\% y 155,826,949; seguido de minerales de plata y Minerales de plomo con un 7.6\% y 7.3\% respectivamente.}{}{Exportación de los principales 10 productos, Primer trimestre 2014}{( cifras preliminares expresadas en US\$ )}{\ \\[6mm]\begin{tikzpicture}[x=1pt,y=1pt,scale=1]  % Created by tikzDevice version 0.7.0 on 2014-12-15 22:10:41
% !TEX encoding = UTF-8 Unicode
\definecolor[named]{fillColor}{rgb}{1.00,1.00,1.00}
\path[use as bounding box,fill=fillColor,fill opacity=0.00] (0,0) rectangle (230.54,138.04);
\begin{scope}
\path[clip] (  0.00,  0.00) rectangle (230.54,138.04);
\definecolor[named]{drawColor}{rgb}{1.00,1.00,1.00}

\path[draw=drawColor,line width= 0.6pt,line join=round,line cap=round] (  0.00, -0.00) rectangle (230.54,138.04);
\end{scope}
\begin{scope}
\path[clip] (  0.00,  0.00) rectangle (230.54,138.04);

\path[] (  5.80, 82.60) rectangle (230.54,123.81);

\path[] ( 19.02, 82.60) --
	( 19.02,123.81);

\path[] ( 41.05, 82.60) --
	( 41.05,123.81);

\path[] ( 63.08, 82.60) --
	( 63.08,123.81);

\path[] ( 85.12, 82.60) --
	( 85.12,123.81);

\path[] (107.15, 82.60) --
	(107.15,123.81);

\path[] (129.19, 82.60) --
	(129.19,123.81);

\path[] (151.22, 82.60) --
	(151.22,123.81);

\path[] (173.25, 82.60) --
	(173.25,123.81);

\path[] (195.29, 82.60) --
	(195.29,123.81);

\path[] (217.32, 82.60) --
	(217.32,123.81);
\definecolor[named]{drawColor}{rgb}{0.00,0.00,0.00}

\path[draw=drawColor,line width= 0.6pt,line join=round] ( 12.41, 82.60) rectangle ( 25.63,123.81);

\path[draw=drawColor,line width= 0.6pt,line join=round] ( 34.44, 82.60) rectangle ( 47.66,108.20);

\path[draw=drawColor,line width= 0.6pt,line join=round] ( 56.47, 82.60) rectangle ( 69.69,104.54);

\path[draw=drawColor,line width= 0.6pt,line join=round] ( 78.51, 82.60) rectangle ( 91.73, 94.70);

\path[draw=drawColor,line width= 0.6pt,line join=round] (100.54, 82.60) rectangle (113.76, 94.27);

\path[draw=drawColor,line width= 0.6pt,line join=round] (122.58, 82.60) rectangle (135.80, 93.01);

\path[draw=drawColor,line width= 0.6pt,line join=round] (144.61, 82.60) rectangle (157.83, 92.45);

\path[draw=drawColor,line width= 0.6pt,line join=round] (166.64, 82.60) rectangle (179.86, 92.02);

\path[draw=drawColor,line width= 0.6pt,line join=round] (188.68, 82.60) rectangle (201.90, 91.74);

\path[draw=drawColor,line width= 0.6pt,line join=round] (210.71, 82.60) rectangle (223.93, 90.48);

\node[text=drawColor,anchor=base,inner sep=0pt, outer sep=0pt, scale=  0.85] at ( 19.02,126.74) {293};

\node[text=drawColor,anchor=base,inner sep=0pt, outer sep=0pt, scale=  0.85] at ( 41.05,111.13) {182};

\node[text=drawColor,anchor=base,inner sep=0pt, outer sep=0pt, scale=  0.85] at ( 63.08,107.47) {156};

\node[text=drawColor,anchor=base,inner sep=0pt, outer sep=0pt, scale=  0.85] at ( 85.12, 97.62) {86};

\node[text=drawColor,anchor=base,inner sep=0pt, outer sep=0pt, scale=  0.85] at (107.15, 97.20) {83};

\node[text=drawColor,anchor=base,inner sep=0pt, outer sep=0pt, scale=  0.85] at (129.19, 95.94) {74};

\node[text=drawColor,anchor=base,inner sep=0pt, outer sep=0pt, scale=  0.85] at (151.22, 95.37) {70};

\node[text=drawColor,anchor=base,inner sep=0pt, outer sep=0pt, scale=  0.85] at (173.25, 94.95) {67};

\node[text=drawColor,anchor=base,inner sep=0pt, outer sep=0pt, scale=  0.85] at (195.29, 94.67) {65};

\node[text=drawColor,anchor=base,inner sep=0pt, outer sep=0pt, scale=  0.85] at (217.32, 93.41) {56};
\end{scope}
\begin{scope}
\path[clip] (  0.00,  0.00) rectangle (230.54,138.04);

\path[] (  5.80, 82.60) --
	(  5.80,123.81);
\end{scope}
\begin{scope}
\path[clip] (  0.00,  0.00) rectangle (230.54,138.04);
\definecolor[named]{drawColor}{rgb}{0.00,0.00,0.00}

\path[draw=drawColor,line width= 0.6pt,line join=round] (  5.80, 82.60) --
	(230.54, 82.60);
\end{scope}
\begin{scope}
\path[clip] (  0.00,  0.00) rectangle (230.54,138.04);

\path[] ( 19.02, 78.33) --
	( 19.02, 82.60);

\path[] ( 41.05, 78.33) --
	( 41.05, 82.60);

\path[] ( 63.08, 78.33) --
	( 63.08, 82.60);

\path[] ( 85.12, 78.33) --
	( 85.12, 82.60);

\path[] (107.15, 78.33) --
	(107.15, 82.60);

\path[] (129.19, 78.33) --
	(129.19, 82.60);

\path[] (151.22, 78.33) --
	(151.22, 82.60);

\path[] (173.25, 78.33) --
	(173.25, 82.60);

\path[] (195.29, 78.33) --
	(195.29, 82.60);

\path[] (217.32, 78.33) --
	(217.32, 82.60);
\end{scope}
\begin{scope}
\path[clip] (  0.00,  0.00) rectangle (230.54,138.04);
\definecolor[named]{drawColor}{rgb}{0.00,0.00,0.00}

\node[text=drawColor,rotate= 90.00,anchor=base east,inner sep=0pt, outer sep=0pt, scale=  1.00] at ( 22.46, 75.49) {Azúcar};

\node[text=drawColor,rotate= 90.00,anchor=base east,inner sep=0pt, outer sep=0pt, scale=  1.00] at ( 44.49, 75.49) {Café oro};

\node[text=drawColor,rotate= 90.00,anchor=base east,inner sep=0pt, outer sep=0pt, scale=  1.00] at ( 66.53, 75.49) {Banano};

\node[text=drawColor,rotate= 90.00,anchor=base east,inner sep=0pt, outer sep=0pt, scale=  1.00] at ( 88.56, 75.49) {Minerales de plata};

\node[text=drawColor,rotate= 90.00,anchor=base east,inner sep=0pt, outer sep=0pt, scale=  1.00] at (110.60, 75.49) {Minerales de plomo};

\node[text=drawColor,rotate= 90.00,anchor=base east,inner sep=0pt, outer sep=0pt, scale=  1.00] at (132.63, 75.49) {Cardamomos};

\node[text=drawColor,rotate= 90.00,anchor=base east,inner sep=0pt, outer sep=0pt, scale=  1.00] at (154.66, 75.49) {Blusas de algodón};

\node[text=drawColor,rotate= 90.00,anchor=base east,inner sep=0pt, outer sep=0pt, scale=  1.00] at (176.70, 75.49) {Melones};

\node[text=drawColor,rotate= 90.00,anchor=base east,inner sep=0pt, outer sep=0pt, scale=  1.00] at (198.73, 75.49) {Petróleo crudo};

\node[text=drawColor,rotate= 90.00,anchor=base east,inner sep=0pt, outer sep=0pt, scale=  1.00] at (220.76, 75.49) {Alcohol Etílico};
\end{scope}
  \end{tikzpicture}}{INE, con datos del BANGUAT.}{\notitasin{Los datos del año 2014 se presentan como preliminares y serán ajustados por el registro tardío de los mismos.}}}
{\columna{Exportaciones 10 principales países trimestre 2014}{El principal socio comercial de Guatemala es Estados unidos a donde se exportó en el primer trimestre 2014,  969,383,445 US\$ con un porcentaje de 35.9\% seguido de El Salvador  a quien se le exportó 318,214,092 con un porcentaje de 11.8\%, en tercer lugar Honduras  a donde se exportó 213,579,857 con un porcentaje de 7.9\%}{}{Exportación de los principales 10 países, Primer trimestre 2014}{(cifras preliminares expresadas en US\$)}{\ \\[6mm]\begin{tikzpicture}[x=1pt,y=1pt,scale=1]  % Created by tikzDevice version 0.7.0 on 2014-12-15 22:10:43
% !TEX encoding = UTF-8 Unicode
\definecolor[named]{fillColor}{rgb}{1.00,1.00,1.00}
\path[use as bounding box,fill=fillColor,fill opacity=0.00] (0,0) rectangle (230.54,138.04);
\begin{scope}
\path[clip] (  0.00,  0.00) rectangle (230.54,138.04);
\definecolor[named]{drawColor}{rgb}{1.00,1.00,1.00}

\path[draw=drawColor,line width= 0.6pt,line join=round,line cap=round] (  0.00, -0.00) rectangle (230.54,138.04);
\end{scope}
\begin{scope}
\path[clip] (  0.00,  0.00) rectangle (230.54,138.04);

\path[] (  5.80, 57.30) rectangle (230.54,123.81);

\path[] ( 19.02, 57.30) --
	( 19.02,123.81);

\path[] ( 41.05, 57.30) --
	( 41.05,123.81);

\path[] ( 63.08, 57.30) --
	( 63.08,123.81);

\path[] ( 85.12, 57.30) --
	( 85.12,123.81);

\path[] (107.15, 57.30) --
	(107.15,123.81);

\path[] (129.19, 57.30) --
	(129.19,123.81);

\path[] (151.22, 57.30) --
	(151.22,123.81);

\path[] (173.25, 57.30) --
	(173.25,123.81);

\path[] (195.29, 57.30) --
	(195.29,123.81);

\path[] (217.32, 57.30) --
	(217.32,123.81);
\definecolor[named]{drawColor}{rgb}{0.00,0.00,0.00}

\path[draw=drawColor,line width= 0.6pt,line join=round] ( 12.41, 57.30) rectangle ( 25.63,123.81);

\path[draw=drawColor,line width= 0.6pt,line join=round] ( 34.44, 57.30) rectangle ( 47.66, 79.15);

\path[draw=drawColor,line width= 0.6pt,line join=round] ( 56.47, 57.30) rectangle ( 69.69, 72.00);

\path[draw=drawColor,line width= 0.6pt,line join=round] ( 78.51, 57.30) rectangle ( 91.73, 68.23);

\path[draw=drawColor,line width= 0.6pt,line join=round] (100.54, 57.30) rectangle (113.76, 65.20);

\path[draw=drawColor,line width= 0.6pt,line join=round] (122.58, 57.30) rectangle (135.80, 64.45);

\path[draw=drawColor,line width= 0.6pt,line join=round] (144.61, 57.30) rectangle (157.83, 64.31);

\path[draw=drawColor,line width= 0.6pt,line join=round] (166.64, 57.30) rectangle (179.86, 61.77);

\path[draw=drawColor,line width= 0.6pt,line join=round] (188.68, 57.30) rectangle (201.90, 61.08);

\path[draw=drawColor,line width= 0.6pt,line join=round] (210.71, 57.30) rectangle (223.93, 60.94);

\node[text=drawColor,anchor=base,inner sep=0pt, outer sep=0pt, scale=  0.85] at ( 19.02,126.74) {968};

\node[text=drawColor,anchor=base,inner sep=0pt, outer sep=0pt, scale=  0.85] at ( 41.05, 82.08) {318};

\node[text=drawColor,anchor=base,inner sep=0pt, outer sep=0pt, scale=  0.85] at ( 63.08, 74.93) {214};

\node[text=drawColor,anchor=base,inner sep=0pt, outer sep=0pt, scale=  0.85] at ( 85.12, 71.15) {159};

\node[text=drawColor,anchor=base,inner sep=0pt, outer sep=0pt, scale=  0.85] at (107.15, 68.13) {115};

\node[text=drawColor,anchor=base,inner sep=0pt, outer sep=0pt, scale=  0.85] at (129.19, 67.38) {104};

\node[text=drawColor,anchor=base,inner sep=0pt, outer sep=0pt, scale=  0.85] at (151.22, 67.24) {102};

\node[text=drawColor,anchor=base,inner sep=0pt, outer sep=0pt, scale=  0.85] at (173.25, 64.70) {65};

\node[text=drawColor,anchor=base,inner sep=0pt, outer sep=0pt, scale=  0.85] at (195.29, 64.01) {55};

\node[text=drawColor,anchor=base,inner sep=0pt, outer sep=0pt, scale=  0.85] at (217.32, 63.87) {53};
\end{scope}
\begin{scope}
\path[clip] (  0.00,  0.00) rectangle (230.54,138.04);

\path[] (  5.80, 57.30) --
	(  5.80,123.81);
\end{scope}
\begin{scope}
\path[clip] (  0.00,  0.00) rectangle (230.54,138.04);
\definecolor[named]{drawColor}{rgb}{0.00,0.00,0.00}

\path[draw=drawColor,line width= 0.6pt,line join=round] (  5.80, 57.30) --
	(230.54, 57.30);
\end{scope}
\begin{scope}
\path[clip] (  0.00,  0.00) rectangle (230.54,138.04);

\path[] ( 19.02, 53.03) --
	( 19.02, 57.30);

\path[] ( 41.05, 53.03) --
	( 41.05, 57.30);

\path[] ( 63.08, 53.03) --
	( 63.08, 57.30);

\path[] ( 85.12, 53.03) --
	( 85.12, 57.30);

\path[] (107.15, 53.03) --
	(107.15, 57.30);

\path[] (129.19, 53.03) --
	(129.19, 57.30);

\path[] (151.22, 53.03) --
	(151.22, 57.30);

\path[] (173.25, 53.03) --
	(173.25, 57.30);

\path[] (195.29, 53.03) --
	(195.29, 57.30);

\path[] (217.32, 53.03) --
	(217.32, 57.30);
\end{scope}
\begin{scope}
\path[clip] (  0.00,  0.00) rectangle (230.54,138.04);
\definecolor[named]{drawColor}{rgb}{0.00,0.00,0.00}

\node[text=drawColor,rotate= 90.00,anchor=base east,inner sep=0pt, outer sep=0pt, scale=  1.00] at ( 22.46, 50.19) {USA};

\node[text=drawColor,rotate= 90.00,anchor=base east,inner sep=0pt, outer sep=0pt, scale=  1.00] at ( 44.49, 50.19) {El Salvador};

\node[text=drawColor,rotate= 90.00,anchor=base east,inner sep=0pt, outer sep=0pt, scale=  1.00] at ( 66.53, 50.19) {Honduras};

\node[text=drawColor,rotate= 90.00,anchor=base east,inner sep=0pt, outer sep=0pt, scale=  1.00] at ( 88.56, 50.19) {Corea del Sur};

\node[text=drawColor,rotate= 90.00,anchor=base east,inner sep=0pt, outer sep=0pt, scale=  1.00] at (110.60, 50.19) {Nicaragua};

\node[text=drawColor,rotate= 90.00,anchor=base east,inner sep=0pt, outer sep=0pt, scale=  1.00] at (132.63, 50.19) {Costa Rica};

\node[text=drawColor,rotate= 90.00,anchor=base east,inner sep=0pt, outer sep=0pt, scale=  1.00] at (154.66, 50.19) {México};

\node[text=drawColor,rotate= 90.00,anchor=base east,inner sep=0pt, outer sep=0pt, scale=  1.00] at (176.70, 50.19) {Panamá};

\node[text=drawColor,rotate= 90.00,anchor=base east,inner sep=0pt, outer sep=0pt, scale=  1.00] at (198.73, 50.19) {Canadá};

\node[text=drawColor,rotate= 90.00,anchor=base east,inner sep=0pt, outer sep=0pt, scale=  1.00] at (220.76, 50.19) {Países Bajos};
\end{scope}
  \end{tikzpicture}}{INE, con datos del BANGUAT.}{\notitasin{Los datos del año 2014 se presentan como preliminares y serán ajustados por el registro tardío de los mismos.}}}
\hojados{\columna{Exportaciones 10 principales secciones}{De las secciones del Sistema Arancelario Centroaméricano (SAC) en el primer trimestre 2014 en las exportaciones de Guatemala, la correspondiente a productos del reino vegetal ocupa el primer lugar con un 25.4\%, el segundo lugar con los productos de las Industrias Alimenticias un 21.6\%,  en tercer lugar los productos Minerales con un 14.5\% seguido de Industrias textiles y Productos y materias plásticas on un 13.9\% y 8.6\% respectivamente}{}{Exportación de las principales secciones del Sistema Arancelario C. A., Primer trimestre 2014}{(Cifras preliminares expresadas en US\$)}{\ \\[6mm]\begin{tikzpicture}[x=1pt,y=1pt,scale=1]  % Created by tikzDevice version 0.7.0 on 2014-12-15 21:46:14
% !TEX encoding = UTF-8 Unicode
\definecolor[named]{fillColor}{rgb}{1.00,1.00,1.00}
\path[use as bounding box,fill=fillColor,fill opacity=0.00] (0,0) rectangle (230.54,138.04);
\begin{scope}
\path[clip] (  0.00,  0.00) rectangle (230.54,138.04);
\definecolor[named]{drawColor}{rgb}{1.00,1.00,1.00}

\path[draw=drawColor,line width= 0.6pt,line join=round,line cap=round] (  0.00,  0.00) rectangle (230.54,138.04);
\end{scope}
\begin{scope}
\path[clip] (  0.00,  0.00) rectangle (230.54,138.04);

\path[] (110.62, -2.74) rectangle (210.06,138.04);

\path[] (110.62,  4.80) --
	(210.06,  4.80);

\path[] (110.62, 17.37) --
	(210.06, 17.37);

\path[] (110.62, 29.94) --
	(210.06, 29.94);

\path[] (110.62, 42.51) --
	(210.06, 42.51);

\path[] (110.62, 55.08) --
	(210.06, 55.08);

\path[] (110.62, 67.65) --
	(210.06, 67.65);

\path[] (110.62, 80.22) --
	(210.06, 80.22);

\path[] (110.62, 92.79) --
	(210.06, 92.79);

\path[] (110.62,105.36) --
	(210.06,105.36);

\path[] (110.62,117.92) --
	(210.06,117.92);

\path[] (110.62,130.49) --
	(210.06,130.49);
\definecolor[named]{drawColor}{rgb}{0.00,0.00,0.00}

\path[draw=drawColor,line width= 0.6pt,line join=round] (110.62,  1.03) rectangle (114.11,  8.57);

\path[draw=drawColor,line width= 0.6pt,line join=round] (110.62, 13.60) rectangle (116.38, 21.14);

\path[draw=drawColor,line width= 0.6pt,line join=round] (110.62, 26.17) rectangle (120.62, 33.71);

\path[draw=drawColor,line width= 0.6pt,line join=round] (110.62, 38.74) rectangle (120.93, 46.28);

\path[draw=drawColor,line width= 0.6pt,line join=round] (110.62, 51.31) rectangle (124.41, 58.85);

\path[draw=drawColor,line width= 0.6pt,line join=round] (110.62, 63.88) rectangle (130.02, 71.42);

\path[draw=drawColor,line width= 0.6pt,line join=round] (110.62, 76.45) rectangle (144.57, 83.99);

\path[draw=drawColor,line width= 0.6pt,line join=round] (110.62, 89.02) rectangle (165.04, 96.56);

\path[draw=drawColor,line width= 0.6pt,line join=round] (110.62,101.59) rectangle (167.46,109.13);

\path[draw=drawColor,line width= 0.6pt,line join=round] (110.62,114.15) rectangle (195.35,121.70);

\path[draw=drawColor,line width= 0.6pt,line join=round] (110.62,126.72) rectangle (210.06,134.26);

\node[text=drawColor,anchor=base west,inner sep=0pt, outer sep=0pt, scale=  0.85] at (118.36,  1.87) {23};

\node[text=drawColor,anchor=base west,inner sep=0pt, outer sep=0pt, scale=  0.85] at (120.63, 14.44) {38};

\node[text=drawColor,anchor=base west,inner sep=0pt, outer sep=0pt, scale=  0.85] at (124.87, 27.01) {66};

\node[text=drawColor,anchor=base west,inner sep=0pt, outer sep=0pt, scale=  0.85] at (125.18, 39.58) {68};

\node[text=drawColor,anchor=base west,inner sep=0pt, outer sep=0pt, scale=  0.85] at (128.66, 52.15) {91};

\node[text=drawColor,anchor=base west,inner sep=0pt, outer sep=0pt, scale=  0.85] at (136.40, 64.72) {128};

\node[text=drawColor,anchor=base west,inner sep=0pt, outer sep=0pt, scale=  0.85] at (150.95, 77.29) {224};

\node[text=drawColor,anchor=base west,inner sep=0pt, outer sep=0pt, scale=  0.85] at (171.41, 89.86) {359};

\node[text=drawColor,anchor=base west,inner sep=0pt, outer sep=0pt, scale=  0.85] at (173.84,102.43) {375};

\node[text=drawColor,anchor=base west,inner sep=0pt, outer sep=0pt, scale=  0.85] at (201.73,115.00) {559};

\node[text=drawColor,anchor=base west,inner sep=0pt, outer sep=0pt, scale=  0.85] at (216.43,127.57) {656};
\end{scope}
\begin{scope}
\path[clip] (  0.00,  0.00) rectangle (230.54,138.04);
\definecolor[named]{drawColor}{rgb}{0.00,0.00,0.00}

\path[draw=drawColor,line width= 0.6pt,line join=round] (110.62,  0.00) --
	(110.62,138.04);
\end{scope}
\begin{scope}
\path[clip] (  0.00,  0.00) rectangle (230.54,138.04);
\definecolor[named]{drawColor}{rgb}{0.00,0.00,0.00}

\node[text=drawColor,anchor=base east,inner sep=0pt, outer sep=0pt, scale=  1.00] at (103.51,  1.36) {Mercancía y Productos diversos};

\node[text=drawColor,anchor=base east,inner sep=0pt, outer sep=0pt, scale=  1.00] at (103.51, 13.93) {Máquinas y Aparatos};

\node[text=drawColor,anchor=base east,inner sep=0pt, outer sep=0pt, scale=  1.00] at (103.51, 26.50) {Pastas de Madera};

\node[text=drawColor,anchor=base east,inner sep=0pt, outer sep=0pt, scale=  1.00] at (103.51, 39.07) {Grasas};

\node[text=drawColor,anchor=base east,inner sep=0pt, outer sep=0pt, scale=  1.00] at (103.51, 51.64) {Metales Comunes};

\node[text=drawColor,anchor=base east,inner sep=0pt, outer sep=0pt, scale=  1.00] at (103.51, 64.20) {Materias Plásticas};

\node[text=drawColor,anchor=base east,inner sep=0pt, outer sep=0pt, scale=  1.00] at (103.51, 76.77) {Industrias Químicas};

\node[text=drawColor,anchor=base east,inner sep=0pt, outer sep=0pt, scale=  1.00] at (103.51, 89.34) {Textiles y sus manufacturas};

\node[text=drawColor,anchor=base east,inner sep=0pt, outer sep=0pt, scale=  1.00] at (103.51,101.91) {Productos Minerales};

\node[text=drawColor,anchor=base east,inner sep=0pt, outer sep=0pt, scale=  1.00] at (103.51,114.48) {Industrias Alimenticias};

\node[text=drawColor,anchor=base east,inner sep=0pt, outer sep=0pt, scale=  1.00] at (103.51,127.05) {Reino Vegetal};
\end{scope}
\begin{scope}
\path[clip] (  0.00,  0.00) rectangle (230.54,138.04);

\path[] (106.35,  4.80) --
	(110.62,  4.80);

\path[] (106.35, 17.37) --
	(110.62, 17.37);

\path[] (106.35, 29.94) --
	(110.62, 29.94);

\path[] (106.35, 42.51) --
	(110.62, 42.51);

\path[] (106.35, 55.08) --
	(110.62, 55.08);

\path[] (106.35, 67.65) --
	(110.62, 67.65);

\path[] (106.35, 80.22) --
	(110.62, 80.22);

\path[] (106.35, 92.79) --
	(110.62, 92.79);

\path[] (106.35,105.36) --
	(110.62,105.36);

\path[] (106.35,117.92) --
	(110.62,117.92);

\path[] (106.35,130.49) --
	(110.62,130.49);
\end{scope}
  \end{tikzpicture}}{INE, con datos del BANGUAT.}{\notitasin{Los datos del año 2014 se presentan como preliminares y serán ajustados por el registro tardío de los mismos.}}}
{\columna{Importaciones 10 principales productos trimestre 2014}{El principal producto de importación para Guatemala del resto del mundo lo constituyó en el primer  trimestre 2014 es el Diesel oil con 34.6\%  de porcentaje en esta serie y US\$423,184,782  seguido de  Gasolina con un 27.5\%  US\$336,131,521 y Medicamentos p/humanos 8.2\%  US\$100,744,903, continuandocon  Gas propano y teléfonos celulares con un 6.9\% y 6.2\% respectivamente}{}{Importación de los principales productos Primer trimestre 2014}{(Cifras preliminares expresadas en US\$)}{\ \\[6mm]\begin{tikzpicture}[x=1pt,y=1pt,scale=1]  % Created by tikzDevice version 0.7.0 on 2014-12-15 22:10:47
% !TEX encoding = UTF-8 Unicode
\definecolor[named]{fillColor}{rgb}{1.00,1.00,1.00}
\path[use as bounding box,fill=fillColor,fill opacity=0.00] (0,0) rectangle (230.54,138.04);
\begin{scope}
\path[clip] (  0.00,  0.00) rectangle (230.54,138.04);
\definecolor[named]{drawColor}{rgb}{1.00,1.00,1.00}

\path[draw=drawColor,line width= 0.6pt,line join=round,line cap=round] (  0.00,  0.00) rectangle (230.54,138.04);
\end{scope}
\begin{scope}
\path[clip] (  0.00,  0.00) rectangle (230.54,138.04);

\path[] ( 89.93, -2.74) rectangle (210.06,138.04);

\path[] ( 89.93,  5.54) --
	(210.06,  5.54);

\path[] ( 89.93, 19.34) --
	(210.06, 19.34);

\path[] ( 89.93, 33.14) --
	(210.06, 33.14);

\path[] ( 89.93, 46.95) --
	(210.06, 46.95);

\path[] ( 89.93, 60.75) --
	(210.06, 60.75);

\path[] ( 89.93, 74.55) --
	(210.06, 74.55);

\path[] ( 89.93, 88.35) --
	(210.06, 88.35);

\path[] ( 89.93,102.15) --
	(210.06,102.15);

\path[] ( 89.93,115.95) --
	(210.06,115.95);

\path[] ( 89.93,129.75) --
	(210.06,129.75);
\definecolor[named]{drawColor}{rgb}{0.00,0.00,0.00}

\path[draw=drawColor,line width= 0.6pt,line join=round] ( 89.93,  1.40) rectangle (100.44,  9.68);

\path[draw=drawColor,line width= 0.6pt,line join=round] ( 89.93, 15.20) rectangle (101.01, 23.48);

\path[draw=drawColor,line width= 0.6pt,line join=round] ( 89.93, 29.00) rectangle (101.01, 37.28);

\path[draw=drawColor,line width= 0.6pt,line join=round] ( 89.93, 42.81) rectangle (102.14, 51.09);

\path[draw=drawColor,line width= 0.6pt,line join=round] ( 89.93, 56.61) rectangle (102.71, 64.89);

\path[draw=drawColor,line width= 0.6pt,line join=round] ( 89.93, 70.41) rectangle (111.23, 78.69);

\path[draw=drawColor,line width= 0.6pt,line join=round] ( 89.93, 84.21) rectangle (114.07, 92.49);

\path[draw=drawColor,line width= 0.6pt,line join=round] ( 89.93, 98.01) rectangle (118.61,106.29);

\path[draw=drawColor,line width= 0.6pt,line join=round] ( 89.93,111.81) rectangle (185.35,120.09);

\path[draw=drawColor,line width= 0.6pt,line join=round] ( 89.93,125.61) rectangle (210.06,133.90);

\node[text=drawColor,anchor=base west,inner sep=0pt, outer sep=0pt, scale=  0.85] at (104.69,  2.61) {37};

\node[text=drawColor,anchor=base west,inner sep=0pt, outer sep=0pt, scale=  0.85] at (105.26, 16.41) {39};

\node[text=drawColor,anchor=base west,inner sep=0pt, outer sep=0pt, scale=  0.85] at (105.26, 30.22) {39};

\node[text=drawColor,anchor=base west,inner sep=0pt, outer sep=0pt, scale=  0.85] at (106.39, 44.02) {43};

\node[text=drawColor,anchor=base west,inner sep=0pt, outer sep=0pt, scale=  0.85] at (106.96, 57.82) {45};

\node[text=drawColor,anchor=base west,inner sep=0pt, outer sep=0pt, scale=  0.85] at (115.48, 71.62) {75};

\node[text=drawColor,anchor=base west,inner sep=0pt, outer sep=0pt, scale=  0.85] at (118.32, 85.42) {85};

\node[text=drawColor,anchor=base west,inner sep=0pt, outer sep=0pt, scale=  0.85] at (124.99, 99.22) {101};

\node[text=drawColor,anchor=base west,inner sep=0pt, outer sep=0pt, scale=  0.85] at (191.73,113.02) {336};

\node[text=drawColor,anchor=base west,inner sep=0pt, outer sep=0pt, scale=  0.85] at (216.43,126.83) {423};
\end{scope}
\begin{scope}
\path[clip] (  0.00,  0.00) rectangle (230.54,138.04);
\definecolor[named]{drawColor}{rgb}{0.00,0.00,0.00}

\path[draw=drawColor,line width= 0.6pt,line join=round] ( 89.93,  0.00) --
	( 89.93,138.04);
\end{scope}
\begin{scope}
\path[clip] (  0.00,  0.00) rectangle (230.54,138.04);
\definecolor[named]{drawColor}{rgb}{0.00,0.00,0.00}

\node[text=drawColor,anchor=base east,inner sep=0pt, outer sep=0pt, scale=  1.00] at ( 82.82,  2.10) {Maíz amarillo};

\node[text=drawColor,anchor=base east,inner sep=0pt, outer sep=0pt, scale=  1.00] at ( 82.82, 15.90) {Harina de soya};

\node[text=drawColor,anchor=base east,inner sep=0pt, outer sep=0pt, scale=  1.00] at ( 82.82, 29.70) {Papel y cartón crudos};

\node[text=drawColor,anchor=base east,inner sep=0pt, outer sep=0pt, scale=  1.00] at ( 82.82, 43.50) {Vehiculos de carga};

\node[text=drawColor,anchor=base east,inner sep=0pt, outer sep=0pt, scale=  1.00] at ( 82.82, 57.30) {Fuel oil (Bunker C)};

\node[text=drawColor,anchor=base east,inner sep=0pt, outer sep=0pt, scale=  1.00] at ( 82.82, 71.11) {Teléfonos celulares};

\node[text=drawColor,anchor=base east,inner sep=0pt, outer sep=0pt, scale=  1.00] at ( 82.82, 84.91) {Gas propano};

\node[text=drawColor,anchor=base east,inner sep=0pt, outer sep=0pt, scale=  1.00] at ( 82.82, 98.71) {Medicamentos P/Humanos};

\node[text=drawColor,anchor=base east,inner sep=0pt, outer sep=0pt, scale=  1.00] at ( 82.82,112.51) {Gasolina};

\node[text=drawColor,anchor=base east,inner sep=0pt, outer sep=0pt, scale=  1.00] at ( 82.82,126.31) {Diesel oil };
\end{scope}
\begin{scope}
\path[clip] (  0.00,  0.00) rectangle (230.54,138.04);

\path[] ( 85.66,  5.54) --
	( 89.93,  5.54);

\path[] ( 85.66, 19.34) --
	( 89.93, 19.34);

\path[] ( 85.66, 33.14) --
	( 89.93, 33.14);

\path[] ( 85.66, 46.95) --
	( 89.93, 46.95);

\path[] ( 85.66, 60.75) --
	( 89.93, 60.75);

\path[] ( 85.66, 74.55) --
	( 89.93, 74.55);

\path[] ( 85.66, 88.35) --
	( 89.93, 88.35);

\path[] ( 85.66,102.15) --
	( 89.93,102.15);

\path[] ( 85.66,115.95) --
	( 89.93,115.95);

\path[] ( 85.66,129.75) --
	( 89.93,129.75);
\end{scope}
  \end{tikzpicture}}{INE, con datos del BANGUAT.}{\notitasin{Los datos del año 2014 se presentan como preliminares y serán ajustados por el registro tardío de los mismos.}}}
\hojados{\columna{Importaciones 10 principales países trimestre 2014}{El principal socio comercial de Guatemala es Estados unidos de donde se importó en el primer trimestre 2014,  US\$1,720,385,540 con un porcentaje de 39.3\% seguido de China  a quien se le compró US\$536,462,801 con un porcentaje de 12.3\%, en tercer lugar Mexico  País al que se le compró US\$431,480,501 con un porcentaje de 9.9\%, continuando con países  como El Salvador y Corea del Sur con un porcentaje de 3.4\% y 3.2\% respectivamente}{}{Importación a los principales 10 países, Primer Trimestre 2014}{(cifras preliminares en US\$)}{\ \\[6mm]\begin{tikzpicture}[x=1pt,y=1pt,scale=1]  % Created by tikzDevice version 0.7.0 on 2014-12-15 22:10:50
% !TEX encoding = UTF-8 Unicode
\definecolor[named]{fillColor}{rgb}{1.00,1.00,1.00}
\path[use as bounding box,fill=fillColor,fill opacity=0.00] (0,0) rectangle (230.54,138.04);
\begin{scope}
\path[clip] (  0.00,  0.00) rectangle (230.54,138.04);
\definecolor[named]{drawColor}{rgb}{1.00,1.00,1.00}

\path[draw=drawColor,line width= 0.6pt,line join=round,line cap=round] (  0.00, -0.00) rectangle (230.54,138.04);
\end{scope}
\begin{scope}
\path[clip] (  0.00,  0.00) rectangle (230.54,138.04);

\path[] (  5.80, 57.30) rectangle (230.54,123.81);

\path[] ( 19.02, 57.30) --
	( 19.02,123.81);

\path[] ( 41.05, 57.30) --
	( 41.05,123.81);

\path[] ( 63.08, 57.30) --
	( 63.08,123.81);

\path[] ( 85.12, 57.30) --
	( 85.12,123.81);

\path[] (107.15, 57.30) --
	(107.15,123.81);

\path[] (129.19, 57.30) --
	(129.19,123.81);

\path[] (151.22, 57.30) --
	(151.22,123.81);

\path[] (173.25, 57.30) --
	(173.25,123.81);

\path[] (195.29, 57.30) --
	(195.29,123.81);

\path[] (217.32, 57.30) --
	(217.32,123.81);
\definecolor[named]{drawColor}{rgb}{0.00,0.00,0.00}

\path[draw=drawColor,line width= 0.6pt,line join=round] ( 12.41, 57.30) rectangle ( 25.63,123.81);

\path[draw=drawColor,line width= 0.6pt,line join=round] ( 34.44, 57.30) rectangle ( 47.66, 78.03);

\path[draw=drawColor,line width= 0.6pt,line join=round] ( 56.47, 57.30) rectangle ( 69.69, 73.97);

\path[draw=drawColor,line width= 0.6pt,line join=round] ( 78.51, 57.30) rectangle ( 91.73, 63.10);

\path[draw=drawColor,line width= 0.6pt,line join=round] (100.54, 57.30) rectangle (113.76, 62.72);

\path[draw=drawColor,line width= 0.6pt,line join=round] (122.58, 57.30) rectangle (135.80, 61.44);

\path[draw=drawColor,line width= 0.6pt,line join=round] (144.61, 57.30) rectangle (157.83, 61.40);

\path[draw=drawColor,line width= 0.6pt,line join=round] (166.64, 57.30) rectangle (179.86, 61.01);

\path[draw=drawColor,line width= 0.6pt,line join=round] (188.68, 57.30) rectangle (201.90, 60.94);

\path[draw=drawColor,line width= 0.6pt,line join=round] (210.71, 57.30) rectangle (223.93, 60.20);

\node[text=drawColor,anchor=base,inner sep=0pt, outer sep=0pt, scale=  0.85] at ( 19.02,126.74) {1,720};

\node[text=drawColor,anchor=base,inner sep=0pt, outer sep=0pt, scale=  0.85] at ( 41.05, 80.96) {536};

\node[text=drawColor,anchor=base,inner sep=0pt, outer sep=0pt, scale=  0.85] at ( 63.08, 76.90) {431};

\node[text=drawColor,anchor=base,inner sep=0pt, outer sep=0pt, scale=  0.85] at ( 85.12, 66.03) {150};

\node[text=drawColor,anchor=base,inner sep=0pt, outer sep=0pt, scale=  0.85] at (107.15, 65.64) {140};

\node[text=drawColor,anchor=base,inner sep=0pt, outer sep=0pt, scale=  0.85] at (129.19, 64.37) {107};

\node[text=drawColor,anchor=base,inner sep=0pt, outer sep=0pt, scale=  0.85] at (151.22, 64.33) {106};

\node[text=drawColor,anchor=base,inner sep=0pt, outer sep=0pt, scale=  0.85] at (173.25, 63.94) {96};

\node[text=drawColor,anchor=base,inner sep=0pt, outer sep=0pt, scale=  0.85] at (195.29, 63.86) {94};

\node[text=drawColor,anchor=base,inner sep=0pt, outer sep=0pt, scale=  0.85] at (217.32, 63.13) {75};
\end{scope}
\begin{scope}
\path[clip] (  0.00,  0.00) rectangle (230.54,138.04);

\path[] (  5.80, 57.30) --
	(  5.80,123.81);
\end{scope}
\begin{scope}
\path[clip] (  0.00,  0.00) rectangle (230.54,138.04);
\definecolor[named]{drawColor}{rgb}{0.00,0.00,0.00}

\path[draw=drawColor,line width= 0.6pt,line join=round] (  5.80, 57.30) --
	(230.54, 57.30);
\end{scope}
\begin{scope}
\path[clip] (  0.00,  0.00) rectangle (230.54,138.04);

\path[] ( 19.02, 53.03) --
	( 19.02, 57.30);

\path[] ( 41.05, 53.03) --
	( 41.05, 57.30);

\path[] ( 63.08, 53.03) --
	( 63.08, 57.30);

\path[] ( 85.12, 53.03) --
	( 85.12, 57.30);

\path[] (107.15, 53.03) --
	(107.15, 57.30);

\path[] (129.19, 53.03) --
	(129.19, 57.30);

\path[] (151.22, 53.03) --
	(151.22, 57.30);

\path[] (173.25, 53.03) --
	(173.25, 57.30);

\path[] (195.29, 53.03) --
	(195.29, 57.30);

\path[] (217.32, 53.03) --
	(217.32, 57.30);
\end{scope}
\begin{scope}
\path[clip] (  0.00,  0.00) rectangle (230.54,138.04);
\definecolor[named]{drawColor}{rgb}{0.00,0.00,0.00}

\node[text=drawColor,rotate= 90.00,anchor=base east,inner sep=0pt, outer sep=0pt, scale=  1.00] at ( 22.46, 50.19) {USA};

\node[text=drawColor,rotate= 90.00,anchor=base east,inner sep=0pt, outer sep=0pt, scale=  1.00] at ( 44.49, 50.19) {China};

\node[text=drawColor,rotate= 90.00,anchor=base east,inner sep=0pt, outer sep=0pt, scale=  1.00] at ( 66.53, 50.19) {México};

\node[text=drawColor,rotate= 90.00,anchor=base east,inner sep=0pt, outer sep=0pt, scale=  1.00] at ( 88.56, 50.19) {El Salvador};

\node[text=drawColor,rotate= 90.00,anchor=base east,inner sep=0pt, outer sep=0pt, scale=  1.00] at (110.60, 50.19) {Corea del Sur};

\node[text=drawColor,rotate= 90.00,anchor=base east,inner sep=0pt, outer sep=0pt, scale=  1.00] at (132.63, 50.19) {Costa Rica};

\node[text=drawColor,rotate= 90.00,anchor=base east,inner sep=0pt, outer sep=0pt, scale=  1.00] at (154.66, 50.19) {Japón};

\node[text=drawColor,rotate= 90.00,anchor=base east,inner sep=0pt, outer sep=0pt, scale=  1.00] at (176.70, 50.19) {Colombia};

\node[text=drawColor,rotate= 90.00,anchor=base east,inner sep=0pt, outer sep=0pt, scale=  1.00] at (198.73, 50.19) {Alemania};

\node[text=drawColor,rotate= 90.00,anchor=base east,inner sep=0pt, outer sep=0pt, scale=  1.00] at (220.76, 50.19) {Perú};
\end{scope}
  \end{tikzpicture}}{INE, con datos del BANGUAT.}{\notitasin{Los datos del año 2014 se presentan como preliminares y serán ajustados por el registro tardío de los mismos.}}}
{\columna{Importaciones 10 principales secciones}{De las secciones del Sistema Arancelario Centroaméricano (SAC) en el primer trimestre 2014 en las importaciones a Guatemala a productos minerales ocupa el primer lugar con un 24.9\%, el segundo lugar con los productos de las Industrias der m áquinas y aparatos con 16.6\%,  en tercer lugar los productos de las Industrias Químicas con un 13.4\% continuando con materias plásticas y metales comunes con un 7.5\% y 7.0\% respectivamente}{}{Importación de los Principales Secciones del Sistema Arancelario C.A., Primer Trimestre 2014}{(Cifras expresadas en US\$ )}{\ \\[6mm]\begin{tikzpicture}[x=1pt,y=1pt,scale=1]  % Created by tikzDevice version 0.7.0 on 2014-12-15 22:10:53
% !TEX encoding = UTF-8 Unicode
\definecolor[named]{fillColor}{rgb}{1.00,1.00,1.00}
\path[use as bounding box,fill=fillColor,fill opacity=0.00] (0,0) rectangle (230.54,138.04);
\begin{scope}
\path[clip] (  0.00,  0.00) rectangle (230.54,138.04);
\definecolor[named]{drawColor}{rgb}{1.00,1.00,1.00}

\path[draw=drawColor,line width= 0.6pt,line join=round,line cap=round] (  0.00,  0.00) rectangle (230.54,138.04);
\end{scope}
\begin{scope}
\path[clip] (  0.00,  0.00) rectangle (230.54,138.04);

\path[] ( 93.26, -2.74) rectangle (206.64,138.04);

\path[] ( 93.26,  4.80) --
	(206.64,  4.80);

\path[] ( 93.26, 17.37) --
	(206.64, 17.37);

\path[] ( 93.26, 29.94) --
	(206.64, 29.94);

\path[] ( 93.26, 42.51) --
	(206.64, 42.51);

\path[] ( 93.26, 55.08) --
	(206.64, 55.08);

\path[] ( 93.26, 67.65) --
	(206.64, 67.65);

\path[] ( 93.26, 80.22) --
	(206.64, 80.22);

\path[] ( 93.26, 92.79) --
	(206.64, 92.79);

\path[] ( 93.26,105.36) --
	(206.64,105.36);

\path[] ( 93.26,117.92) --
	(206.64,117.92);

\path[] ( 93.26,130.49) --
	(206.64,130.49);
\definecolor[named]{drawColor}{rgb}{0.00,0.00,0.00}

\path[draw=drawColor,line width= 0.6pt,line join=round] ( 93.26,  1.03) rectangle (103.90,  8.57);

\path[draw=drawColor,line width= 0.6pt,line join=round] ( 93.26, 13.60) rectangle (110.77, 21.14);

\path[draw=drawColor,line width= 0.6pt,line join=round] ( 93.26, 26.17) rectangle (113.10, 33.71);

\path[draw=drawColor,line width= 0.6pt,line join=round] ( 93.26, 38.74) rectangle (121.30, 46.28);

\path[draw=drawColor,line width= 0.6pt,line join=round] ( 93.26, 51.31) rectangle (124.74, 58.85);

\path[draw=drawColor,line width= 0.6pt,line join=round] ( 93.26, 63.88) rectangle (124.96, 71.42);

\path[draw=drawColor,line width= 0.6pt,line join=round] ( 93.26, 76.45) rectangle (125.29, 83.99);

\path[draw=drawColor,line width= 0.6pt,line join=round] ( 93.26, 89.02) rectangle (127.29, 96.56);

\path[draw=drawColor,line width= 0.6pt,line join=round] ( 93.26,101.59) rectangle (154.44,109.13);

\path[draw=drawColor,line width= 0.6pt,line join=round] ( 93.26,114.15) rectangle (168.85,121.70);

\path[draw=drawColor,line width= 0.6pt,line join=round] ( 93.26,126.72) rectangle (206.64,134.26);

\node[text=drawColor,anchor=base west,inner sep=0pt, outer sep=0pt, scale=  0.85] at (108.15,  1.87) {96};

\node[text=drawColor,anchor=base west,inner sep=0pt, outer sep=0pt, scale=  0.85] at (117.15, 14.44) {158};

\node[text=drawColor,anchor=base west,inner sep=0pt, outer sep=0pt, scale=  0.85] at (119.48, 27.01) {179};

\node[text=drawColor,anchor=base west,inner sep=0pt, outer sep=0pt, scale=  0.85] at (127.68, 39.58) {253};

\node[text=drawColor,anchor=base west,inner sep=0pt, outer sep=0pt, scale=  0.85] at (131.11, 52.15) {284};

\node[text=drawColor,anchor=base west,inner sep=0pt, outer sep=0pt, scale=  0.85] at (131.34, 64.72) {286};

\node[text=drawColor,anchor=base west,inner sep=0pt, outer sep=0pt, scale=  0.85] at (131.67, 77.29) {289};

\node[text=drawColor,anchor=base west,inner sep=0pt, outer sep=0pt, scale=  0.85] at (133.66, 89.86) {307};

\node[text=drawColor,anchor=base west,inner sep=0pt, outer sep=0pt, scale=  0.85] at (160.82,102.43) {552};

\node[text=drawColor,anchor=base west,inner sep=0pt, outer sep=0pt, scale=  0.85] at (175.22,115.00) {682};

\node[text=drawColor,anchor=base west,inner sep=0pt, outer sep=0pt, scale=  0.85] at (216.32,127.57) {1,023};
\end{scope}
\begin{scope}
\path[clip] (  0.00,  0.00) rectangle (230.54,138.04);
\definecolor[named]{drawColor}{rgb}{0.00,0.00,0.00}

\path[draw=drawColor,line width= 0.6pt,line join=round] ( 93.26,  0.00) --
	( 93.26,138.04);
\end{scope}
\begin{scope}
\path[clip] (  0.00,  0.00) rectangle (230.54,138.04);
\definecolor[named]{drawColor}{rgb}{0.00,0.00,0.00}

\node[text=drawColor,anchor=base east,inner sep=0pt, outer sep=0pt, scale=  1.00] at ( 86.15,  1.36) {Reino animal};

\node[text=drawColor,anchor=base east,inner sep=0pt, outer sep=0pt, scale=  1.00] at ( 86.15, 13.93) {Reino Vegetal};

\node[text=drawColor,anchor=base east,inner sep=0pt, outer sep=0pt, scale=  1.00] at ( 86.15, 26.50) {Pastas de Madera};

\node[text=drawColor,anchor=base east,inner sep=0pt, outer sep=0pt, scale=  1.00] at ( 86.15, 39.07) {Material de Transporte};

\node[text=drawColor,anchor=base east,inner sep=0pt, outer sep=0pt, scale=  1.00] at ( 86.15, 51.64) {Textiles y sus manufacturas};

\node[text=drawColor,anchor=base east,inner sep=0pt, outer sep=0pt, scale=  1.00] at ( 86.15, 64.20) {Industrias Alimenticias};

\node[text=drawColor,anchor=base east,inner sep=0pt, outer sep=0pt, scale=  1.00] at ( 86.15, 76.77) {Metales Comunes};

\node[text=drawColor,anchor=base east,inner sep=0pt, outer sep=0pt, scale=  1.00] at ( 86.15, 89.34) {Materias Plásticas};

\node[text=drawColor,anchor=base east,inner sep=0pt, outer sep=0pt, scale=  1.00] at ( 86.15,101.91) {Industrias Químicas};

\node[text=drawColor,anchor=base east,inner sep=0pt, outer sep=0pt, scale=  1.00] at ( 86.15,114.48) {Máquinas y Aparatos};

\node[text=drawColor,anchor=base east,inner sep=0pt, outer sep=0pt, scale=  1.00] at ( 86.15,127.05) {Productos Minerales};
\end{scope}
\begin{scope}
\path[clip] (  0.00,  0.00) rectangle (230.54,138.04);

\path[] ( 88.99,  4.80) --
	( 93.26,  4.80);

\path[] ( 88.99, 17.37) --
	( 93.26, 17.37);

\path[] ( 88.99, 29.94) --
	( 93.26, 29.94);

\path[] ( 88.99, 42.51) --
	( 93.26, 42.51);

\path[] ( 88.99, 55.08) --
	( 93.26, 55.08);

\path[] ( 88.99, 67.65) --
	( 93.26, 67.65);

\path[] ( 88.99, 80.22) --
	( 93.26, 80.22);

\path[] ( 88.99, 92.79) --
	( 93.26, 92.79);

\path[] ( 88.99,105.36) --
	( 93.26,105.36);

\path[] ( 88.99,117.92) --
	( 93.26,117.92);

\path[] ( 88.99,130.49) --
	( 93.26,130.49);
\end{scope}
  \end{tikzpicture}}{INE, con datos del BANGUAT.}{\notitasin{Los datos del año 2014 se presentan como preliminares y serán ajustados por el registro tardío de los mismos.}}}


\end{document}