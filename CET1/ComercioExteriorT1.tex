
\documentclass[10pt,twoside]{book}

%\usepackage[utf8]{inputenc}  %Para compilar en PdfLaTeX


%Recordatorio de relleno

	%macro del capítulo
	%\INEchapter{TÍTULO: SUBTÍTULO (para índice)}{TÍTULO:}{SUBTÍTULO}{Descripción}
	%Hoja
	
	%caja de media hoja
	%\cajita{Título}{Descripción}{Subtítulo}{Desagregación}{Gráfica}{Fuente}{\notita{Nota}}
	
	%caja de hoja completa
	%\cajota{Título}{Descripción}{Subtítulo}{Desagregación}{Gráfica}{Fuente}{\notita{Nota}}



%Paquetes estándar
\usepackage{amsmath}
\usepackage{amsfonts}
\usepackage{amssymb}
\usepackage{graphicx}
\usepackage{pdfpages}
\usepackage{setspace} 
\usepackage{xltxtra}
\usepackage{enumitem}

\usepackage{soul} %faby... lo puse para tachar
%Tablas de Excel convertidas a LaTeX
\usepackage{booktabs}
\usepackage{multirow}
\newcounter{Cuadro}[chapter]
\renewcommand{\theCuadro}{\thechapter.\arabic{Cuadro}}


%Columnas definibles en ancho
\usepackage{array}
	\newcolumntype{x}[1]{%
	>{\centering\arraybackslash}p{#1}}%
	
	\newcolumntype{g}[1]{%
	>{\raggedleft\arraybackslash}p{#1}}%


\usepackage[input-decimal-markers={.}, input-ignore={,}, group-separator={,}]{siunitx}


%Para pruebas
\usepackage{lipsum}
\newcommand{\comop}{}
\newcommand{\comcl}{}
\newcommand{\guiop}{}
\newcommand{\guicl}{}
\newcommand{\apartado}[1]{\addtocounter{section}{1}
{\noindent\Bold\Large\color{color1!80!black}\thesection $\,-$ #1}\\[2mm]}


%Para compilar en XeLaTeX con tildes
\usepackage{polyglossia}
	\setmainlanguage{spanish}

%Tipo de letra
\usepackage{fontspec}
	\setmainfont[
		BoldFont = OpenSans-CondBold.ttf ,
		ItalicFont = texgyrepagella-italic.otf ,
		BoldItalicFont = OpenSans-CondLightItalic.ttf ]{OpenSans-CondLight.ttf}
	\newfontfamily\Bold{Open Sans Condensed Bold}
	
	\newfontfamily\Sans{Open Sans}
	\newfontfamily\Italic{Open Sans Condensed Light Italic}
	\newfontfamily\Logos{Latin Modern Roman}
	


%Diseño global del documento
\usepackage[paperwidth=8.5in, paperheight=6.5in, left=0.950in, right=0.8in, top=0.525in, bottom=0.675in]{geometry}
%	\setlength{\headsep}{0pt}
%	\setlength{\footskip}{46pt}
	\setlength{\parindent}{2em}		%sangría
	\setlength{\parskip}{2ex}		%separación entre párrafos  
	
	%Distancias
		\newlength{\cuadri} 
		\setlength{\cuadri}{0.125in}




%Tabla de contenidos y vinculaciones
\usepackage{tocloft}
\usepackage[hidelinks]{hyperref}
\usepackage{url}

	%Formato de  de contenidos
	\setlength{\cftbeforetoctitleskip}{0em}
	\AtBeginDocument{\addtocontents{toc}{\protect\thispagestyle{empty}}} 

	\makeatletter
		\renewcommand*\l@subsection{\@dottedtocline{2}{5.2em}{3.2em}}
	\makeatother
	
	\renewcommand{\thesection}{\thechapter.\arabic{section}}
	
	\cftsetpnumwidth{2\cuadri}
	\cftsetrmarg{8\cuadri}
	\renewcommand{\cftsecnumwidth}{2.0\cuadri}
	\renewcommand{\cftchapnumwidth}{2\cuadri}
	\renewcommand{\cftsecindent}{2\cuadri}



% Elementos geométricos de diseño del cuerpo (cajas de colores, etc.)
\usepackage{colortbl}


%\usepackage[usenames,dvipsnames,svgnames,table]{xcolor}

	%Cambios de márgenes y según paridad de hojas
		\usepackage[strict]{changepage}
			\strictpagecheck

	%Colores base del documento
		\definecolor{color1}{rgb}{0,0,0}
		\definecolor{color2}{rgb}{0,0,0}


	%Para que las páginas en blanco no estén numeradas
		\let\origdoublepage\cleardoublepage
		\newcommand{\clearemptydoublepage}{
  		\clearpage
  		{\pagestyle{empty}\origdoublepage}}
		\let\cleardoublepage\clearemptydoublepage
		
	%Llamadas hacia notas
		\newcommand{\llamada}{*$\ $}
		\newcommand{\llamadaD}{**$\ $}
	%tablas
		\usepackage{pdflscape}
		\usepackage{rotating}
		\usepackage{bigstrut}
		\usepackage{longtable}
		\LTcapwidth=1.234\textwidth
		\setlength{\arrayrulewidth}{0.8pt}
		\arrayrulecolor{color1!80!black}
		
		%Columnas centradas definibles en ancho
		\newcolumntype{x}[1]{%
		>{\centering\arraybackslash}p{#1}}%



% Tcolorbox
\usepackage[skins, breakable, hooks]{tcolorbox}


\newtcolorbox{tocbox}{skin=enhancedmiddle, width=39\cuadri, nobeforeafter, boxrule=0pt, colframe=white, left=0\cuadri, enlarge left by = 4\cuadri, enlarge right by=2\cuadri, bottom=0pt, top=0pt, right=0\cuadri, left=0\cuadri, arc=0pt, colback= white, breakable,check odd page,toggle left and right}

\newtcolorbox{fondo}{width=6.75in, height=5.3in, nobeforeafter, enlarge left by=-0mm, enlarge top by=-0mm, boxrule=0pt, colframe=white, left=-10pt, bottom=-1pt, top=-3pt, right=-10pt, arc=0pt, colback= white}

\newtcolorbox{columnatipoA}{width=3.19in, height=5.28in, nobeforeafter, enlarge left by=0.15pt, enlarge top by=-0mm, boxrule=0pt, colframe=white, left=-3pt, bottom=-1pt, top=0.9pt, right=-3pt, arc=0pt, colback= white}

\newtcolorbox{fondo-landscape}{width=39\cuadri, height=38\cuadri, nobeforeafter, enlarge left by=-3mm,
enlarge bottom by=-40mm, enlarge top by=-11mm, boxrule=0pt, colframe=white, left=0pt, bottom=0pt, top=0pt, right=0pt, arc=0pt, colback= white}

\newtcolorbox{bloque-media}{width=40\cuadri, height=19\cuadri, enlarge top by=-3pt, enlarge left by=-3pt, enlarge bottom by=1.4\cuadri, nobeforeafter, colframe=white, colback=white, left=0pt, right=0pt, bottom= 0pt, top=0pt, arc=0pt, boxrule=0pt}

\newtcolorbox{bloque-una}{width=40\cuadri, height=1\cuadri, enlarge top by=-3pt, enlarge left by=-3pt, enlarge bottom by=1.4\cuadri, nobeforeafter, colframe=white, colback=white, left=0pt, right=0pt, bottom= 0pt, top=0pt, arc=0pt, boxrule=0pt}

\newtcolorbox{descripcion}{width=10.5\cuadri, height=15.9\cuadri, enlarge bottom by=0\cuadri, enlarge top by=-3pt, enlarge left by=-3pt, nobeforeafter, boxrule=3pt, colback=color1!8!white, colframe=color1!8!white, left=5pt, right=5pt, top=4pt,bottom=4pt}

\newtcolorbox{descripcion-una}{width=33.3\cuadri, height=6.2\cuadri, enlarge bottom by=1\cuadri, enlarge top by=-5pt, enlarge left by=-3pt, nobeforeafter, boxrule=3pt, colback=color1!8!white, colframe=color1!8!white, left=5pt, right=5pt, top=3pt,bottom=4pt}




\newtcolorbox{descripcion-titulin}{width=33.3\cuadri, enlarge bottom by=1\cuadri, enlarge top by=-5pt, enlarge left by=-3pt, nobeforeafter, boxrule=3pt, colback=white, colframe=white, left=5pt, right=5pt, top=3pt,bottom=4pt}

\newtcolorbox{titulin}{width=33.3\cuadri, height=3.2\cuadri, enlarge bottom by=1\cuadri, enlarge top by=0pt, enlarge left by=-3pt, nobeforeafter, boxrule=0pt, colback=white, colframe=white, left=5pt, right=5pt, top=6pt, bottom=4pt}


\newcommand{\titulito}[2]{
\addtocontents{toc}{\protect\addvspace{0.4\baselineskip}\color{black}}
\addcontentsline{toc}{section}{\textbf{#1}}
\begin{bloque-media}
$\ $\\[1.4cm]
\begin{titulin}
\begin{center}
{\Huge\Bold\color{color1!90!black} #1}\\[1.5cm]
\end{center}
\end{titulin}

\begin{tabular}{x{33.3\cuadri}}
\hline
$\ $\\
\begin{descripcion-titulin}
#2
\end{descripcion-titulin} \\
$\ $\\[-0.9cm] \hline 
\end{tabular}
\end{bloque-media} 
}

\newcommand{\titulitond}[1]{
\addtocontents{toc}{\protect\addvspace{0.4\baselineskip}\color{black}}
\addcontentsline{toc}{section}{\textbf{#1}}
\begin{bloque-media}
$\ $\\[1.4cm]
\begin{titulin}
\begin{center}
{\Huge\Bold\color{color1!90!black} $\ $}\\[1.5cm]
\end{center}
\end{titulin}

\begin{tabular}{x{33.3\cuadri}}
\hline
$\ $\\
\begin{descripcion-titulin}
\begin{center}
{\Huge\Bold\color{color1!90!black}#1}
\end{center}
\end{descripcion-titulin} \\
$\ $\\[-0.9cm] \hline 
\end{tabular}
\end{bloque-media} 
}



\newtcolorbox{notita-impar}{width=5.9\cuadri, enlarge bottom by=0\cuadri, enlarge top by=1\cuadri, enlarge left by=0.1\cuadri, nobeforeafter, boxrule=0pt, colback=white, colframe=color2!40!white, left=3pt, right=1pt, top=2pt,bottom=2pt, leftrule=0.6pt, rightrule=0pt,toprule=0pt,bottomrule=0pt, arc=0pt}

\newtcolorbox{notita-par}{width=5.9\cuadri, enlarge bottom by=0\cuadri, enlarge top by=1\cuadri, enlarge left by=-0.55\cuadri, nobeforeafter, boxrule=0pt, colback=white, colframe=color2!40!white, left=1pt, right=3pt, top=2pt,bottom=2pt, rightrule=0.6pt, leftrule=0pt,toprule=0pt,bottomrule=0pt, arc=0pt}

\newtcolorbox{numero-subseccion}{width=2.6\cuadri, height=1.4\cuadri, enlarge top by=-3pt, enlarge left by=-2.8\cuadri, enlarge right by= 4.613\cuadri, nobeforeafter, boxrule=0pt, colback=white, colframe=white, left=0.1\cuadri, right=0.1\cuadri, bottom= 0.455\cuadri, top=0.145\cuadri, arc=0pt}

\newtcolorbox{titulo-subseccion}{width=28.8\cuadri, enlarge top by=-3pt, enlarge left by=-0.2\cuadri, enlarge bottom by=0\cuadri, nobeforeafter, boxrule=0pt, colback=white, colframe=white, left=0.1\cuadri, right=0.1\cuadri, bottom= -0.4\cuadri, top=0\cuadri, arc=0pt}

\newtcolorbox{titulo-subseccion-blanco}{width=28\cuadri, enlarge top by=-0pt, enlarge left by=0pt, enlarge bottom by=0pt, nobeforeafter, colback=white, colframe=white, left=0\cuadri, right=0.5\cuadri, bottom= -4pt, top=-2pt, arc=0pt, bottomrule=0pt, leftrule=0pt, toprule= 0mm, rightrule= 0mm}

\newtcolorbox{titulo-subseccion-continuacion}{width=32\cuadri, enlarge top by=-3pt, enlarge left by=-0.2\cuadri, enlarge bottom by=0.4\cuadri, nobeforeafter, boxrule=0pt, colback=white, colframe=white, left=0.3\cuadri, right=0.1\cuadri, bottom= -0.4\cuadri, top=0\cuadri, arc=0pt}

\newtcolorbox{titulo}{width=34\cuadri, height=3\cuadri, enlarge top by=-3pt, enlarge left by=-3pt, enlarge bottom by=0.2\cuadri, nobeforeafter, colback=white, colframe=white, left=0.5\cuadri, right=0.5\cuadri, top=32pt,bottom=-48pt, arc=0pt, boxrule=0pt}

\newtcolorbox{centrador}{width=34\cuadri, enlarge top by=-48pt, enlarge left by=-3pt, enlarge bottom by=0pt, nobeforeafter, colback=white, colframe=white, left=-3pt, right=-3pt, bottom= -3pt, top=-3pt, arc=0pt, boxrule=0pt}

\newtcolorbox{centrador-par}{width=34\cuadri, enlarge top by=-48pt, enlarge left by=66.7pt, enlarge bottom by=0pt, nobeforeafter, colback=white, colframe=white, left=-3pt, right=-3pt, bottom= -3pt, top=-3pt, arc=0pt, boxrule=0pt}

\newtcolorbox{subtitulo}{width=22\cuadri, height=3\cuadri, enlarge top by=-3pt, enlarge left by=-3pt, enlarge bottom by=0.1\cuadri, nobeforeafter, colframe=white, colback=white, left=0\cuadri, right=0\cuadri, bottom= 0pt, top=0pt, arc=0pt, boxrule=0pt}

\newtcolorbox{subtitulo-una}{width=28\cuadri, height=3\cuadri, enlarge top by=-3pt, enlarge left by=-3pt, enlarge bottom by=0.1\cuadri, nobeforeafter, colframe=white, colback=white, left=0\cuadri, right=0\cuadri, bottom= 0pt, top=0pt, arc=0pt, boxrule=0pt}

\newtcolorbox{grafica}{width=22.1\cuadri, height=12\cuadri, enlarge top by=-3pt, enlarge left by=-5pt, enlarge bottom by=0.1\cuadri, nobeforeafter, colframe=white, colback=white, left=-2pt, right=-2pt, bottom= -2pt, top=-4pt, arc=0pt, boxrule=0pt}

\newtcolorbox{grafica-una}{width=33.4\cuadri, height=24.9\cuadri, enlarge top by=-3pt, enlarge left by=-5pt, enlarge bottom by=0.1\cuadri, nobeforeafter, colframe=white, colback=white, left=-2pt, right=-2pt, bottom= -2pt, top=-4pt, arc=0pt, boxrule=0pt}

\newtcolorbox{fuente}{width=22\cuadri, height=1\cuadri, enlarge top by=-3pt, enlarge left by=-3pt, enlarge bottom by=0\cuadri, nobeforeafter, colframe=white, colback=white, left=0pt, right=0pt, bottom= 0pt, top=2pt, arc=0pt, boxrule=0pt}

\newtcolorbox{fuente-una}{width=33.4\cuadri, height=1\cuadri, enlarge top by=-3pt, enlarge left by=-3pt, enlarge bottom by=0\cuadri, nobeforeafter, colframe=white, colback=white, left=0pt, right=0pt, bottom= 0pt, top=2pt, arc=0pt, boxrule=0pt}

\newtcolorbox{columna-central}{width=22\cuadri, height=15.9\cuadri, enlarge top by=-3pt, enlarge left by=-8pt, enlarge bottom by=0\cuadri, nobeforeafter, colframe=white, colback=white, left=0pt, right=0pt, bottom= 0pt, top=0pt, arc=0pt, boxrule=0pt}

\newtcolorbox{columna-central-una}{width=33.4\cuadri, height=28.9\cuadri, enlarge top by=-3pt, enlarge left by=-4pt, enlarge bottom by=0\cuadri, nobeforeafter, colframe=white, colback=white, left=0pt, right=0pt, bottom= 0pt, top=0pt, arc=0pt, boxrule=0pt}

\newtcolorbox{vacio1}{width=6\cuadri, height=3\cuadri, enlarge top by=-3pt, enlarge left by=-3pt, enlarge bottom by=0.1\cuadri, nobeforeafter, colframe=white, colback=white, left=0pt, right=0pt, bottom= 0pt, top=0pt, arc=0pt, boxrule=0pt}

\newtcolorbox{nota}{width=6\cuadri, height=12\cuadri, enlarge top by=-3pt, enlarge left by=-3pt, enlarge bottom by=0.1\cuadri, nobeforeafter, colframe=white, colback=white, left=0pt, right=0pt, bottom= 0pt, top=0pt, arc=0pt, boxrule=0pt}

\newtcolorbox{nota-una}{width=6\cuadri, height=24.9\cuadri, enlarge top by=-3pt, enlarge left by=-3pt, enlarge bottom by=0.1\cuadri, nobeforeafter, colframe=white, colback=white, left=0pt, right=0pt, bottom= 0pt, top=0pt, arc=0pt, boxrule=0pt}

\newtcolorbox{vacio2}{width=6\cuadri, height=1\cuadri, enlarge top by=-3pt, enlarge left by=-3pt, enlarge bottom by=0\cuadri, nobeforeafter, colframe=white, colback=white, left=0pt, right=0pt, bottom= 0pt, top=0pt, arc=0pt, boxrule=0pt}

\newtcolorbox{columna-marginal}{width=6\cuadri, height=15.9\cuadri, enlarge top by=-3pt, enlarge left by=-8pt, enlarge bottom by=0\cuadri, nobeforeafter, colframe=white, colback=white, left=0pt, right=0pt, bottom= 0pt, top=0pt, arc=0pt, boxrule=0pt}

\newtcolorbox{columna-marginal-una}{width=6\cuadri, height=28.9\cuadri, enlarge top by=-3pt, enlarge left by=-8pt, enlarge bottom by=0\cuadri, nobeforeafter, colframe=white, colback=white, left=0pt, right=0pt, bottom= 0pt, top=0pt, arc=0pt, boxrule=0pt}


% Encabezado y pie de página
\usepackage{fancyhdr}

	\newlength{\nombrecapitulo}

	%cajitas de encabezado y pie de página
	\newtcbox{pagina}{nobeforeafter, boxrule=0pt,width= 3\cuadri, height=2\cuadri,  colback=white, left=2pt, right=2pt,   top=-1pt, bottom=-2pt, arc=0pt, enlarge left by=-10pt, enlarge right by=-10pt, width=2\cuadri, colframe = white}
	
	\newtcbox{piecapituloderecho}{enhanced, nobeforeafter, width=1\cuadri, boxrule=0pt, colback=white, left=0pt, right=0pt, bottom= 0pt, arc=0pt, enlarge left by=-0.5\cuadri, enlarge right by=-0.5\cuadri, enlarge top by=-\nombrecapitulo, enlarge bottom by=1\cuadri}
	
	\newtcbox{piecapituloizquierdo}{enhanced, nobeforeafter, width=1\cuadri, boxrule=0pt, colback=white, left=0pt, right=-3pt, bottom= 0pt, arc=0pt, enlarge left by=-0.5\cuadri, enlarge right by=-0.5\cuadri, enlarge top by=-\nombrecapitulo, enlarge bottom by=1\cuadri}
	
	\newtcbox{margenderecho}{nobeforeafter, height=42\cuadri, width=1\cuadri, boxrule=0pt, colback=white,enlarge left by=4\cuadri, enlarge right by= -5\cuadri, enlarge top by= -50\cuadri, enlarge bottom by=4.2\cuadri, left=-3pt, bottom= 0pt, top=-3pt, arc=0pt}
	
	
	\newtcbox{margenizquierdo}{nobeforeafter, height=42\cuadri, width=1\cuadri, boxrule=0pt, enlarge left by=-5\cuadri, enlarge right by= 4\cuadri,colback=white, enlarge top by= -50\cuadri, enlarge bottom by=4.2\cuadri, right=-3pt, bottom= 0pt, top=-3pt, arc=0pt}
	
	\newtcolorbox{artepieizquierdo}{skin=enhancedmiddle,watermark graphics=pie.png, watermark opacity=1.00, watermark overzoom=0.95, nobeforeafter, height=1\cuadri, width=63\cuadri, boxrule=0pt, enlarge left by=2\cuadri, enlarge right by= -65\cuadri, enlarge top by= -2.3\cuadri, enlarge bottom by=2.4\cuadri, right=-3pt, bottom= 0pt, top=-3pt, arc=0pt, colback=white, colframe=white}
	
	
	\newtcolorbox{artepiederecho}{skin=enhancedmiddle,watermark graphics=pie.png, watermark opacity=1.00,watermark overzoom=0.95, nobeforeafter, height=1\cuadri, width=63\cuadri, boxrule=0pt, enlarge left by=-65\cuadri, enlarge right by= 2\cuadri, enlarge top by= -2.3\cuadri, enlarge bottom by=2.4\cuadri, right=-3pt, bottom= 0pt, top=-3pt, arc=0pt,colback=white, colframe=white}

	%definición de estilo estándar de página
	\fancypagestyle{estandar}{%
	\fancyhf{}
	\fancyfoot[RO]{\settowidth{\nombrecapitulo}{\chaptitle}\setlength{\arrayrulewidth}{0.7pt}\setlength{\tabcolsep}{3pt}\arrayrulecolor{white!60!black}
		\margenderecho{
		\begin{tabular}{x{0.2cm}}
		$\ $\\[41.2\cuadri]
			\piecapituloderecho{\rotatebox[origin=c]{270}{\color{white!60!black}\chaptitle$\vphantom{q}$}} \\[-1\cuadri] \hline \\[-0.6\cuadri]
			\pagina{\textbf{\thepage }}  \\ \begin{artepiederecho}\end{artepiederecho}
		\end{tabular}}
	}
	\fancyfoot[LE]{\settowidth{\nombrecapitulo}{Nombre del documento}\setlength{\arrayrulewidth}{0.7pt}\setlength{\tabcolsep}{3pt}\arrayrulecolor{white!60!black}
			\margenizquierdo{
			\begin{tabular}{x{0.2cm}}
			$\ $\\[41.2\cuadri]
				\piecapituloizquierdo{\rotatebox{90}{\color{white!60!black}Nombre del documento$\vphantom{q}$}} \\[-1\cuadri] \hline \\[-0.6\cuadri]
				\pagina{\textbf{\thepage }}  \\ \begin{artepieizquierdo}\end{artepieizquierdo}
			\end{tabular}}
	}
	\renewcommand{\headrulewidth}{0pt}
	\renewcommand{\footrulewidth}{0pt}}

\pagestyle{empty}


%Estilo de los pies de columna
\makeatletter
\renewcommand\footnoterule{%
  \kern-3\p@
  \color{color2}\hrule\@width2.5cm\@height0.5pt
  \kern2.6\p@}
\makeatother

%Formato de encabezados
	\usepackage[explicit]{titlesec}
	%\usepackage{sectsty}


	\titleformat{\subsection}[runin]{}{}{0pt}{}[]
	\titlespacing{\subsection}{0pt}{-20pt}{0pt}
	
	\titleformat{\section}[runin]{}{}{0pt}{}[]
	\titlespacing{\section}{0pt}{-20pt}{0pt}
		
	\titleformat{\chapter}[runin]{}{}{0pt}{}[]
	\titlespacing{\section}{0pt}{-20pt}{0pt}
		
	%Comandos recolectores de información del pie de página
		\newcommand{\chaptitle}{}
		\newcommand{\subsectitle}{}
		\newcommand{\sectitle}{}
	 	 
	%Cajitas de capítulo
		\newtcolorbox{numero-capitulo}{width=4.5\cuadri, height= 5\cuadri, enlarge top by=-3pt, enlarge left by=-3pt, nobeforeafter, boxrule=0pt, colback=color1!95!black, colframe=color1!95!black, left=0\cuadri, right=0\cuadri, bottom= 0.2\cuadri, top=1.2\cuadri, arc=0pt}	
		
		\newtcolorbox{titulo-capitulo}{enhanced,width=34\cuadri, height= 5\cuadri, enlarge top by=-3pt, enlarge left by=1.5cm, nobeforeafter, boxrule=0pt, interior style={left color=color1!10!white,
		right color=white}, left=0\cuadri, right=-4.5\cuadri, bottom= 0\cuadri, top=0\cuadri, arc=0pt}
		
		\newtcolorbox{numero-capitulo-long}{width=6\cuadri, height= 5\cuadri, enlarge top by=-3pt, enlarge left by=-3pt, nobeforeafter, boxrule=0pt, colback=color1!95!black, colframe=color1!95!black, left=0\cuadri, right=0\cuadri, bottom= 0.2\cuadri, top=1.2\cuadri, arc=0pt}	
		
		\newtcolorbox{titulo-capitulo-long}{enhanced,width=34\cuadri, height= 5\cuadri, enlarge top by=-3pt, enlarge left by=1.5cm, nobeforeafter, boxrule=0pt, interior style={left color=color1!10!white,
		right color=white}, left=0\cuadri, right=-6\cuadri, bottom= 0\cuadri, top=0\cuadri, arc=0pt}
		
		\newtcolorbox{capitulo-descripcion}{width=22\cuadri, enlarge top by=-3pt, enlarge left by=-3pt, nobeforeafter, enlarge right by = 6\cuadri, colback=white, colframe=color1!20!white, left=1.5\cuadri, right=2\cuadri, bottom= 1\cuadri, top= 1\cuadri, arc=0pt, bottomrule=0pt, leftrule=2pt, toprule= 0pt, rightrule= 0pt}
		
		
	%Cajitas de apéndice
		\newtcolorbox{numero-capitulo-app}{width=4.5\cuadri, height= 5\cuadri, enlarge top by=-3pt, enlarge left by=-3pt, nobeforeafter, boxrule=0pt, colback=color2!95!black, colframe=color2!95!black, left=0\cuadri, right=0\cuadri, bottom= 0.2\cuadri, top=1.2\cuadri, arc=0pt}	
		
		\newtcolorbox{titulo-capitulo-app}{enhanced,width=34\cuadri, height= 5\cuadri, enlarge top by=-3pt, enlarge left by=1.5cm, nobeforeafter, boxrule=0pt, interior style={left color=color2!10!white,
		right color=white}, left=0\cuadri, right=-4\cuadri, bottom= 0\cuadri, top=0\cuadri, arc=0pt}
		
		\newtcolorbox{capitulo-descripcion-app}{width=22\cuadri, enlarge top by=-3pt, enlarge left by=-3pt, nobeforeafter, enlarge right by = 6\cuadri, colback=white, colframe=color2!20!white, left=1.5\cuadri, right=2\cuadri, bottom= 1\cuadri, top= 1\cuadri, arc=0pt, bottomrule=0pt, leftrule=2pt, toprule= 0pt, rightrule= 0pt}


%macro de sección	
	 	
	\newcommand{\subsecnew}[1]{\renewcommand{\subsectitle}{#1} \subsection{#1} {\Bold\large \subsectitle\\[-0.35cm]}}

	\newcounter{secnumber}[chapter]
	
	\newcommand{\secnew}[1]{\stepcounter{secnumber}\renewcommand{\sectitle}{#1} \section{#1} { \raisebox{0pt}{\begin{titulo-subseccion-blanco} \Bold\large\sectitle\vphantom{p}\end{titulo-subseccion-blanco}}\\[-0.35cm]}}



% Formato de números de sección, subsección, etc...
	
	\newcommand{\secnumbering}{{\large\Bold \thechapter.\thesecnumber}}



%Macros de relleno de contenido

\newcommand{\maco}[4]{
\begin{landscape}
$\ $

\hojarotada{
\begin{center}
$\ $\\[#1]


\begin{minipage}{32\cuadri}
\noindent{\Bold\color{color1!80!black} Cuadro \theCuadro $\,-$   #2}\\ #3\\[-3mm]
\end{minipage}

\includegraphics{#4}
\end{center}
}\stepcounter{Cuadro}
\end{landscape}}


\newcommand{\INEchapter}[4]{\cleardoublepage\addtocontents{toc}{\protect\addvspace{0.6\baselineskip}\color{color1!80!black}}\chapter[\texorpdfstring{\color{color1!80!black}#1}{#1}]{#2}\renewcommand{\chaptitle}{#1}\thispagestyle{empty}\addtocontents{toc}{\protect\addvspace{0.2\baselineskip}{\color{color1!10!white}\hrule height 0.9pt} \addvspace{0.6\baselineskip} \color{black}} \stepcounter{secnumber} $\ $\\[-1cm]
	 \begin{titulo-capitulo}
	 	\begin{numero-capitulo}\centering
			{\fontsize{24mm}{1em}\selectfont\color{white} \Bold \thechapter}
		\end{numero-capitulo}\quad 
		\raisebox{2.1\cuadri}{\begin{tabular}{l}
		\fontsize{8.5mm}{1em}\selectfont \Bold \color{color1!95!black} #2\\[1mm] 
		\fontsize{8.5mm}{1em}\selectfont \Bold \color{color1!95!black} #3 \vphantom{Í} 
		\end{tabular}}
	\end{titulo-capitulo}
		$\ $\\[1.5cm]
		\begin{flushright}
			\begin{capitulo-descripcion}
				\large #4
			\end{capitulo-descripcion}
		\end{flushright}
		\cleardoublepage
		}
		

\newcommand{\INEchapterlong}[4]{\cleardoublepage\addtocontents{toc}{\protect\addvspace{0.6\baselineskip}\color{color1!80!black}}\chapter[\texorpdfstring{\color{color1!80!black}#1}{#1}]{#2}\renewcommand{\chaptitle}{#1}\thispagestyle{empty}\addtocontents{toc}{\protect\addvspace{0.3\baselineskip}{\color{color1!10!white}\hrule height 0.9pt} \addvspace{0.6\baselineskip} \color{black}} \stepcounter{secnumber} $\ $\\[-1cm]
	 \begin{titulo-capitulo-long}
	 	\begin{numero-capitulo-long}\centering
			{\fontsize{24mm}{1em}\selectfont\color{white} \Bold \thechapter}
		\end{numero-capitulo-long}\quad 
		\raisebox{2.1\cuadri}{\begin{tabular}{l}
		\fontsize{9.5mm}{1em}\selectfont \Bold \color{color1!95!black} #2\\ 
		\fontsize{9.5mm}{1em}\selectfont \Bold \color{color1!95!black} #3 \vphantom{Í} 
		\end{tabular}}
	\end{titulo-capitulo-long}
		$\ $\\[1.5cm]
		\begin{flushright}
			\begin{capitulo-descripcion}
				\large #4
			\end{capitulo-descripcion}
		\end{flushright}
		\cleardoublepage
		}
		
		
\newcommand{\appchapter}[4]{\cleardoublepage\addtocontents{toc}{\protect\addvspace{-0.1\baselineskip}\color{color2!80!black}}\chapter[\texorpdfstring{\color{color2!80!black}#1}{#1}]{#2}\renewcommand{\chaptitle}{#1}\thispagestyle{empty} \stepcounter{secnumber}\stepcounter{Cuadro} $\ $\\[1cm]
	 \begin{titulo-capitulo-app}
	 	\begin{numero-capitulo-app}\centering
			{\fontsize{24mm}{1em}\selectfont\color{white} \Bold \thechapter}
		\end{numero-capitulo-app}\quad 
		\raisebox{2.1\cuadri}{\begin{tabular}{l}
		\fontsize{8.5mm}{1em}\selectfont \Bold \color{color2!95!black} #2\\ 
		\fontsize{8.5mm}{1em}\selectfont \Bold \color{color2!95!black} #3 \vphantom{Í} 
		\end{tabular}}
	\end{titulo-capitulo-app}
		$\ $\\[1.5cm]

		\cleardoublepage
		}



\newcommand{\hoja}[1]{\noindent
\begin{fondo}
 #1 
\end{fondo}\clearpage}



\newcommand{\hojados}[2]{\noindent
\begin{fondo}
\begin{tabular}{>{\centering\arraybackslash}p{3.19in}>{\centering\arraybackslash}p{0.01in}>{\centering\arraybackslash}p{3.19in}}

\begin{columnatipoA}
 #1
\end{columnatipoA}  & & 
\begin{columnatipoA}
 #2
\end{columnatipoA}
 \\ 
 
\end{tabular} 


\end{fondo}\clearpage}




\newcommand{\columna}[8]{
$\hspace{-2.1mm}$\begin{tabular}{b{0.17in}p{2.85in}}
\color{white}\section{#1}&\\[-5.1mm] {\color{color2} \textbf{\thesection}} & \textbf{#1.}\\[-1mm]
\end{tabular}\\[2.5mm]


\small $\quad$#2  

\vspace{3mm}
{\color{color2}
\hrule }
\vspace{-1mm}

\begin{center}
{\footnotesize{\Bold#4}} \\[-1mm]
{\scriptsize\texttwelveudash$\,\,$#5$\,\,$\texttwelveudash}
#6\\[0mm]
\end{center}
$\ $\\[-6mm]
{\scriptsize Fuente: #7}\\[-2mm]

{\color{color2}
\hrule }
\vspace{3mm}

#3 

#8
}





\newcommand{\hojarotada}[1]{\noindent\begin{fondo-landscape} #1 \end{fondo-landscape}\clearpage}



\newcommand{\notita}[1]{\footnotetext{\color{color2}$\hspace{-2.3mm}$\scriptsize#1\\[-1.5mm]}


}

\newcommand{\notitasin}[1]{\color{color2}$\hspace{2.6mm}$\scriptsize#1\\[-1.5mm]


}



\newcommand{\cajita}[7]{\checkoddpage\ifoddpage
\begin{bloque-media}
\begin{titulo}
\begin{centrador}
	\begin{tabular}{p{2.5\cuadri}p{28.8\cuadri}}
	 & \\[-3mm]	
	 & {\begin{titulo-subseccion}\begin{numero-subseccion}\secnumbering \end{numero-subseccion}  \phantomsection{\secnew{#1}}  \end{titulo-subseccion}}  \\[-5mm]
	 &  \\[-1.3pt]
	\end{tabular}
\end{centrador}
\end{titulo}

\begin{tabular}{p{11\cuadri}p{21\cuadri}p{5\cuadri}}
		\begin{descripcion}
			#2
		\end{descripcion} 
	& 
		\begin{columna-central}
			\begin{subtitulo}
				\centering\footnotesize{\Bold #3} \\
				\texttwelveudash$\,\,$#4$\,\,$\texttwelveudash
			\end{subtitulo}
	
			\begin{grafica}\centering
				#5
			\end{grafica}
	
			\begin{fuente}
				\footnotesize #6 
			\end{fuente}
		\end{columna-central}
	& 
		\begin{columna-marginal}
			\begin{vacio1}
			
			\end{vacio1}
		
			\begin{nota}
				#7
			\end{nota}
		
			\begin{vacio2}
			
			\end{vacio2}
		\end{columna-marginal}
	\\
\end{tabular}
\end{bloque-media}
\else
\begin{bloque-media}
\begin{titulo}
\begin{centrador-par}
	\begin{tabular}{p{2.5\cuadri}p{28.8\cuadri}}
	 & \\[-3mm]	
	 & {\begin{titulo-subseccion}\begin{numero-subseccion}\centering\secnumbering\end{numero-subseccion}  \phantomsection{\secnew{#1}}  \end{titulo-subseccion}}  \\[-5mm]
	 &  \\[-1.3pt]
	\end{tabular}
\end{centrador-par}
\end{titulo}

\begin{tabular}{p{5\cuadri}p{21.36\cuadri}p{11\cuadri}}
		\begin{columna-marginal}
			\begin{vacio1}
			
			\end{vacio1}
		
			\begin{nota}
				#7
			\end{nota}
		
			\begin{vacio2}
			
			\end{vacio2}
		\end{columna-marginal}		
	& 
		\begin{columna-central}
			\begin{subtitulo}
				\centering\footnotesize{\Bold #3} \\
				\texttwelveudash$\,\,$#4$\,\,$\texttwelveudash
			\end{subtitulo}
	
			\begin{grafica}\centering
				#5
			\end{grafica}
	
			\begin{fuente}
				\footnotesize #6 
			\end{fuente}
		\end{columna-central}
	& 
		\begin{descripcion}
			#2
		\end{descripcion} 
	\\
\end{tabular}
\end{bloque-media}
\fi
}







\begin{document}

%\includepdf{portadaFALTAS_T.pdf}

$\ $
\vspace{4.5cm}

\noindent\begin{tabular}{p{0.1cm}p{6.8cm}}
	& 2014.$\,$ Guatemala, América Central \\
	&\Bold Instituto Nacional de Estadística\\[-0.4cm]
	&\color{blue!50!black}\url{www.ine.gob.gt}\\[0.9cm]
\end{tabular}\\
\noindent\begin{tabular}{p{0.1cm}p{6.8cm}}
	& Está permitida la reproducción parcial o total de los contenidos de este documento con la mención de la fuente. \\[0.5cm]
	
	& Este documento fue elaborado empleando  {\Sans R}, Inkscape, Libre Office y {\Logos \XeLaTeX}.\\
\end{tabular} 


\clearpage




$\ $
\vspace{3.5cm}

\begin{center}
	\Bold \LARGE REPÚBLICA DE GUATEMALA\\
	COMERCIO EXTERIOR I-2014\
\end{center}
\cleardoublepage

\hoja{
	$\ $
	\vspace{0.3cm}
	
	\begin{center}
		{\Bold \LARGE AUTORIDADES}\\[0.7cm]
		
		
		{\Bold \large \color{color1!89!black} JUNTA  DIRECTIVA} \\[0.5cm]
		
		\begin{center}
			\begin{tabular}{x{6.0cm}x{6.0cm}}
				{\Bold Ministerio de Economía}   & 		{\Bold Ministerio de Finanzas}\\
				Lic. Sergio de la Torre, Titular & Lic. Dorval Carías, Titular \\
				Lic. Jacobo Rey Sigfrido Lee, Suplente & Lic. Edwin Oswaldo Martínez, Suplente \\
				& \\
				{\Bold Ministerio de Agricultura,} & {\Bold Ministerio de Energía y Minas}\\ 
				{\Bold Ganadería y Alimentación} & Lic. Erick Archila, Titular \\
				Ing. Elmer López, Titular & Licda. Ivanova Ancheta, Suplente \\
				Ing. Carlos Alfonso Anzueto, Suplente & \\
				
				& {\Bold Banco de Guatemala} \\
				{\Bold Secretaría de Planificación y} & Lic. Julio Roberto Suárez, Titular\\
				{\Bold Programación de la Presidencia} & Lic. Sergio Francisco Recinos Rivera, Suplente\\
				Licda. Ekaterina Arbolievna Parrilla, Titular & \\
				
				Licda. Dora Marina Coc, Suplente & {\Bold Universidad de San Carlos de Guatemala} \\
				
				& Ing. Murphy Olimpo Paiz, Titular \\
				{\Bold Comité Coordinador de } & Lic. Oscar René Paniagua Carrera, Suplente \\  
				{\Bold Asociaciones  Agrícolas, Comerciales, } & \\
				{\Bold Industriales y Financieras} & {\Bold Universidades Privadas} \\
				Lic. Juan Raúl Aguilar , Titular & Dr. Oscar Guillermo Peláez, Titular \\
				Lic. Oscar Augusto Sequeira, Suplente & Lic. Ariel Rivera Irías, Suplente \\[0.3cm]
			\end{tabular} 	
		\end{center}	
		
		
		
		
		
		{\Bold \large \color{color1!89!black} GERENCIA}\\[0.2cm]
		Lic. Rubén Narciso, Gerente\\
		Lic. Jaime Mejía Salguero, Subgerente Técnico\\
		Ing. Orlando Monzón, Subgerente Administrativo Financiero\\
		
		
	\end{center}
}{}
\clearpage

$\ $
\vspace{0.5cm}

\begin{center}
	{\Bold \LARGE EQUIPO RESPONSABLE}\\[1.5cm]
	
	{\Bold \large \color{color1!89!black} REVISIÓN GENERAL}\\[0.2cm]
	Rubén Narciso\\[0.8cm]
	
	
	{\Bold \large \color{color1!89!black} EQUIPO TÉCNICO}\\[0.2cm]
	Jaime Roberto Mejía Salguero\\
	LLLLLLLLLLLLLL\\
	LLLLLLLLLLLLLLLLL\\[0.8cm]
	
	{\Bold \large \color{color1!89!black} DIAGRAMACIÓN Y DISEÑO}\\[0.2cm]
	Hugo Allan García Monterrosa\\
	Fabiola Beatriz Ramírez Pinto\\
	José Carlos Bonilla Aldana\\[0.8cm]
	
	
	
\end{center}\setcounter{page}{0}\cleardoublepage



$\ $\\[0.7cm]

\tableofcontents

\cleardoublepage
\pagestyle{estandar}
\setcounter{page}{1}
\setlength{\arrayrulewidth}{1.0pt}


\cleardoublepage





$\ $\\[0.5cm]
\thispagestyle{empty}
\noindent {\Bold \LARGE Presentación}




$\ $\\



La presentación....

\thispagestyle{empty}


\cleardoublepage

%---------------------------------presentación
%quité la presentación, para fines de trabajar un documento compacto



%--------------------------CONTENIDOS---------------------------





\INEchapter{Comercio Total}{Comercio Total}{}{\quad ¿Qué es comercio total?}

\hojados{\columna{Exportaciones totales}{Las exportaciones del primer trimestre 2014  revelan un crecimiento de 4\% comparado con el primer trimestre 2013.  Según la serie histórica el primer trimestre del 2013 tuvo una disminución de 1\% comparado al 2012 y este a su vez manifiesta tambien una diferencia de -3\%  al 2011.  No así el 2011 manifiesta un crecimiento de 26\% comparado a 2010}{}{Exportaciones Trimestrales Comercio Total    2011-2014}{(cifras preliminares en US\$)}{\ \\[6mm]\begin{tikzpicture}[x=1pt,y=1pt,scale=1]  % Created by tikzDevice version 0.7.0 on 2014-11-24 15:11:13
% !TEX encoding = UTF-8 Unicode
\definecolor[named]{fillColor}{rgb}{1.00,1.00,1.00}
\path[use as bounding box,fill=fillColor,fill opacity=0.00] (0,0) rectangle (280.41,195.13);
\begin{scope}
\path[clip] (  0.00,  0.00) rectangle (280.41,195.13);
\definecolor[named]{drawColor}{rgb}{1.00,1.00,1.00}

\path[draw=drawColor,line width= 0.6pt,line join=round,line cap=round] (  0.00,  0.00) rectangle (280.41,195.13);
\end{scope}
\begin{scope}
\path[clip] (  0.00,  0.00) rectangle (280.41,195.13);

\path[] (  3.13, 14.00) rectangle (271.87,188.02);

\path[] (  3.13, 43.22) --
	(271.87, 43.22);

\path[] (  3.13, 89.03) --
	(271.87, 89.03);

\path[] (  3.13,134.83) --
	(271.87,134.83);

\path[] (  3.13,180.63) --
	(271.87,180.63);

\path[] ( 45.89, 14.00) --
	( 45.89,188.02);

\path[] (106.96, 14.00) --
	(106.96,188.02);

\path[] (168.04, 14.00) --
	(168.04,188.02);

\path[] (229.12, 14.00) --
	(229.12,188.02);

\path[] (  3.13, 20.32) --
	(271.87, 20.32);

\path[] (  3.13, 66.12) --
	(271.87, 66.12);

\path[] (  3.13,111.93) --
	(271.87,111.93);

\path[] (  3.13,157.73) --
	(271.87,157.73);

\path[] ( 15.35, 14.00) --
	( 15.35,188.02);

\path[] ( 76.43, 14.00) --
	( 76.43,188.02);

\path[] (137.50, 14.00) --
	(137.50,188.02);

\path[] (198.58, 14.00) --
	(198.58,188.02);

\path[] (259.66, 14.00) --
	(259.66,188.02);
\definecolor[named]{drawColor}{rgb}{0.86,0.68,0.43}

\path[draw=drawColor,line width= 1.9pt,line join=round] ( 15.35, 61.42) --
	( 76.43, 58.42) --
	(137.50, 71.68) --
	(198.58, 83.24) --
	(259.66,180.11);
\definecolor[named]{drawColor}{rgb}{0.00,0.00,0.00}

\node[text=drawColor,anchor=base,inner sep=0pt, outer sep=0pt, scale=  0.76] at ( 15.35, 64.54) {31,589.0};

\node[text=drawColor,anchor=base,inner sep=0pt, outer sep=0pt, scale=  0.76] at ( 76.43, 49.05) {31,327.0};

\node[text=drawColor,anchor=base east,inner sep=0pt, outer sep=0pt, scale=  0.76] at (131.19, 71.68) {32,485.0};

\node[text=drawColor,anchor=base east,inner sep=0pt, outer sep=0pt, scale=  0.76] at (192.27, 83.24) {33,495.0};

\node[text=drawColor,anchor=base,inner sep=0pt, outer sep=0pt, scale=  0.76] at (259.66,183.23) {41,954.0};
\end{scope}
\begin{scope}
\path[clip] (  0.00,  0.00) rectangle (280.41,195.13);

\path[] (  3.13, 14.00) --
	(  3.13,188.02);
\end{scope}
\begin{scope}
\path[clip] (  0.00,  0.00) rectangle (280.41,195.13);

\path[] (  0.00, 20.32) --
	(  3.13, 20.32);

\path[] (  0.00, 66.12) --
	(  3.13, 66.12);

\path[] (  0.00,111.93) --
	(  3.13,111.93);

\path[] (  0.00,157.73) --
	(  3.13,157.73);
\end{scope}
\begin{scope}
\path[clip] (  0.00,  0.00) rectangle (280.41,195.13);
\definecolor[named]{drawColor}{rgb}{0.60,0.60,0.60}

\path[draw=drawColor,line width= 0.6pt,line join=round] (  3.13, 14.00) --
	(271.87, 14.00);
\end{scope}
\begin{scope}
\path[clip] (  0.00,  0.00) rectangle (280.41,195.13);

\path[] ( 15.35,  9.73) --
	( 15.35, 14.00);

\path[] ( 76.43,  9.73) --
	( 76.43, 14.00);

\path[] (137.50,  9.73) --
	(137.50, 14.00);

\path[] (198.58,  9.73) --
	(198.58, 14.00);

\path[] (259.66,  9.73) --
	(259.66, 14.00);
\end{scope}
\begin{scope}
\path[clip] (  0.00,  0.00) rectangle (280.41,195.13);
\definecolor[named]{drawColor}{rgb}{0.00,0.00,0.00}

\node[text=drawColor,anchor=base west,inner sep=0pt, outer sep=0pt, scale=  0.83] at ( 15.35, -0.00) {2009};

\node[text=drawColor,anchor=base west,inner sep=0pt, outer sep=0pt, scale=  0.83] at ( 76.43, -0.00) {2010};

\node[text=drawColor,anchor=base west,inner sep=0pt, outer sep=0pt, scale=  0.83] at (137.50, -0.00) {2011};

\node[text=drawColor,anchor=base west,inner sep=0pt, outer sep=0pt, scale=  0.83] at (198.58, -0.00) {2012};

\node[text=drawColor,anchor=base west,inner sep=0pt, outer sep=0pt, scale=  0.83] at (259.66, -0.00) {2013};
\end{scope}
  \end{tikzpicture}}{INE, con datos del BANGUAT.}{\notitasin{Los datos del año 2014 se presentan como preliminares y serán ajustados por el registro tardío de los mismos.}}}
{\columna{Variación porcentual de las exportaciones, mismo trimestre año anterior}{La gráfica muestra la variación porcentual en las exportaciones de la serie histórica  2010 a 2014.  se observa que a partir del primer trimestre 2012 una baja en las exportaciones de Guatemala con el resto del mundo, teniendo una leve recuperación el el  segundo y cuarto trimestre 2013, seguido del primer trimestre 2014.}{}{Exportaciones Comercio Total (Variación porcentual) 2010-2014}{( mismo trimestre años anteriores)}{\ \\[6mm]\begin{tikzpicture}[x=1pt,y=1pt,scale=1]  % Created by tikzDevice version 0.7.0 on 2014-12-03 22:07:04
% !TEX encoding = UTF-8 Unicode
\definecolor[named]{fillColor}{rgb}{1.00,1.00,1.00}
\path[use as bounding box,fill=fillColor,fill opacity=0.00] (0,0) rectangle (280.41,195.13);
\begin{scope}
\path[clip] (  0.00,  0.00) rectangle (280.41,195.13);
\definecolor[named]{drawColor}{rgb}{1.00,1.00,1.00}

\path[draw=drawColor,line width= 0.6pt,line join=round,line cap=round] (  0.00,  0.00) rectangle (280.41,195.13);
\end{scope}
\begin{scope}
\path[clip] (  0.00,  0.00) rectangle (280.41,195.13);

\path[] (  7.00, 65.36) rectangle (280.41,180.90);

\path[] ( 14.39, 65.36) --
	( 14.39,180.90);

\path[] ( 26.71, 65.36) --
	( 26.71,180.90);

\path[] ( 39.02, 65.36) --
	( 39.02,180.90);

\path[] ( 51.34, 65.36) --
	( 51.34,180.90);

\path[] ( 63.65, 65.36) --
	( 63.65,180.90);

\path[] ( 75.97, 65.36) --
	( 75.97,180.90);

\path[] ( 88.28, 65.36) --
	( 88.28,180.90);

\path[] (100.60, 65.36) --
	(100.60,180.90);

\path[] (112.92, 65.36) --
	(112.92,180.90);

\path[] (125.23, 65.36) --
	(125.23,180.90);

\path[] (137.55, 65.36) --
	(137.55,180.90);

\path[] (149.86, 65.36) --
	(149.86,180.90);

\path[] (162.18, 65.36) --
	(162.18,180.90);

\path[] (174.49, 65.36) --
	(174.49,180.90);

\path[] (186.81, 65.36) --
	(186.81,180.90);

\path[] (199.12, 65.36) --
	(199.12,180.90);

\path[] (211.44, 65.36) --
	(211.44,180.90);

\path[] (223.76, 65.36) --
	(223.76,180.90);

\path[] (236.07, 65.36) --
	(236.07,180.90);

\path[] (248.39, 65.36) --
	(248.39,180.90);

\path[] (260.70, 65.36) --
	(260.70,180.90);

\path[] (273.02, 65.36) --
	(273.02,180.90);
\definecolor[named]{fillColor}{rgb}{0.86,0.68,0.43}

\path[fill=fillColor] ( 10.70, 65.36) rectangle ( 18.09,180.90);

\path[fill=fillColor] ( 23.01, 65.36) rectangle ( 30.40, 91.74);

\path[fill=fillColor] ( 35.33, 65.36) rectangle ( 42.72, 78.97);

\path[fill=fillColor] ( 47.64, 65.36) rectangle ( 55.03, 74.80);

\path[fill=fillColor] ( 59.96, 65.36) rectangle ( 67.35, 74.80);

\path[fill=fillColor] ( 72.27, 65.36) rectangle ( 79.66, 74.80);

\path[fill=fillColor] ( 84.59, 65.36) rectangle ( 91.98, 74.80);

\path[fill=fillColor] ( 96.90, 65.36) rectangle (104.29, 74.52);

\path[fill=fillColor] (109.22, 65.36) rectangle (116.61, 74.52);

\path[fill=fillColor] (121.54, 65.36) rectangle (128.93, 73.14);

\path[fill=fillColor] (133.85, 65.36) rectangle (141.24, 73.14);

\path[fill=fillColor] (146.17, 65.36) rectangle (153.56, 72.86);

\path[fill=fillColor] (158.48, 65.36) rectangle (165.87, 72.30);

\path[fill=fillColor] (170.80, 65.36) rectangle (178.19, 72.02);

\path[fill=fillColor] (183.11, 65.36) rectangle (190.50, 71.19);

\path[fill=fillColor] (195.43, 65.36) rectangle (202.82, 69.80);

\path[fill=fillColor] (207.75, 65.36) rectangle (215.13, 69.52);

\path[fill=fillColor] (220.06, 65.36) rectangle (227.45, 69.25);

\path[fill=fillColor] (232.38, 65.36) rectangle (239.77, 68.14);

\path[fill=fillColor] (244.69, 65.36) rectangle (252.08, 68.14);

\path[fill=fillColor] (257.01, 65.36) rectangle (264.40, 67.86);

\path[fill=fillColor] (269.32, 65.36) rectangle (276.71, 67.86);
\definecolor[named]{drawColor}{rgb}{0.00,0.00,0.00}

\node[text=drawColor,anchor=base,inner sep=0pt, outer sep=0pt, scale=  0.71] at ( 14.39,183.83) {41.6};

\node[text=drawColor,anchor=base,inner sep=0pt, outer sep=0pt, scale=  0.71] at ( 26.71, 94.67) {9.5};

\node[text=drawColor,anchor=base,inner sep=0pt, outer sep=0pt, scale=  0.71] at ( 39.02, 81.90) {4.9};

\node[text=drawColor,anchor=base,inner sep=0pt, outer sep=0pt, scale=  0.71] at ( 51.34, 77.73) {3.4};

\node[text=drawColor,anchor=base,inner sep=0pt, outer sep=0pt, scale=  0.71] at ( 63.65, 77.73) {3.4};

\node[text=drawColor,anchor=base,inner sep=0pt, outer sep=0pt, scale=  0.71] at ( 75.97, 77.73) {3.4};

\node[text=drawColor,anchor=base,inner sep=0pt, outer sep=0pt, scale=  0.71] at ( 88.28, 77.73) {3.4};

\node[text=drawColor,anchor=base,inner sep=0pt, outer sep=0pt, scale=  0.71] at (100.60, 77.45) {3.3};

\node[text=drawColor,anchor=base,inner sep=0pt, outer sep=0pt, scale=  0.71] at (112.92, 77.45) {3.3};

\node[text=drawColor,anchor=base,inner sep=0pt, outer sep=0pt, scale=  0.71] at (125.23, 76.06) {2.8};

\node[text=drawColor,anchor=base,inner sep=0pt, outer sep=0pt, scale=  0.71] at (137.55, 76.06) {2.8};

\node[text=drawColor,anchor=base,inner sep=0pt, outer sep=0pt, scale=  0.71] at (149.86, 75.79) {2.7};

\node[text=drawColor,anchor=base,inner sep=0pt, outer sep=0pt, scale=  0.71] at (162.18, 75.23) {2.5};

\node[text=drawColor,anchor=base,inner sep=0pt, outer sep=0pt, scale=  0.71] at (174.49, 74.95) {2.4};

\node[text=drawColor,anchor=base,inner sep=0pt, outer sep=0pt, scale=  0.71] at (186.81, 74.12) {2.1};

\node[text=drawColor,anchor=base,inner sep=0pt, outer sep=0pt, scale=  0.71] at (199.12, 72.73) {1.6};

\node[text=drawColor,anchor=base,inner sep=0pt, outer sep=0pt, scale=  0.71] at (211.44, 72.45) {1.5};

\node[text=drawColor,anchor=base,inner sep=0pt, outer sep=0pt, scale=  0.71] at (223.76, 72.18) {1.4};

\node[text=drawColor,anchor=base,inner sep=0pt, outer sep=0pt, scale=  0.71] at (236.07, 71.06) {1.0};

\node[text=drawColor,anchor=base,inner sep=0pt, outer sep=0pt, scale=  0.71] at (248.39, 71.06) {1.0};

\node[text=drawColor,anchor=base,inner sep=0pt, outer sep=0pt, scale=  0.71] at (260.70, 70.79) {0.9};

\node[text=drawColor,anchor=base,inner sep=0pt, outer sep=0pt, scale=  0.71] at (273.02, 70.79) {0.9};
\end{scope}
\begin{scope}
\path[clip] (  0.00,  0.00) rectangle (280.41,195.13);

\path[] (  7.00, 65.36) --
	(  7.00,180.90);
\end{scope}
\begin{scope}
\path[clip] (  0.00,  0.00) rectangle (280.41,195.13);
\definecolor[named]{drawColor}{rgb}{0.60,0.60,0.60}

\path[draw=drawColor,line width= 0.6pt,line join=round] (  7.00, 65.36) --
	(280.41, 65.36);
\end{scope}
\begin{scope}
\path[clip] (  0.00,  0.00) rectangle (280.41,195.13);

\path[] ( 14.39, 61.09) --
	( 14.39, 65.36);

\path[] ( 26.71, 61.09) --
	( 26.71, 65.36);

\path[] ( 39.02, 61.09) --
	( 39.02, 65.36);

\path[] ( 51.34, 61.09) --
	( 51.34, 65.36);

\path[] ( 63.65, 61.09) --
	( 63.65, 65.36);

\path[] ( 75.97, 61.09) --
	( 75.97, 65.36);

\path[] ( 88.28, 61.09) --
	( 88.28, 65.36);

\path[] (100.60, 61.09) --
	(100.60, 65.36);

\path[] (112.92, 61.09) --
	(112.92, 65.36);

\path[] (125.23, 61.09) --
	(125.23, 65.36);

\path[] (137.55, 61.09) --
	(137.55, 65.36);

\path[] (149.86, 61.09) --
	(149.86, 65.36);

\path[] (162.18, 61.09) --
	(162.18, 65.36);

\path[] (174.49, 61.09) --
	(174.49, 65.36);

\path[] (186.81, 61.09) --
	(186.81, 65.36);

\path[] (199.12, 61.09) --
	(199.12, 65.36);

\path[] (211.44, 61.09) --
	(211.44, 65.36);

\path[] (223.76, 61.09) --
	(223.76, 65.36);

\path[] (236.07, 61.09) --
	(236.07, 65.36);

\path[] (248.39, 61.09) --
	(248.39, 65.36);

\path[] (260.70, 61.09) --
	(260.70, 65.36);

\path[] (273.02, 61.09) --
	(273.02, 65.36);
\end{scope}
\begin{scope}
\path[clip] (  0.00,  0.00) rectangle (280.41,195.13);
\definecolor[named]{drawColor}{rgb}{0.00,0.00,0.00}

\node[text=drawColor,rotate= 90.00,anchor=base east,inner sep=0pt, outer sep=0pt, scale=  0.83] at ( 17.83, 58.24) {Guatemala};

\node[text=drawColor,rotate= 90.00,anchor=base east,inner sep=0pt, outer sep=0pt, scale=  0.83] at ( 30.15, 58.24) {Escuintla};

\node[text=drawColor,rotate= 90.00,anchor=base east,inner sep=0pt, outer sep=0pt, scale=  0.83] at ( 42.47, 58.24) {Quetzaltenango};

\node[text=drawColor,rotate= 90.00,anchor=base east,inner sep=0pt, outer sep=0pt, scale=  0.83] at ( 54.78, 58.24) {Huehuetenango};

\node[text=drawColor,rotate= 90.00,anchor=base east,inner sep=0pt, outer sep=0pt, scale=  0.83] at ( 67.10, 58.24) {Suchitepéquez};

\node[text=drawColor,rotate= 90.00,anchor=base east,inner sep=0pt, outer sep=0pt, scale=  0.83] at ( 79.41, 58.24) {Alta Verapaz};

\node[text=drawColor,rotate= 90.00,anchor=base east,inner sep=0pt, outer sep=0pt, scale=  0.83] at ( 91.73, 58.24) {Izabal};

\node[text=drawColor,rotate= 90.00,anchor=base east,inner sep=0pt, outer sep=0pt, scale=  0.83] at (104.04, 58.24) {Chiquimula};

\node[text=drawColor,rotate= 90.00,anchor=base east,inner sep=0pt, outer sep=0pt, scale=  0.83] at (116.36, 58.24) {Petén};

\node[text=drawColor,rotate= 90.00,anchor=base east,inner sep=0pt, outer sep=0pt, scale=  0.83] at (128.67, 58.24) {Chimaltenango};

\node[text=drawColor,rotate= 90.00,anchor=base east,inner sep=0pt, outer sep=0pt, scale=  0.83] at (140.99, 58.24) {Santa Rosa};

\node[text=drawColor,rotate= 90.00,anchor=base east,inner sep=0pt, outer sep=0pt, scale=  0.83] at (153.31, 58.24) {San Marcos};

\node[text=drawColor,rotate= 90.00,anchor=base east,inner sep=0pt, outer sep=0pt, scale=  0.83] at (165.62, 58.24) {Jutiapa};

\node[text=drawColor,rotate= 90.00,anchor=base east,inner sep=0pt, outer sep=0pt, scale=  0.83] at (177.94, 58.24) {Sacatepéquez};

\node[text=drawColor,rotate= 90.00,anchor=base east,inner sep=0pt, outer sep=0pt, scale=  0.83] at (190.25, 58.24) {Zacapa};

\node[text=drawColor,rotate= 90.00,anchor=base east,inner sep=0pt, outer sep=0pt, scale=  0.83] at (202.57, 58.24) {Quiché};

\node[text=drawColor,rotate= 90.00,anchor=base east,inner sep=0pt, outer sep=0pt, scale=  0.83] at (214.88, 58.24) {Retalhuleu};

\node[text=drawColor,rotate= 90.00,anchor=base east,inner sep=0pt, outer sep=0pt, scale=  0.83] at (227.20, 58.24) {Jalapa};

\node[text=drawColor,rotate= 90.00,anchor=base east,inner sep=0pt, outer sep=0pt, scale=  0.83] at (239.51, 58.24) {El Progreso};

\node[text=drawColor,rotate= 90.00,anchor=base east,inner sep=0pt, outer sep=0pt, scale=  0.83] at (251.83, 58.24) {Baja Verapaz};

\node[text=drawColor,rotate= 90.00,anchor=base east,inner sep=0pt, outer sep=0pt, scale=  0.83] at (264.15, 58.24) {Sololá};

\node[text=drawColor,rotate= 90.00,anchor=base east,inner sep=0pt, outer sep=0pt, scale=  0.83] at (276.46, 58.24) {Totonicapán};
\end{scope}
  \end{tikzpicture}}{INE, con datos del BANGUAT.}{\notitasin{Los datos del año 2014 se presentan como preliminares y serán ajustados por el registro tardío de los mismos.}}}
\hojados{\columna{Importaciones totales}{Las importaciones muestran un comportamiento de crecimiento a partir del primer trimestre 2011 con un 26\%,  se observan valores negativos en el segundo y tercer trimestre del 2012,  En el primer trimestre 2013 se advierte nuevamente una recuperación de 1\%   y el 2014 este porcentaje  se eleva a 5\%.}{}{Importaciones Trimestrales Comercio Total    2011-2014}{(cifras preliminares en US\$)}{\ \\[6mm]\begin{tikzpicture}[x=1pt,y=1pt,scale=1]  % Created by tikzDevice version 0.7.0 on 2014-12-22 21:31:46
% !TEX encoding = UTF-8 Unicode
\definecolor[named]{fillColor}{rgb}{1.00,1.00,1.00}
\path[use as bounding box,fill=fillColor,fill opacity=0.00] (0,0) rectangle (230.54,138.04);
\begin{scope}
\path[clip] (  0.00,  0.00) rectangle (230.54,138.04);
\definecolor[named]{drawColor}{rgb}{1.00,1.00,1.00}

\path[draw=drawColor,line width= 0.6pt,line join=round,line cap=round] (  0.00,  0.00) rectangle (230.54,138.04);
\end{scope}
\begin{scope}
\path[clip] (  0.00,  0.00) rectangle (230.54,138.04);

\path[] ( -7.11, 14.25) rectangle (230.54,123.81);

\path[] (  8.39, 14.25) --
	(  8.39,123.81);

\path[] ( 34.22, 14.25) --
	( 34.22,123.81);

\path[] ( 60.05, 14.25) --
	( 60.05,123.81);

\path[] ( 85.88, 14.25) --
	( 85.88,123.81);

\path[] (111.71, 14.25) --
	(111.71,123.81);

\path[] (137.55, 14.25) --
	(137.55,123.81);

\path[] (163.38, 14.25) --
	(163.38,123.81);

\path[] (189.21, 14.25) --
	(189.21,123.81);

\path[] (215.04, 14.25) --
	(215.04,123.81);
\definecolor[named]{drawColor}{rgb}{0.00,0.00,0.00}

\path[draw=drawColor,line width= 0.6pt,line join=round] (  0.64, 14.25) rectangle ( 16.14, 83.83);

\path[draw=drawColor,line width= 0.6pt,line join=round] ( 26.47, 14.25) rectangle ( 41.97,123.81);

\path[draw=drawColor,line width= 0.6pt,line join=round] ( 52.30, 14.25) rectangle ( 67.80, 99.19);

\path[draw=drawColor,line width= 0.6pt,line join=round] ( 78.13, 14.25) rectangle ( 93.63, 72.35);

\path[draw=drawColor,line width= 0.6pt,line join=round] (103.96, 14.25) rectangle (119.46, 47.13);

\path[draw=drawColor,line width= 0.6pt,line join=round] (129.80, 14.25) rectangle (145.30, 25.82);

\path[draw=drawColor,line width= 0.6pt,line join=round] (155.63, 14.25) rectangle (171.13, 15.31);

\path[draw=drawColor,line width= 0.6pt,line join=round] (181.46, 14.25) rectangle (196.96, 14.38);
\definecolor[named]{drawColor}{rgb}{0.78,0.78,0.78}
\definecolor[named]{fillColor}{rgb}{0.78,0.78,0.78}

\path[draw=drawColor,line width= 0.6pt,line join=round,fill=fillColor] (207.29, 14.25) rectangle (222.79, 14.36);
\definecolor[named]{drawColor}{rgb}{0.00,0.00,0.00}

\node[text=drawColor,anchor=base,inner sep=0pt, outer sep=0pt, scale=  0.85] at (  8.39, 86.86) {17,363};

\node[text=drawColor,anchor=base,inner sep=0pt, outer sep=0pt, scale=  0.85] at ( 34.22,126.84) {27,340};

\node[text=drawColor,anchor=base,inner sep=0pt, outer sep=0pt, scale=  0.85] at ( 60.05,102.22) {21,196};

\node[text=drawColor,anchor=base,inner sep=0pt, outer sep=0pt, scale=  0.85] at ( 85.88, 75.39) {14,499};

\node[text=drawColor,anchor=base,inner sep=0pt, outer sep=0pt, scale=  0.85] at (111.71, 50.17) {8,205};

\node[text=drawColor,anchor=base,inner sep=0pt, outer sep=0pt, scale=  0.85] at (137.55, 28.85) {2,886};

\node[text=drawColor,anchor=base,inner sep=0pt, outer sep=0pt, scale=  0.85] at (163.38, 18.35) {265};

\node[text=drawColor,anchor=base,inner sep=0pt, outer sep=0pt, scale=  0.85] at (189.21, 17.42) {33};

\node[text=drawColor,anchor=base,inner sep=0pt, outer sep=0pt, scale=  0.85] at (215.04, 17.40) {27};
\end{scope}
\begin{scope}
\path[clip] (  0.00,  0.00) rectangle (230.54,138.04);
\definecolor[named]{drawColor}{rgb}{0.00,0.00,0.00}

\path[draw=drawColor,line width= 0.6pt,line join=round] (  0.00, 14.25) --
	(230.54, 14.25);
\end{scope}
\begin{scope}
\path[clip] (  0.00,  0.00) rectangle (230.54,138.04);

\path[] (  8.39,  9.98) --
	(  8.39, 14.25);

\path[] ( 34.22,  9.98) --
	( 34.22, 14.25);

\path[] ( 60.05,  9.98) --
	( 60.05, 14.25);

\path[] ( 85.88,  9.98) --
	( 85.88, 14.25);

\path[] (111.71,  9.98) --
	(111.71, 14.25);

\path[] (137.55,  9.98) --
	(137.55, 14.25);

\path[] (163.38,  9.98) --
	(163.38, 14.25);

\path[] (189.21,  9.98) --
	(189.21, 14.25);

\path[] (215.04,  9.98) --
	(215.04, 14.25);
\end{scope}
\begin{scope}
\path[clip] (  0.00,  0.00) rectangle (230.54,138.04);
\definecolor[named]{drawColor}{rgb}{0.00,0.00,0.00}

\node[text=drawColor,anchor=base,inner sep=0pt, outer sep=0pt, scale=  1.00] at (  8.39, -0.00) {15 a 19};

\node[text=drawColor,anchor=base,inner sep=0pt, outer sep=0pt, scale=  1.00] at ( 34.22, -0.00) {20 a 24};

\node[text=drawColor,anchor=base,inner sep=0pt, outer sep=0pt, scale=  1.00] at ( 60.05, -0.00) {25 a 29};

\node[text=drawColor,anchor=base,inner sep=0pt, outer sep=0pt, scale=  1.00] at ( 85.88, -0.00) {30 a 34};

\node[text=drawColor,anchor=base,inner sep=0pt, outer sep=0pt, scale=  1.00] at (111.71, -0.00) {35 a 39};

\node[text=drawColor,anchor=base,inner sep=0pt, outer sep=0pt, scale=  1.00] at (137.55, -0.00) {40 a 44};

\node[text=drawColor,anchor=base,inner sep=0pt, outer sep=0pt, scale=  1.00] at (163.38, -0.00) {45 a 49};

\node[text=drawColor,anchor=base,inner sep=0pt, outer sep=0pt, scale=  1.00] at (189.21, -0.00) {50 ó más};

\node[text=drawColor,anchor=base,inner sep=0pt, outer sep=0pt, scale=  1.00] at (215.04, -0.00) {Ignorado};
\end{scope}
  \end{tikzpicture}}{INE, con datos del BANGUAT.}{\notitasin{Los datos del año 2014 se presentan como preliminares y serán ajustados por el registro tardío de los mismos.}}}
{\columna{Variación porcentual de las importaciones, mismo trimestre año anterior}{La gráfica muestra la variación porcentual en las Importaciones de la serie histórica  2011 a 2014.  Se observa que a partir del primer trimestre 2012 una baja en las Importaciones a Guatemala del resto del mundo, la serie se recupera nuevamente a partir del tercer trimestre del 2012,  trimestre 2013,  mostrando trimestres con intervalos de baja y  crecimiento intercalados, primer trimestre  2014 reporta un porcentaje de 5\%.}{}{Importaciones Comercio Total (Variación porcentual) 2011-2014}{( mismo trimestre años anteriores)}{\ \\[6mm]\begin{tikzpicture}[x=1pt,y=1pt,scale=1]  % Created by tikzDevice version 0.7.0 on 2014-12-22 21:33:07
% !TEX encoding = UTF-8 Unicode
\definecolor[named]{fillColor}{rgb}{1.00,1.00,1.00}
\path[use as bounding box,fill=fillColor,fill opacity=0.00] (0,0) rectangle (230.54,138.04);
\begin{scope}
\path[clip] (  0.00,  0.00) rectangle (230.54,138.04);
\definecolor[named]{drawColor}{rgb}{1.00,1.00,1.00}

\path[draw=drawColor,line width= 0.6pt,line join=round,line cap=round] (  0.00,  0.00) rectangle (230.54,138.04);
\end{scope}
\begin{scope}
\path[clip] (  0.00,  0.00) rectangle (230.54,138.04);

\path[] ( -7.11, 15.11) rectangle (230.54,123.81);

\path[] (  6.87, 15.11) --
	(  6.87,123.81);

\path[] ( 30.17, 15.11) --
	( 30.17,123.81);

\path[] ( 53.47, 15.11) --
	( 53.47,123.81);

\path[] ( 76.76, 15.11) --
	( 76.76,123.81);

\path[] (100.06, 15.11) --
	(100.06,123.81);

\path[] (123.36, 15.11) --
	(123.36,123.81);

\path[] (146.66, 15.11) --
	(146.66,123.81);

\path[] (169.96, 15.11) --
	(169.96,123.81);

\path[] (193.26, 15.11) --
	(193.26,123.81);

\path[] (216.56, 15.11) --
	(216.56,123.81);
\definecolor[named]{drawColor}{rgb}{0.00,0.00,0.00}

\path[draw=drawColor,line width= 0.6pt,line join=round] ( -0.12, 15.11) rectangle ( 13.86, 15.15);

\path[draw=drawColor,line width= 0.6pt,line join=round] ( 23.18, 15.11) rectangle ( 37.16, 15.15);

\path[draw=drawColor,line width= 0.6pt,line join=round] ( 46.48, 15.11) rectangle ( 60.46, 15.21);

\path[draw=drawColor,line width= 0.6pt,line join=round] ( 69.78, 15.11) rectangle ( 83.75, 16.42);

\path[draw=drawColor,line width= 0.6pt,line join=round] ( 93.07, 15.11) rectangle (107.05, 22.69);

\path[draw=drawColor,line width= 0.6pt,line join=round] (116.37, 15.11) rectangle (130.35, 37.58);

\path[draw=drawColor,line width= 0.6pt,line join=round] (139.67, 15.11) rectangle (153.65, 64.80);

\path[draw=drawColor,line width= 0.6pt,line join=round] (162.97, 15.11) rectangle (176.95, 89.73);

\path[draw=drawColor,line width= 0.6pt,line join=round] (186.27, 15.11) rectangle (200.25,104.86);

\path[draw=drawColor,line width= 0.6pt,line join=round] (209.57, 15.11) rectangle (223.55,123.81);

\node[text=drawColor,anchor=base,inner sep=0pt, outer sep=0pt, scale=  0.85] at (  6.87, 18.19) {2};

\node[text=drawColor,anchor=base,inner sep=0pt, outer sep=0pt, scale=  0.85] at ( 30.17, 18.19) {2};

\node[text=drawColor,anchor=base,inner sep=0pt, outer sep=0pt, scale=  0.85] at ( 53.47, 18.25) {5};

\node[text=drawColor,anchor=base,inner sep=0pt, outer sep=0pt, scale=  0.85] at ( 76.76, 19.46) {66};

\node[text=drawColor,anchor=base,inner sep=0pt, outer sep=0pt, scale=  0.85] at (100.06, 25.72) {381};

\node[text=drawColor,anchor=base,inner sep=0pt, outer sep=0pt, scale=  0.85] at (123.36, 40.61) {1,130};

\node[text=drawColor,anchor=base,inner sep=0pt, outer sep=0pt, scale=  0.85] at (146.66, 67.83) {2,499};

\node[text=drawColor,anchor=base,inner sep=0pt, outer sep=0pt, scale=  0.85] at (169.96, 92.77) {3,753};

\node[text=drawColor,anchor=base,inner sep=0pt, outer sep=0pt, scale=  0.85] at (193.26,107.90) {4,514};

\node[text=drawColor,anchor=base,inner sep=0pt, outer sep=0pt, scale=  0.85] at (216.56,126.84) {5,467};
\end{scope}
\begin{scope}
\path[clip] (  0.00,  0.00) rectangle (230.54,138.04);
\definecolor[named]{drawColor}{rgb}{0.00,0.00,0.00}

\path[draw=drawColor,line width= 0.6pt,line join=round] (  0.00, 15.11) --
	(230.54, 15.11);
\end{scope}
\begin{scope}
\path[clip] (  0.00,  0.00) rectangle (230.54,138.04);

\path[] (  6.87, 10.84) --
	(  6.87, 15.11);

\path[] ( 30.17, 10.84) --
	( 30.17, 15.11);

\path[] ( 53.47, 10.84) --
	( 53.47, 15.11);

\path[] ( 76.76, 10.84) --
	( 76.76, 15.11);

\path[] (100.06, 10.84) --
	(100.06, 15.11);

\path[] (123.36, 10.84) --
	(123.36, 15.11);

\path[] (146.66, 10.84) --
	(146.66, 15.11);

\path[] (169.96, 10.84) --
	(169.96, 15.11);

\path[] (193.26, 10.84) --
	(193.26, 15.11);

\path[] (216.56, 10.84) --
	(216.56, 15.11);
\end{scope}
\begin{scope}
\path[clip] (  0.00,  0.00) rectangle (230.54,138.04);
\definecolor[named]{drawColor}{rgb}{0.00,0.00,0.00}

\node[text=drawColor,rotate= 90.00,anchor=base east,inner sep=0pt, outer sep=0pt, scale=  1.00] at ( 10.44,  8.00) {10};

\node[text=drawColor,rotate= 90.00,anchor=base east,inner sep=0pt, outer sep=0pt, scale=  1.00] at ( 33.74,  8.00) {11};

\node[text=drawColor,rotate= 90.00,anchor=base east,inner sep=0pt, outer sep=0pt, scale=  1.00] at ( 57.03,  8.00) {12};

\node[text=drawColor,rotate= 90.00,anchor=base east,inner sep=0pt, outer sep=0pt, scale=  1.00] at ( 80.33,  8.00) {13};

\node[text=drawColor,rotate= 90.00,anchor=base east,inner sep=0pt, outer sep=0pt, scale=  1.00] at (103.63,  8.00) {14};

\node[text=drawColor,rotate= 90.00,anchor=base east,inner sep=0pt, outer sep=0pt, scale=  1.00] at (126.93,  8.00) {15};

\node[text=drawColor,rotate= 90.00,anchor=base east,inner sep=0pt, outer sep=0pt, scale=  1.00] at (150.23,  8.00) {16};

\node[text=drawColor,rotate= 90.00,anchor=base east,inner sep=0pt, outer sep=0pt, scale=  1.00] at (173.53,  8.00) {17};

\node[text=drawColor,rotate= 90.00,anchor=base east,inner sep=0pt, outer sep=0pt, scale=  1.00] at (196.83,  8.00) {18};

\node[text=drawColor,rotate= 90.00,anchor=base east,inner sep=0pt, outer sep=0pt, scale=  1.00] at (220.13,  8.00) {19};
\end{scope}
  \end{tikzpicture}}{INE, con datos del BANGUAT.}{\notitasin{Los datos del año 2014 se presentan como preliminares y serán ajustados por el registro tardío de los mismos.}}}
\hojados { \columna{Balanza Comercial General}{En el primer trimestre 2013 las exportaciones de Guatemala disminuyeron en 1.2\%  alcanzando US\$ 2,606,486,767,  comparando con el primer trimestre 2014 que crecieron en 3.5\% alcanzando un valor de US\$2,698,912,994.                                      Por su parte las importaciones en el mismo periodo se elevaron den un 51.6\% alcanzando el monto de US\$4,148,760,077  y el 2014  el crecimiento fue de 5.5\% con un monto de US\$4,376,926,898}{}{Balanza Comercial General trimestral  2011-2014}{(cifras preliminares en US\$)}{\ \\[6mm]\begin{tikzpicture}[x=1pt,y=1pt,scale=1]  % Created by tikzDevice version 0.7.0 on 2014-12-12 12:43:26
% !TEX encoding = UTF-8 Unicode
\definecolor[named]{fillColor}{rgb}{1.00,1.00,1.00}
\path[use as bounding box,fill=fillColor,fill opacity=0.00] (0,0) rectangle (230.54,138.04);
\begin{scope}
\path[clip] (  0.00,  0.00) rectangle (230.54,138.04);
\definecolor[named]{drawColor}{rgb}{1.00,1.00,1.00}

\path[draw=drawColor,line width= 0.6pt,line join=round,line cap=round] (  0.00,  0.00) rectangle (230.54,138.04);
\end{scope}
\begin{scope}
\path[clip] (  0.00,  0.00) rectangle (230.54,138.04);

\path[] (  7.00, 28.11) rectangle (230.54,123.81);

\path[] ( 38.94, 28.11) --
	( 38.94,123.81);

\path[] ( 92.16, 28.11) --
	( 92.16,123.81);

\path[] (145.38, 28.11) --
	(145.38,123.81);

\path[] (198.61, 28.11) --
	(198.61,123.81);
\definecolor[named]{drawColor}{rgb}{0.00,0.00,0.00}

\path[draw=drawColor,line width= 0.6pt,line join=round] ( 22.97, 28.11) rectangle ( 54.90,123.81);

\path[draw=drawColor,line width= 0.6pt,line join=round] ( 76.19, 28.11) rectangle (108.13, 99.34);

\path[draw=drawColor,line width= 0.6pt,line join=round] (129.42, 28.11) rectangle (161.35, 28.56);
\definecolor[named]{drawColor}{rgb}{0.78,0.78,0.78}
\definecolor[named]{fillColor}{rgb}{0.78,0.78,0.78}

\path[draw=drawColor,line width= 0.6pt,line join=round,fill=fillColor] (182.64, 28.11) rectangle (214.57, 28.15);
\definecolor[named]{drawColor}{rgb}{0.00,0.00,0.00}

\node[text=drawColor,anchor=base,inner sep=0pt, outer sep=0pt, scale=  0.71] at ( 38.94,126.74) {57.2};

\node[text=drawColor,anchor=base,inner sep=0pt, outer sep=0pt, scale=  0.71] at ( 92.16,102.27) {42.5};

\node[text=drawColor,anchor=base,inner sep=0pt, outer sep=0pt, scale=  0.71] at (145.38, 31.49) {0.3};

\node[text=drawColor,anchor=base,inner sep=0pt, outer sep=0pt, scale=  0.71] at (198.61, 31.08) {0.0};
\end{scope}
\begin{scope}
\path[clip] (  0.00,  0.00) rectangle (230.54,138.04);

\path[] (  7.00, 28.11) --
	(  7.00,123.81);
\end{scope}
\begin{scope}
\path[clip] (  0.00,  0.00) rectangle (230.54,138.04);
\definecolor[named]{drawColor}{rgb}{0.00,0.00,0.00}

\path[draw=drawColor,line width= 0.6pt,line join=round] (  7.00, 28.11) --
	(230.54, 28.11);
\end{scope}
\begin{scope}
\path[clip] (  0.00,  0.00) rectangle (230.54,138.04);

\path[] ( 38.94, 23.85) --
	( 38.94, 28.11);

\path[] ( 92.16, 23.85) --
	( 92.16, 28.11);

\path[] (145.38, 23.85) --
	(145.38, 28.11);

\path[] (198.61, 23.85) --
	(198.61, 28.11);
\end{scope}
\begin{scope}
\path[clip] (  0.00,  0.00) rectangle (230.54,138.04);
\definecolor[named]{drawColor}{rgb}{0.00,0.00,0.00}

\node[text=drawColor,anchor=base,inner sep=0pt, outer sep=0pt, scale=  0.83] at ( 38.94, 14.11) {Soltera};

\node[text=drawColor,anchor=base,inner sep=0pt, outer sep=0pt, scale=  0.83] at ( 92.16, 14.11) {Casada};

\node[text=drawColor,anchor=base,inner sep=0pt, outer sep=0pt, scale=  0.83] at (145.38, 14.11) {Unida};

\node[text=drawColor,anchor=base,inner sep=0pt, outer sep=0pt, scale=  0.83] at (198.61, 14.11) {Ignorado};
\end{scope}
  \end{tikzpicture}}{INE, con datos del BANGUAT.}{\notitasin{Los datos del año 2014 se presentan como preliminares y serán ajustados por el registro tardío de los mismos.}}}
{\columna{Variación}{La gráfica muestra la variación porcentual del saldo de la balanza comercial de  2011 a 2014.  Se observa que el saldo siendo negativo  muestra periodos en los cuales es más bajo,  comparando el primer trimestre 2012 el porcentaje es de -31.1\% al  primer trimestre 2013 el cual es  de -3.5\%  y el 2014  el porcentaje se eleva -8.8\%, siendo menos desfavorable para Guatemala.}{}{Variación porcentual del saldo de la Balanza comercial 2011-2014 }{( mismo trimestre años anteriores)}{\ \\[6mm]\begin{tikzpicture}[x=1pt,y=1pt,scale=1]  % Created by tikzDevice version 0.7.0 on 2014-12-12 12:43:26
% !TEX encoding = UTF-8 Unicode
\definecolor[named]{fillColor}{rgb}{1.00,1.00,1.00}
\path[use as bounding box,fill=fillColor,fill opacity=0.00] (0,0) rectangle (230.54,138.04);
\begin{scope}
\path[clip] (  0.00,  0.00) rectangle (230.54,138.04);
\definecolor[named]{drawColor}{rgb}{1.00,1.00,1.00}

\path[draw=drawColor,line width= 0.6pt,line join=round,line cap=round] (  0.00,  0.00) rectangle (230.54,138.04);
\end{scope}
\begin{scope}
\path[clip] (  0.00,  0.00) rectangle (230.54,138.04);

\path[] (  7.00, 28.11) rectangle (230.54,123.81);

\path[] ( 38.94, 28.11) --
	( 38.94,123.81);

\path[] ( 92.16, 28.11) --
	( 92.16,123.81);

\path[] (145.38, 28.11) --
	(145.38,123.81);

\path[] (198.61, 28.11) --
	(198.61,123.81);
\definecolor[named]{drawColor}{rgb}{0.00,0.00,0.00}

\path[draw=drawColor,line width= 0.6pt,line join=round] ( 22.97, 28.11) rectangle ( 54.90,123.81);

\path[draw=drawColor,line width= 0.6pt,line join=round] ( 76.19, 28.11) rectangle (108.13, 99.34);

\path[draw=drawColor,line width= 0.6pt,line join=round] (129.42, 28.11) rectangle (161.35, 28.56);
\definecolor[named]{drawColor}{rgb}{0.78,0.78,0.78}
\definecolor[named]{fillColor}{rgb}{0.78,0.78,0.78}

\path[draw=drawColor,line width= 0.6pt,line join=round,fill=fillColor] (182.64, 28.11) rectangle (214.57, 28.15);
\definecolor[named]{drawColor}{rgb}{0.00,0.00,0.00}

\node[text=drawColor,anchor=base,inner sep=0pt, outer sep=0pt, scale=  0.71] at ( 38.94,126.74) {57.2};

\node[text=drawColor,anchor=base,inner sep=0pt, outer sep=0pt, scale=  0.71] at ( 92.16,102.27) {42.5};

\node[text=drawColor,anchor=base,inner sep=0pt, outer sep=0pt, scale=  0.71] at (145.38, 31.49) {0.3};

\node[text=drawColor,anchor=base,inner sep=0pt, outer sep=0pt, scale=  0.71] at (198.61, 31.08) {0.0};
\end{scope}
\begin{scope}
\path[clip] (  0.00,  0.00) rectangle (230.54,138.04);

\path[] (  7.00, 28.11) --
	(  7.00,123.81);
\end{scope}
\begin{scope}
\path[clip] (  0.00,  0.00) rectangle (230.54,138.04);
\definecolor[named]{drawColor}{rgb}{0.00,0.00,0.00}

\path[draw=drawColor,line width= 0.6pt,line join=round] (  7.00, 28.11) --
	(230.54, 28.11);
\end{scope}
\begin{scope}
\path[clip] (  0.00,  0.00) rectangle (230.54,138.04);

\path[] ( 38.94, 23.85) --
	( 38.94, 28.11);

\path[] ( 92.16, 23.85) --
	( 92.16, 28.11);

\path[] (145.38, 23.85) --
	(145.38, 28.11);

\path[] (198.61, 23.85) --
	(198.61, 28.11);
\end{scope}
\begin{scope}
\path[clip] (  0.00,  0.00) rectangle (230.54,138.04);
\definecolor[named]{drawColor}{rgb}{0.00,0.00,0.00}

\node[text=drawColor,anchor=base,inner sep=0pt, outer sep=0pt, scale=  0.83] at ( 38.94, 14.11) {Soltera};

\node[text=drawColor,anchor=base,inner sep=0pt, outer sep=0pt, scale=  0.83] at ( 92.16, 14.11) {Casada};

\node[text=drawColor,anchor=base,inner sep=0pt, outer sep=0pt, scale=  0.83] at (145.38, 14.11) {Unida};

\node[text=drawColor,anchor=base,inner sep=0pt, outer sep=0pt, scale=  0.83] at (198.61, 14.11) {Ignorado};
\end{scope}
  \end{tikzpicture}}{INE, con datos del BANGUAT.}{\notitasin{Los datos del año 2014 se presentan como preliminares y serán ajustados por el registro tardío de los mismos.}}}
%\hojados{ \columna{Balanza Comercial por continente}{La gráfica muestra la variación porcentual del saldo de la balanza comercial de  2014.  El continente con un mayor volumen de operaciones es con el de América con  un 79\% en las exportaciones y  67\% en las importaciones, seguido  por el continente  de Asia con 13\% en exportaciones y   24\% en las importaciones y en tercer lugar el continente  con un 6\% en las exportaciones y  9\% en las importaciones.}{}{Balanza Comercial por Continente  primer trimestre 2014}{(cifras preliminares en US\$)}{\ \\[6mm]\begin{tikzpicture}[x=1pt,y=1pt,scale=1]  % Created by tikzDevice version 0.7.0 on 2014-12-11 15:14:01
% !TEX encoding = UTF-8 Unicode
\definecolor[named]{fillColor}{rgb}{1.00,1.00,1.00}
\path[use as bounding box,fill=fillColor,fill opacity=0.00] (0,0) rectangle (230.54,138.04);
\begin{scope}
\path[clip] (  0.00,  0.00) rectangle (230.54,138.04);
\definecolor[named]{drawColor}{rgb}{1.00,1.00,1.00}

\path[draw=drawColor,line width= 0.6pt,line join=round,line cap=round] (  0.00,  0.00) rectangle (230.54,138.04);
\end{scope}
\begin{scope}
\path[clip] (  0.00,  0.00) rectangle (230.54,138.04);

\path[] (  7.00, 28.11) rectangle (230.54,123.81);

\path[] ( 67.97, 28.11) --
	( 67.97,123.81);

\path[] (169.58, 28.11) --
	(169.58,123.81);
\definecolor[named]{drawColor}{rgb}{0.00,0.00,0.00}

\path[draw=drawColor,line width= 0.6pt,line join=round] ( 37.48, 28.11) rectangle ( 98.45,123.81);

\path[draw=drawColor,line width= 0.6pt,line join=round] (139.09, 28.11) rectangle (200.06,120.45);

\node[text=drawColor,anchor=base,inner sep=0pt, outer sep=0pt, scale=  0.71] at ( 67.97,126.74) {50.9};

\node[text=drawColor,anchor=base,inner sep=0pt, outer sep=0pt, scale=  0.71] at (169.58,123.38) {49.1};
\end{scope}
\begin{scope}
\path[clip] (  0.00,  0.00) rectangle (230.54,138.04);

\path[] (  7.00, 28.11) --
	(  7.00,123.81);
\end{scope}
\begin{scope}
\path[clip] (  0.00,  0.00) rectangle (230.54,138.04);
\definecolor[named]{drawColor}{rgb}{0.00,0.00,0.00}

\path[draw=drawColor,line width= 0.6pt,line join=round] (  7.00, 28.11) --
	(230.54, 28.11);
\end{scope}
\begin{scope}
\path[clip] (  0.00,  0.00) rectangle (230.54,138.04);

\path[] ( 67.97, 23.85) --
	( 67.97, 28.11);

\path[] (169.58, 23.85) --
	(169.58, 28.11);
\end{scope}
\begin{scope}
\path[clip] (  0.00,  0.00) rectangle (230.54,138.04);
\definecolor[named]{drawColor}{rgb}{0.00,0.00,0.00}

\node[text=drawColor,anchor=base,inner sep=0pt, outer sep=0pt, scale=  0.83] at ( 67.97, 14.11) {Hombre};

\node[text=drawColor,anchor=base,inner sep=0pt, outer sep=0pt, scale=  0.83] at (169.58, 14.11) {Mujer};
\end{scope}
  \end{tikzpicture}}{INE, con datos del BANGUAT.}{\notitasin{Los datos del año 2014 se presentan como preliminares y serán ajustados por el registro tardío de los mismos.}}}
%{\columna{Balanza Comercial con Centroamérica}{Los países centroaméricanos como principales socios comerciales se  el Salvador con un porcentaje de 38.2\% en las exportaciones y 42.9\% en las importaciones seguido de Honduras con 25.7\% en las Exportaciones y 16\% en las importaciones, en tercer lugar es Costa Rica con 12.5\%  en las exportaciones y 30.6\% en las importaciones,  con los otros países se sostiene un porcentaje más bajo en ambas vías.}{}{    Balanza  Comercial con el Mercado Común  Centroaméricano,}{( Cifras expresadas en US dólares )}{\ \\[6mm]\begin{tikzpicture}[x=1pt,y=1pt,scale=1]  % Created by tikzDevice version 0.7.0 on 2014-12-03 22:07:12
% !TEX encoding = UTF-8 Unicode
\definecolor[named]{fillColor}{rgb}{1.00,1.00,1.00}
\path[use as bounding box,fill=fillColor,fill opacity=0.00] (0,0) rectangle (280.41,195.13);
\begin{scope}
\path[clip] (  0.00,  0.00) rectangle (280.41,195.13);
\definecolor[named]{drawColor}{rgb}{1.00,1.00,1.00}

\path[draw=drawColor,line width= 0.6pt,line join=round,line cap=round] ( -0.00,  0.00) rectangle (280.41,195.13);
\end{scope}
\begin{scope}
\path[clip] (  0.00,  0.00) rectangle (280.41,195.13);

\path[] ( 88.47, -1.53) rectangle (253.09,195.13);

\path[] ( 88.47, 14.85) --
	(253.09, 14.85);

\path[] ( 88.47, 42.17) --
	(253.09, 42.17);

\path[] ( 88.47, 69.48) --
	(253.09, 69.48);

\path[] ( 88.47, 96.80) --
	(253.09, 96.80);

\path[] ( 88.47,124.11) --
	(253.09,124.11);

\path[] ( 88.47,151.43) --
	(253.09,151.43);

\path[] ( 88.47,178.74) --
	(253.09,178.74);
\definecolor[named]{fillColor}{rgb}{0.86,0.68,0.43}

\path[fill=fillColor] ( 88.47,  6.66) rectangle ( 96.40, 23.05);

\path[fill=fillColor] ( 88.47, 33.97) rectangle (114.27, 50.36);

\path[fill=fillColor] ( 88.47, 61.29) rectangle (128.41, 77.68);

\path[fill=fillColor] ( 88.47, 88.60) rectangle (138.32,104.99);

\path[fill=fillColor] ( 88.47,115.92) rectangle (141.46,132.31);

\path[fill=fillColor] ( 88.47,143.23) rectangle (145.52,159.62);

\path[fill=fillColor] ( 88.47,170.55) rectangle (253.09,186.93);
\definecolor[named]{drawColor}{rgb}{0.00,0.00,0.00}

\node[text=drawColor,anchor=base west,inner sep=0pt, outer sep=0pt, scale=  0.71] at (102.64, 11.93) {835};

\node[text=drawColor,anchor=base west,inner sep=0pt, outer sep=0pt, scale=  0.71] at (122.59, 39.24) {2718};

\node[text=drawColor,anchor=base west,inner sep=0pt, outer sep=0pt, scale=  0.71] at (136.73, 66.55) {4208};

\node[text=drawColor,anchor=base west,inner sep=0pt, outer sep=0pt, scale=  0.71] at (146.65, 93.87) {5253};

\node[text=drawColor,anchor=base west,inner sep=0pt, outer sep=0pt, scale=  0.71] at (149.78,121.18) {5583};

\node[text=drawColor,anchor=base west,inner sep=0pt, outer sep=0pt, scale=  0.71] at (153.84,148.50) {6011};

\node[text=drawColor,anchor=base west,inner sep=0pt, outer sep=0pt, scale=  0.71] at (263.50,175.81) {17346};
\end{scope}
\begin{scope}
\path[clip] (  0.00,  0.00) rectangle (280.41,195.13);
\definecolor[named]{drawColor}{rgb}{0.60,0.60,0.60}

\path[draw=drawColor,line width= 0.6pt,line join=round] ( 88.47,  0.00) --
	( 88.47,195.13);
\end{scope}
\begin{scope}
\path[clip] (  0.00,  0.00) rectangle (280.41,195.13);
\definecolor[named]{drawColor}{rgb}{0.00,0.00,0.00}

\node[text=drawColor,anchor=base east,inner sep=0pt, outer sep=0pt, scale=  0.83] at ( 81.36, 11.41) {Contra la sexualidad};

\node[text=drawColor,anchor=base east,inner sep=0pt, outer sep=0pt, scale=  0.83] at ( 81.36, 38.72) {Otras causas};

\node[text=drawColor,anchor=base east,inner sep=0pt, outer sep=0pt, scale=  0.83] at ( 81.36, 66.04) {Contra la libertad};

\node[text=drawColor,anchor=base east,inner sep=0pt, outer sep=0pt, scale=  0.83] at ( 81.36, 93.35) {Homicidios};

\node[text=drawColor,anchor=base east,inner sep=0pt, outer sep=0pt, scale=  0.83] at ( 81.36,120.67) {Extorsión y chantaje};

\node[text=drawColor,anchor=base east,inner sep=0pt, outer sep=0pt, scale=  0.83] at ( 81.36,147.98) {Lesiones};

\node[text=drawColor,anchor=base east,inner sep=0pt, outer sep=0pt, scale=  0.83] at ( 81.36,175.30) {Contra el patrimonio};
\end{scope}
\begin{scope}
\path[clip] (  0.00,  0.00) rectangle (280.41,195.13);

\path[] ( 84.20, 14.85) --
	( 88.47, 14.85);

\path[] ( 84.20, 42.17) --
	( 88.47, 42.17);

\path[] ( 84.20, 69.48) --
	( 88.47, 69.48);

\path[] ( 84.20, 96.80) --
	( 88.47, 96.80);

\path[] ( 84.20,124.11) --
	( 88.47,124.11);

\path[] ( 84.20,151.43) --
	( 88.47,151.43);

\path[] ( 84.20,178.74) --
	( 88.47,178.74);
\end{scope}
  \end{tikzpicture}}{INE, con datos del BANGUAT.}{\notitasin{Los datos del año 2014 se presentan como preliminares y serán ajustados por el registro tardío de los mismos.}}}
\hojados{\columna{Exportaciones 10 principales productos trimestre 2014}{El principal producto de exportación de Guatemala al resto del mundo, constituyó en el primer trimestre 2014, en el azucar con 25.9\%  de porcentaje en esta serie y 292,652,996 US\$, seguido de  café oro con un 16.1\%  182,169,782 y banano con 13.8\% y 155,826,949; seguido de minerales de plata y Minerales de plomo con un 7.6\% y 7.3\% respectivamente.}{}{Exportación de los principales 10 productos, Primer trimestre 2014}{( cifras preliminares expresadas en US\$ )}{\ \\[6mm]\begin{tikzpicture}[x=1pt,y=1pt,scale=1]  % Created by tikzDevice version 0.7.0 on 2014-11-24 15:12:07
% !TEX encoding = UTF-8 Unicode
\definecolor[named]{fillColor}{rgb}{1.00,1.00,1.00}
\path[use as bounding box,fill=fillColor,fill opacity=0.00] (0,0) rectangle (280.41,195.13);
\begin{scope}
\path[clip] (  0.00,  0.00) rectangle (280.41,195.13);
\definecolor[named]{drawColor}{rgb}{1.00,1.00,1.00}

\path[draw=drawColor,line width= 0.6pt,line join=round,line cap=round] (  0.00,  0.00) rectangle (280.41,195.13);
\end{scope}
\begin{scope}
\path[clip] (  0.00,  0.00) rectangle (280.41,195.13);

\path[] ( -1.76, 14.00) rectangle (271.87,188.02);

\path[] (  0.00, 21.72) --
	(271.87, 21.72);

\path[] (  0.00, 67.02) --
	(271.87, 67.02);

\path[] (  0.00,112.33) --
	(271.87,112.33);

\path[] (  0.00,157.63) --
	(271.87,157.63);

\path[] ( 41.77, 14.00) --
	( 41.77,188.02);

\path[] (103.96, 14.00) --
	(103.96,188.02);

\path[] (166.15, 14.00) --
	(166.15,188.02);

\path[] (228.34, 14.00) --
	(228.34,188.02);

\path[] (  0.00, 44.37) --
	(271.87, 44.37);

\path[] (  0.00, 89.68) --
	(271.87, 89.68);

\path[] (  0.00,134.98) --
	(271.87,134.98);

\path[] (  0.00,180.29) --
	(271.87,180.29);

\path[] ( 10.68, 14.00) --
	( 10.68,188.02);

\path[] ( 72.87, 14.00) --
	( 72.87,188.02);

\path[] (135.06, 14.00) --
	(135.06,188.02);

\path[] (197.24, 14.00) --
	(197.24,188.02);

\path[] (259.43, 14.00) --
	(259.43,188.02);
\definecolor[named]{drawColor}{rgb}{0.86,0.68,0.43}

\path[draw=drawColor,line width= 1.9pt,line join=round] ( 10.68,180.11) --
	( 72.87,131.36) --
	(135.06,106.08) --
	(197.24, 58.42) --
	(259.43, 67.30);
\definecolor[named]{drawColor}{rgb}{0.00,0.00,0.00}

\node[text=drawColor,anchor=base,inner sep=0pt, outer sep=0pt, scale=  0.76] at ( 10.68,183.23) {6,498.0};

\node[text=drawColor,anchor=base west,inner sep=0pt, outer sep=0pt, scale=  0.76] at ( 72.87,134.48) {5,960.0};

\node[text=drawColor,anchor=base west,inner sep=0pt, outer sep=0pt, scale=  0.76] at (135.06,109.20) {5,681.0};

\node[text=drawColor,anchor=base,inner sep=0pt, outer sep=0pt, scale=  0.76] at (197.24, 49.05) {5,155.0};

\node[text=drawColor,anchor=base,inner sep=0pt, outer sep=0pt, scale=  0.76] at (259.43, 70.42) {5,253.0};
\end{scope}
\begin{scope}
\path[clip] (  0.00,  0.00) rectangle (280.41,195.13);
\definecolor[named]{drawColor}{rgb}{0.60,0.60,0.60}

\path[draw=drawColor,line width= 0.6pt,line join=round] (  0.00, 14.00) --
	(271.87, 14.00);
\end{scope}
\begin{scope}
\path[clip] (  0.00,  0.00) rectangle (280.41,195.13);

\path[] ( 10.68,  9.73) --
	( 10.68, 14.00);

\path[] ( 72.87,  9.73) --
	( 72.87, 14.00);

\path[] (135.06,  9.73) --
	(135.06, 14.00);

\path[] (197.24,  9.73) --
	(197.24, 14.00);

\path[] (259.43,  9.73) --
	(259.43, 14.00);
\end{scope}
\begin{scope}
\path[clip] (  0.00,  0.00) rectangle (280.41,195.13);
\definecolor[named]{drawColor}{rgb}{0.00,0.00,0.00}

\node[text=drawColor,anchor=base west,inner sep=0pt, outer sep=0pt, scale=  0.83] at ( 10.68, -0.00) {2009};

\node[text=drawColor,anchor=base west,inner sep=0pt, outer sep=0pt, scale=  0.83] at ( 72.87, -0.00) {2010};

\node[text=drawColor,anchor=base west,inner sep=0pt, outer sep=0pt, scale=  0.83] at (135.06, -0.00) {2011};

\node[text=drawColor,anchor=base west,inner sep=0pt, outer sep=0pt, scale=  0.83] at (197.24, -0.00) {2012};

\node[text=drawColor,anchor=base west,inner sep=0pt, outer sep=0pt, scale=  0.83] at (259.43, -0.00) {2013};
\end{scope}
  \end{tikzpicture}}{INE, con datos del BANGUAT.}{\notitasin{Los datos del año 2014 se presentan como preliminares y serán ajustados por el registro tardío de los mismos.}}}
{\columna{Exportaciones 10 principales países trimestre 2014}{El principal socio comercial de Guatemala es Estados unidos a donde se exportó en el primer trimestre 2014,  969,383,445 US\$ con un porcentaje de 35.9\% seguido de El Salvador  a quien se le exportó 318,214,092 con un porcentaje de 11.8\%, en tercer lugar Honduras  a donde se exportó 213,579,857 con un porcentaje de 7.9\%}{}{Exportación de los principales 10 países, Primer trimestre 2014}{(cifras preliminares expresadas en US\$)}{\ \\[6mm]\begin{tikzpicture}[x=1pt,y=1pt,scale=1]  % Created by tikzDevice version 0.7.0 on 2014-12-11 15:14:28
% !TEX encoding = UTF-8 Unicode
\definecolor[named]{fillColor}{rgb}{1.00,1.00,1.00}
\path[use as bounding box,fill=fillColor,fill opacity=0.00] (0,0) rectangle (230.54,138.04);
\begin{scope}
\path[clip] (  0.00,  0.00) rectangle (230.54,138.04);
\definecolor[named]{drawColor}{rgb}{1.00,1.00,1.00}

\path[draw=drawColor,line width= 0.6pt,line join=round,line cap=round] (  0.00,  0.00) rectangle (230.54,138.04);
\end{scope}
\begin{scope}
\path[clip] (  0.00,  0.00) rectangle (230.54,138.04);

\path[] (  7.00, 28.11) rectangle (230.54,123.81);

\path[] ( 32.79, 28.11) --
	( 32.79,123.81);

\path[] ( 75.78, 28.11) --
	( 75.78,123.81);

\path[] (118.77, 28.11) --
	(118.77,123.81);

\path[] (161.76, 28.11) --
	(161.76,123.81);

\path[] (204.75, 28.11) --
	(204.75,123.81);
\definecolor[named]{drawColor}{rgb}{0.00,0.00,0.00}

\path[draw=drawColor,line width= 0.6pt,line join=round] ( 19.90, 28.11) rectangle ( 45.69,123.81);

\path[draw=drawColor,line width= 0.6pt,line join=round] ( 62.89, 28.11) rectangle ( 88.68, 73.17);

\path[draw=drawColor,line width= 0.6pt,line join=round] (105.87, 28.11) rectangle (131.67, 30.59);

\path[draw=drawColor,line width= 0.6pt,line join=round] (148.86, 28.11) rectangle (174.66, 30.10);

\path[draw=drawColor,line width= 0.6pt,line join=round] (191.85, 28.11) rectangle (217.64, 29.43);

\node[text=drawColor,anchor=base,inner sep=0pt, outer sep=0pt, scale=  0.71] at ( 32.79,126.74) {65.3};

\node[text=drawColor,anchor=base,inner sep=0pt, outer sep=0pt, scale=  0.71] at ( 75.78, 76.09) {30.7};

\node[text=drawColor,anchor=base,inner sep=0pt, outer sep=0pt, scale=  0.71] at (118.77, 33.52) {1.7};

\node[text=drawColor,anchor=base,inner sep=0pt, outer sep=0pt, scale=  0.71] at (161.76, 33.03) {1.4};

\node[text=drawColor,anchor=base,inner sep=0pt, outer sep=0pt, scale=  0.71] at (204.75, 32.36) {0.9};
\end{scope}
\begin{scope}
\path[clip] (  0.00,  0.00) rectangle (230.54,138.04);

\path[] (  7.00, 28.11) --
	(  7.00,123.81);
\end{scope}
\begin{scope}
\path[clip] (  0.00,  0.00) rectangle (230.54,138.04);
\definecolor[named]{drawColor}{rgb}{0.00,0.00,0.00}

\path[draw=drawColor,line width= 0.6pt,line join=round] (  7.00, 28.11) --
	(230.54, 28.11);
\end{scope}
\begin{scope}
\path[clip] (  0.00,  0.00) rectangle (230.54,138.04);

\path[] ( 32.79, 23.85) --
	( 32.79, 28.11);

\path[] ( 75.78, 23.85) --
	( 75.78, 28.11);

\path[] (118.77, 23.85) --
	(118.77, 28.11);

\path[] (161.76, 23.85) --
	(161.76, 28.11);

\path[] (204.75, 23.85) --
	(204.75, 28.11);
\end{scope}
\begin{scope}
\path[clip] (  0.00,  0.00) rectangle (230.54,138.04);
\definecolor[named]{drawColor}{rgb}{0.00,0.00,0.00}

\node[text=drawColor,anchor=base,inner sep=0pt, outer sep=0pt, scale=  0.83] at ( 32.79, 14.11) {Médica};

\node[text=drawColor,anchor=base,inner sep=0pt, outer sep=0pt, scale=  0.83] at ( 75.78, 14.11) {Comadrona};

\node[text=drawColor,anchor=base,inner sep=0pt, outer sep=0pt, scale=  0.83] at (118.77, 14.11) {Ninguna};

\node[text=drawColor,anchor=base,inner sep=0pt, outer sep=0pt, scale=  0.83] at (161.76, 14.11) {Empírica};

\node[text=drawColor,anchor=base,inner sep=0pt, outer sep=0pt, scale=  0.83] at (204.75, 14.11) {Paramédica};
\end{scope}
  \end{tikzpicture}}{INE, con datos del BANGUAT.}{\notitasin{Los datos del año 2014 se presentan como preliminares y serán ajustados por el registro tardío de los mismos.}}}
\hojados{\columna{Exportaciones 10 principales secciones}{De las secciones del Sistema Arancelario Centroaméricano (SAC) en el primer trimestre 2014 en las exportaciones de Guatemala, la correspondiente a productos del reino vegetal ocupa el primer lugar con un 25.4\%, el segundo lugar con los productos de las Industrias Alimenticias un 21.6\%,  en tercer lugar los productos Minerales con un 14.5\% seguido de Industrias textiles y Productos y materias plásticas on un 13.9\% y 8.6\% respectivamente}{}{Exportación de las principales secciones del Sistema Arancelario C. A., Primer trimestre 2014}{(Cifras preliminares expresadas en US\$)}{\ \\[6mm]\begin{tikzpicture}[x=1pt,y=1pt,scale=1]  % Created by tikzDevice version 0.7.0 on 2014-11-24 15:12:08
% !TEX encoding = UTF-8 Unicode
\definecolor[named]{fillColor}{rgb}{1.00,1.00,1.00}
\path[use as bounding box,fill=fillColor,fill opacity=0.00] (0,0) rectangle (280.41,195.13);
\begin{scope}
\path[clip] (  0.00,  0.00) rectangle (280.41,195.13);
\definecolor[named]{drawColor}{rgb}{1.00,1.00,1.00}

\path[draw=drawColor,line width= 0.6pt,line join=round,line cap=round] (  0.00,  0.00) rectangle (280.41,195.13);
\end{scope}
\begin{scope}
\path[clip] (  0.00,  0.00) rectangle (280.41,195.13);

\path[] (  7.00, 93.37) rectangle (280.41,180.90);

\path[] ( 38.55, 93.37) --
	( 38.55,180.90);

\path[] ( 91.13, 93.37) --
	( 91.13,180.90);

\path[] (143.70, 93.37) --
	(143.70,180.90);

\path[] (196.28, 93.37) --
	(196.28,180.90);

\path[] (248.86, 93.37) --
	(248.86,180.90);
\definecolor[named]{fillColor}{rgb}{0.86,0.68,0.43}

\path[fill=fillColor] ( 22.77, 93.37) rectangle ( 54.32,180.90);

\path[fill=fillColor] ( 75.35, 93.37) rectangle (106.90,104.27);

\path[fill=fillColor] (127.93, 93.37) rectangle (159.48, 98.44);

\path[fill=fillColor] (180.51, 93.37) rectangle (212.06, 97.14);

\path[fill=fillColor] (233.09, 93.37) rectangle (264.63, 94.01);
\definecolor[named]{drawColor}{rgb}{0.00,0.00,0.00}

\node[text=drawColor,anchor=base,inner sep=0pt, outer sep=0pt, scale=  0.71] at ( 38.55,183.83) {81.1};

\node[text=drawColor,anchor=base,inner sep=0pt, outer sep=0pt, scale=  0.71] at ( 91.13,107.20) {10.1};

\node[text=drawColor,anchor=base,inner sep=0pt, outer sep=0pt, scale=  0.71] at (143.70,101.37) {4.7};

\node[text=drawColor,anchor=base,inner sep=0pt, outer sep=0pt, scale=  0.71] at (196.28,100.07) {3.5};

\node[text=drawColor,anchor=base,inner sep=0pt, outer sep=0pt, scale=  0.71] at (248.86, 96.94) {0.6};
\end{scope}
\begin{scope}
\path[clip] (  0.00,  0.00) rectangle (280.41,195.13);

\path[] (  7.00, 93.37) --
	(  7.00,180.90);
\end{scope}
\begin{scope}
\path[clip] (  0.00,  0.00) rectangle (280.41,195.13);
\definecolor[named]{drawColor}{rgb}{0.60,0.60,0.60}

\path[draw=drawColor,line width= 0.6pt,line join=round] (  7.00, 93.37) --
	(280.41, 93.37);
\end{scope}
\begin{scope}
\path[clip] (  0.00,  0.00) rectangle (280.41,195.13);

\path[] ( 38.55, 89.10) --
	( 38.55, 93.37);

\path[] ( 91.13, 89.10) --
	( 91.13, 93.37);

\path[] (143.70, 89.10) --
	(143.70, 93.37);

\path[] (196.28, 89.10) --
	(196.28, 93.37);

\path[] (248.86, 89.10) --
	(248.86, 93.37);
\end{scope}
\begin{scope}
\path[clip] (  0.00,  0.00) rectangle (280.41,195.13);
\definecolor[named]{drawColor}{rgb}{0.00,0.00,0.00}

\node[text=drawColor,rotate= 90.00,anchor=base east,inner sep=0pt, outer sep=0pt, scale=  0.83] at ( 41.99, 86.25) {por arma de fuego};

\node[text=drawColor,rotate= 90.00,anchor=base east,inner sep=0pt, outer sep=0pt, scale=  0.83] at ( 94.57, 86.25) {por arma blanca};

\node[text=drawColor,rotate= 90.00,anchor=base east,inner sep=0pt, outer sep=0pt, scale=  0.83] at (147.15, 86.25) {por arma contundente};

\node[text=drawColor,rotate= 90.00,anchor=base east,inner sep=0pt, outer sep=0pt, scale=  0.83] at (199.73, 86.25) {por estrangulamiento};

\node[text=drawColor,rotate= 90.00,anchor=base east,inner sep=0pt, outer sep=0pt, scale=  0.83] at (252.30, 86.25) {por linchamiento};
\end{scope}
  \end{tikzpicture}}{INE, con datos del BANGUAT.}{\notitasin{Los datos del año 2014 se presentan como preliminares y serán ajustados por el registro tardío de los mismos.}}}
{\columna{Importaciones 10 principales productos trimestre 2014}{El principal producto de importación para Guatemala del resto del mundo lo constituyó en el primer  trimestre 2014 es el Diesel oil con 34.6\%  de porcentaje en esta serie y US\$423,184,782  seguido de  Gasolina con un 27.5\%  US\$336,131,521 y Medicamentos p/humanos 8.2\%  US\$100,744,903, continuandocon  Gas propano y teléfonos celulares con un 6.9\% y 6.2\% respectivamente}{}{Importación de los principales productos Primer trimestre 2014}{(Cifras preliminares expresadas en US\$)}{\ \\[6mm]\begin{tikzpicture}[x=1pt,y=1pt,scale=1]  % Created by tikzDevice version 0.7.0 on 2014-12-15 22:10:47
% !TEX encoding = UTF-8 Unicode
\definecolor[named]{fillColor}{rgb}{1.00,1.00,1.00}
\path[use as bounding box,fill=fillColor,fill opacity=0.00] (0,0) rectangle (230.54,138.04);
\begin{scope}
\path[clip] (  0.00,  0.00) rectangle (230.54,138.04);
\definecolor[named]{drawColor}{rgb}{1.00,1.00,1.00}

\path[draw=drawColor,line width= 0.6pt,line join=round,line cap=round] (  0.00,  0.00) rectangle (230.54,138.04);
\end{scope}
\begin{scope}
\path[clip] (  0.00,  0.00) rectangle (230.54,138.04);

\path[] ( 89.93, -2.74) rectangle (210.06,138.04);

\path[] ( 89.93,  5.54) --
	(210.06,  5.54);

\path[] ( 89.93, 19.34) --
	(210.06, 19.34);

\path[] ( 89.93, 33.14) --
	(210.06, 33.14);

\path[] ( 89.93, 46.95) --
	(210.06, 46.95);

\path[] ( 89.93, 60.75) --
	(210.06, 60.75);

\path[] ( 89.93, 74.55) --
	(210.06, 74.55);

\path[] ( 89.93, 88.35) --
	(210.06, 88.35);

\path[] ( 89.93,102.15) --
	(210.06,102.15);

\path[] ( 89.93,115.95) --
	(210.06,115.95);

\path[] ( 89.93,129.75) --
	(210.06,129.75);
\definecolor[named]{drawColor}{rgb}{0.00,0.00,0.00}

\path[draw=drawColor,line width= 0.6pt,line join=round] ( 89.93,  1.40) rectangle (100.44,  9.68);

\path[draw=drawColor,line width= 0.6pt,line join=round] ( 89.93, 15.20) rectangle (101.01, 23.48);

\path[draw=drawColor,line width= 0.6pt,line join=round] ( 89.93, 29.00) rectangle (101.01, 37.28);

\path[draw=drawColor,line width= 0.6pt,line join=round] ( 89.93, 42.81) rectangle (102.14, 51.09);

\path[draw=drawColor,line width= 0.6pt,line join=round] ( 89.93, 56.61) rectangle (102.71, 64.89);

\path[draw=drawColor,line width= 0.6pt,line join=round] ( 89.93, 70.41) rectangle (111.23, 78.69);

\path[draw=drawColor,line width= 0.6pt,line join=round] ( 89.93, 84.21) rectangle (114.07, 92.49);

\path[draw=drawColor,line width= 0.6pt,line join=round] ( 89.93, 98.01) rectangle (118.61,106.29);

\path[draw=drawColor,line width= 0.6pt,line join=round] ( 89.93,111.81) rectangle (185.35,120.09);

\path[draw=drawColor,line width= 0.6pt,line join=round] ( 89.93,125.61) rectangle (210.06,133.90);

\node[text=drawColor,anchor=base west,inner sep=0pt, outer sep=0pt, scale=  0.85] at (104.69,  2.61) {37};

\node[text=drawColor,anchor=base west,inner sep=0pt, outer sep=0pt, scale=  0.85] at (105.26, 16.41) {39};

\node[text=drawColor,anchor=base west,inner sep=0pt, outer sep=0pt, scale=  0.85] at (105.26, 30.22) {39};

\node[text=drawColor,anchor=base west,inner sep=0pt, outer sep=0pt, scale=  0.85] at (106.39, 44.02) {43};

\node[text=drawColor,anchor=base west,inner sep=0pt, outer sep=0pt, scale=  0.85] at (106.96, 57.82) {45};

\node[text=drawColor,anchor=base west,inner sep=0pt, outer sep=0pt, scale=  0.85] at (115.48, 71.62) {75};

\node[text=drawColor,anchor=base west,inner sep=0pt, outer sep=0pt, scale=  0.85] at (118.32, 85.42) {85};

\node[text=drawColor,anchor=base west,inner sep=0pt, outer sep=0pt, scale=  0.85] at (124.99, 99.22) {101};

\node[text=drawColor,anchor=base west,inner sep=0pt, outer sep=0pt, scale=  0.85] at (191.73,113.02) {336};

\node[text=drawColor,anchor=base west,inner sep=0pt, outer sep=0pt, scale=  0.85] at (216.43,126.83) {423};
\end{scope}
\begin{scope}
\path[clip] (  0.00,  0.00) rectangle (230.54,138.04);
\definecolor[named]{drawColor}{rgb}{0.00,0.00,0.00}

\path[draw=drawColor,line width= 0.6pt,line join=round] ( 89.93,  0.00) --
	( 89.93,138.04);
\end{scope}
\begin{scope}
\path[clip] (  0.00,  0.00) rectangle (230.54,138.04);
\definecolor[named]{drawColor}{rgb}{0.00,0.00,0.00}

\node[text=drawColor,anchor=base east,inner sep=0pt, outer sep=0pt, scale=  1.00] at ( 82.82,  2.10) {Maíz amarillo};

\node[text=drawColor,anchor=base east,inner sep=0pt, outer sep=0pt, scale=  1.00] at ( 82.82, 15.90) {Harina de soya};

\node[text=drawColor,anchor=base east,inner sep=0pt, outer sep=0pt, scale=  1.00] at ( 82.82, 29.70) {Papel y cartón crudos};

\node[text=drawColor,anchor=base east,inner sep=0pt, outer sep=0pt, scale=  1.00] at ( 82.82, 43.50) {Vehiculos de carga};

\node[text=drawColor,anchor=base east,inner sep=0pt, outer sep=0pt, scale=  1.00] at ( 82.82, 57.30) {Fuel oil (Bunker C)};

\node[text=drawColor,anchor=base east,inner sep=0pt, outer sep=0pt, scale=  1.00] at ( 82.82, 71.11) {Teléfonos celulares};

\node[text=drawColor,anchor=base east,inner sep=0pt, outer sep=0pt, scale=  1.00] at ( 82.82, 84.91) {Gas propano};

\node[text=drawColor,anchor=base east,inner sep=0pt, outer sep=0pt, scale=  1.00] at ( 82.82, 98.71) {Medicamentos P/Humanos};

\node[text=drawColor,anchor=base east,inner sep=0pt, outer sep=0pt, scale=  1.00] at ( 82.82,112.51) {Gasolina};

\node[text=drawColor,anchor=base east,inner sep=0pt, outer sep=0pt, scale=  1.00] at ( 82.82,126.31) {Diesel oil };
\end{scope}
\begin{scope}
\path[clip] (  0.00,  0.00) rectangle (230.54,138.04);

\path[] ( 85.66,  5.54) --
	( 89.93,  5.54);

\path[] ( 85.66, 19.34) --
	( 89.93, 19.34);

\path[] ( 85.66, 33.14) --
	( 89.93, 33.14);

\path[] ( 85.66, 46.95) --
	( 89.93, 46.95);

\path[] ( 85.66, 60.75) --
	( 89.93, 60.75);

\path[] ( 85.66, 74.55) --
	( 89.93, 74.55);

\path[] ( 85.66, 88.35) --
	( 89.93, 88.35);

\path[] ( 85.66,102.15) --
	( 89.93,102.15);

\path[] ( 85.66,115.95) --
	( 89.93,115.95);

\path[] ( 85.66,129.75) --
	( 89.93,129.75);
\end{scope}
  \end{tikzpicture}}{INE, con datos del BANGUAT.}{\notitasin{Los datos del año 2014 se presentan como preliminares y serán ajustados por el registro tardío de los mismos.}}}
\hojados{\columna{Importaciones 10 principales países trimestre 2014}{El principal socio comercial de Guatemala es Estados unidos de donde se importó en el primer trimestre 2014,  US\$1,720,385,540 con un porcentaje de 39.3\% seguido de China  a quien se le compró US\$536,462,801 con un porcentaje de 12.3\%, en tercer lugar Mexico  País al que se le compró US\$431,480,501 con un porcentaje de 9.9\%, continuando con países  como El Salvador y Corea del Sur con un porcentaje de 3.4\% y 3.2\% respectivamente}{}{Importación a los principales 10 países, Primer Trimestre 2014}{(cifras preliminares en US\$)}{\ \\[6mm]\begin{tikzpicture}[x=1pt,y=1pt,scale=1]  % Created by tikzDevice version 0.7.0 on 2014-12-15 22:10:50
% !TEX encoding = UTF-8 Unicode
\definecolor[named]{fillColor}{rgb}{1.00,1.00,1.00}
\path[use as bounding box,fill=fillColor,fill opacity=0.00] (0,0) rectangle (230.54,138.04);
\begin{scope}
\path[clip] (  0.00,  0.00) rectangle (230.54,138.04);
\definecolor[named]{drawColor}{rgb}{1.00,1.00,1.00}

\path[draw=drawColor,line width= 0.6pt,line join=round,line cap=round] (  0.00, -0.00) rectangle (230.54,138.04);
\end{scope}
\begin{scope}
\path[clip] (  0.00,  0.00) rectangle (230.54,138.04);

\path[] (  5.80, 57.30) rectangle (230.54,123.81);

\path[] ( 19.02, 57.30) --
	( 19.02,123.81);

\path[] ( 41.05, 57.30) --
	( 41.05,123.81);

\path[] ( 63.08, 57.30) --
	( 63.08,123.81);

\path[] ( 85.12, 57.30) --
	( 85.12,123.81);

\path[] (107.15, 57.30) --
	(107.15,123.81);

\path[] (129.19, 57.30) --
	(129.19,123.81);

\path[] (151.22, 57.30) --
	(151.22,123.81);

\path[] (173.25, 57.30) --
	(173.25,123.81);

\path[] (195.29, 57.30) --
	(195.29,123.81);

\path[] (217.32, 57.30) --
	(217.32,123.81);
\definecolor[named]{drawColor}{rgb}{0.00,0.00,0.00}

\path[draw=drawColor,line width= 0.6pt,line join=round] ( 12.41, 57.30) rectangle ( 25.63,123.81);

\path[draw=drawColor,line width= 0.6pt,line join=round] ( 34.44, 57.30) rectangle ( 47.66, 78.03);

\path[draw=drawColor,line width= 0.6pt,line join=round] ( 56.47, 57.30) rectangle ( 69.69, 73.97);

\path[draw=drawColor,line width= 0.6pt,line join=round] ( 78.51, 57.30) rectangle ( 91.73, 63.10);

\path[draw=drawColor,line width= 0.6pt,line join=round] (100.54, 57.30) rectangle (113.76, 62.72);

\path[draw=drawColor,line width= 0.6pt,line join=round] (122.58, 57.30) rectangle (135.80, 61.44);

\path[draw=drawColor,line width= 0.6pt,line join=round] (144.61, 57.30) rectangle (157.83, 61.40);

\path[draw=drawColor,line width= 0.6pt,line join=round] (166.64, 57.30) rectangle (179.86, 61.01);

\path[draw=drawColor,line width= 0.6pt,line join=round] (188.68, 57.30) rectangle (201.90, 60.94);

\path[draw=drawColor,line width= 0.6pt,line join=round] (210.71, 57.30) rectangle (223.93, 60.20);

\node[text=drawColor,anchor=base,inner sep=0pt, outer sep=0pt, scale=  0.85] at ( 19.02,126.74) {1,720};

\node[text=drawColor,anchor=base,inner sep=0pt, outer sep=0pt, scale=  0.85] at ( 41.05, 80.96) {536};

\node[text=drawColor,anchor=base,inner sep=0pt, outer sep=0pt, scale=  0.85] at ( 63.08, 76.90) {431};

\node[text=drawColor,anchor=base,inner sep=0pt, outer sep=0pt, scale=  0.85] at ( 85.12, 66.03) {150};

\node[text=drawColor,anchor=base,inner sep=0pt, outer sep=0pt, scale=  0.85] at (107.15, 65.64) {140};

\node[text=drawColor,anchor=base,inner sep=0pt, outer sep=0pt, scale=  0.85] at (129.19, 64.37) {107};

\node[text=drawColor,anchor=base,inner sep=0pt, outer sep=0pt, scale=  0.85] at (151.22, 64.33) {106};

\node[text=drawColor,anchor=base,inner sep=0pt, outer sep=0pt, scale=  0.85] at (173.25, 63.94) {96};

\node[text=drawColor,anchor=base,inner sep=0pt, outer sep=0pt, scale=  0.85] at (195.29, 63.86) {94};

\node[text=drawColor,anchor=base,inner sep=0pt, outer sep=0pt, scale=  0.85] at (217.32, 63.13) {75};
\end{scope}
\begin{scope}
\path[clip] (  0.00,  0.00) rectangle (230.54,138.04);

\path[] (  5.80, 57.30) --
	(  5.80,123.81);
\end{scope}
\begin{scope}
\path[clip] (  0.00,  0.00) rectangle (230.54,138.04);
\definecolor[named]{drawColor}{rgb}{0.00,0.00,0.00}

\path[draw=drawColor,line width= 0.6pt,line join=round] (  5.80, 57.30) --
	(230.54, 57.30);
\end{scope}
\begin{scope}
\path[clip] (  0.00,  0.00) rectangle (230.54,138.04);

\path[] ( 19.02, 53.03) --
	( 19.02, 57.30);

\path[] ( 41.05, 53.03) --
	( 41.05, 57.30);

\path[] ( 63.08, 53.03) --
	( 63.08, 57.30);

\path[] ( 85.12, 53.03) --
	( 85.12, 57.30);

\path[] (107.15, 53.03) --
	(107.15, 57.30);

\path[] (129.19, 53.03) --
	(129.19, 57.30);

\path[] (151.22, 53.03) --
	(151.22, 57.30);

\path[] (173.25, 53.03) --
	(173.25, 57.30);

\path[] (195.29, 53.03) --
	(195.29, 57.30);

\path[] (217.32, 53.03) --
	(217.32, 57.30);
\end{scope}
\begin{scope}
\path[clip] (  0.00,  0.00) rectangle (230.54,138.04);
\definecolor[named]{drawColor}{rgb}{0.00,0.00,0.00}

\node[text=drawColor,rotate= 90.00,anchor=base east,inner sep=0pt, outer sep=0pt, scale=  1.00] at ( 22.46, 50.19) {USA};

\node[text=drawColor,rotate= 90.00,anchor=base east,inner sep=0pt, outer sep=0pt, scale=  1.00] at ( 44.49, 50.19) {China};

\node[text=drawColor,rotate= 90.00,anchor=base east,inner sep=0pt, outer sep=0pt, scale=  1.00] at ( 66.53, 50.19) {México};

\node[text=drawColor,rotate= 90.00,anchor=base east,inner sep=0pt, outer sep=0pt, scale=  1.00] at ( 88.56, 50.19) {El Salvador};

\node[text=drawColor,rotate= 90.00,anchor=base east,inner sep=0pt, outer sep=0pt, scale=  1.00] at (110.60, 50.19) {Corea del Sur};

\node[text=drawColor,rotate= 90.00,anchor=base east,inner sep=0pt, outer sep=0pt, scale=  1.00] at (132.63, 50.19) {Costa Rica};

\node[text=drawColor,rotate= 90.00,anchor=base east,inner sep=0pt, outer sep=0pt, scale=  1.00] at (154.66, 50.19) {Japón};

\node[text=drawColor,rotate= 90.00,anchor=base east,inner sep=0pt, outer sep=0pt, scale=  1.00] at (176.70, 50.19) {Colombia};

\node[text=drawColor,rotate= 90.00,anchor=base east,inner sep=0pt, outer sep=0pt, scale=  1.00] at (198.73, 50.19) {Alemania};

\node[text=drawColor,rotate= 90.00,anchor=base east,inner sep=0pt, outer sep=0pt, scale=  1.00] at (220.76, 50.19) {Perú};
\end{scope}
  \end{tikzpicture}}{INE, con datos del BANGUAT.}{\notitasin{Los datos del año 2014 se presentan como preliminares y serán ajustados por el registro tardío de los mismos.}}}
{\columna{Importaciones 10 principales secciones}{De las secciones del Sistema Arancelario Centroaméricano (SAC) en el primer trimestre 2014 en las importaciones a Guatemala a productos minerales ocupa el primer lugar con un 24.9\%, el segundo lugar con los productos de las Industrias der m áquinas y aparatos con 16.6\%,  en tercer lugar los productos de las Industrias Químicas con un 13.4\% continuando con materias plásticas y metales comunes con un 7.5\% y 7.0\% respectivamente}{}{Importación de los Principales Secciones del Sistema Arancelario C.A., Primer Trimestre 2014}{(Cifras expresadas en US\$ )}{\ \\[6mm]\begin{tikzpicture}[x=1pt,y=1pt,scale=1]  % Created by tikzDevice version 0.7.0 on 2014-11-24 15:12:40
% !TEX encoding = UTF-8 Unicode
\definecolor[named]{fillColor}{rgb}{1.00,1.00,1.00}
\path[use as bounding box,fill=fillColor,fill opacity=0.00] (0,0) rectangle (280.41,195.13);
\begin{scope}
\path[clip] (  0.00,  0.00) rectangle (280.41,195.13);
\definecolor[named]{drawColor}{rgb}{1.00,1.00,1.00}

\path[draw=drawColor,line width= 0.6pt,line join=round,line cap=round] (  0.00,  0.00) rectangle (280.41,195.13);
\end{scope}
\begin{scope}
\path[clip] (  0.00,  0.00) rectangle (280.41,195.13);

\path[] (  7.00, 46.05) rectangle (280.41,180.90);

\path[] ( 20.45, 46.05) --
	( 20.45,180.90);

\path[] ( 42.86, 46.05) --
	( 42.86,180.90);

\path[] ( 65.27, 46.05) --
	( 65.27,180.90);

\path[] ( 87.68, 46.05) --
	( 87.68,180.90);

\path[] (110.09, 46.05) --
	(110.09,180.90);

\path[] (132.50, 46.05) --
	(132.50,180.90);

\path[] (154.91, 46.05) --
	(154.91,180.90);

\path[] (177.32, 46.05) --
	(177.32,180.90);

\path[] (199.73, 46.05) --
	(199.73,180.90);

\path[] (222.14, 46.05) --
	(222.14,180.90);

\path[] (244.55, 46.05) --
	(244.55,180.90);

\path[] (266.96, 46.05) --
	(266.96,180.90);
\definecolor[named]{fillColor}{rgb}{0.86,0.68,0.43}

\path[fill=fillColor] ( 13.72, 46.05) rectangle ( 27.17,180.90);

\path[fill=fillColor] ( 36.13, 46.05) rectangle ( 49.58,164.05);

\path[fill=fillColor] ( 58.54, 46.05) rectangle ( 71.99,171.07);

\path[fill=fillColor] ( 80.96, 46.05) rectangle ( 94.40,169.67);

\path[fill=fillColor] (103.37, 46.05) rectangle (116.81,159.83);

\path[fill=fillColor] (125.78, 46.05) rectangle (139.22,162.64);

\path[fill=fillColor] (148.19, 46.05) rectangle (161.63,162.64);

\path[fill=fillColor] (170.60, 46.05) rectangle (184.04,158.43);

\path[fill=fillColor] (193.01, 46.05) rectangle (206.45,168.26);

\path[fill=fillColor] (215.42, 46.05) rectangle (228.86,154.21);

\path[fill=fillColor] (237.83, 46.05) rectangle (251.27,158.43);

\path[fill=fillColor] (260.24, 46.05) rectangle (273.68,145.79);
\definecolor[named]{drawColor}{rgb}{0.00,0.00,0.00}

\node[text=drawColor,anchor=base,inner sep=0pt, outer sep=0pt, scale=  0.71] at ( 20.45,183.83) {9.6};

\node[text=drawColor,anchor=base,inner sep=0pt, outer sep=0pt, scale=  0.71] at ( 42.86,166.97) {8.4};

\node[text=drawColor,anchor=base,inner sep=0pt, outer sep=0pt, scale=  0.71] at ( 65.27,174.00) {8.9};

\node[text=drawColor,anchor=base,inner sep=0pt, outer sep=0pt, scale=  0.71] at ( 87.68,172.59) {8.8};

\node[text=drawColor,anchor=base,inner sep=0pt, outer sep=0pt, scale=  0.71] at (110.09,162.76) {8.1};

\node[text=drawColor,anchor=base,inner sep=0pt, outer sep=0pt, scale=  0.71] at (132.50,165.57) {8.3};

\node[text=drawColor,anchor=base,inner sep=0pt, outer sep=0pt, scale=  0.71] at (154.91,165.57) {8.3};

\node[text=drawColor,anchor=base,inner sep=0pt, outer sep=0pt, scale=  0.71] at (177.32,161.36) {8.0};

\node[text=drawColor,anchor=base,inner sep=0pt, outer sep=0pt, scale=  0.71] at (199.73,171.19) {8.7};

\node[text=drawColor,anchor=base,inner sep=0pt, outer sep=0pt, scale=  0.71] at (222.14,157.14) {7.7};

\node[text=drawColor,anchor=base,inner sep=0pt, outer sep=0pt, scale=  0.71] at (244.55,161.36) {8.0};

\node[text=drawColor,anchor=base,inner sep=0pt, outer sep=0pt, scale=  0.71] at (266.96,148.71) {7.1};
\end{scope}
\begin{scope}
\path[clip] (  0.00,  0.00) rectangle (280.41,195.13);

\path[] (  7.00, 46.05) --
	(  7.00,180.90);
\end{scope}
\begin{scope}
\path[clip] (  0.00,  0.00) rectangle (280.41,195.13);
\definecolor[named]{drawColor}{rgb}{0.60,0.60,0.60}

\path[draw=drawColor,line width= 0.6pt,line join=round] (  7.00, 46.05) --
	(280.41, 46.05);
\end{scope}
\begin{scope}
\path[clip] (  0.00,  0.00) rectangle (280.41,195.13);

\path[] ( 20.45, 41.78) --
	( 20.45, 46.05);

\path[] ( 42.86, 41.78) --
	( 42.86, 46.05);

\path[] ( 65.27, 41.78) --
	( 65.27, 46.05);

\path[] ( 87.68, 41.78) --
	( 87.68, 46.05);

\path[] (110.09, 41.78) --
	(110.09, 46.05);

\path[] (132.50, 41.78) --
	(132.50, 46.05);

\path[] (154.91, 41.78) --
	(154.91, 46.05);

\path[] (177.32, 41.78) --
	(177.32, 46.05);

\path[] (199.73, 41.78) --
	(199.73, 46.05);

\path[] (222.14, 41.78) --
	(222.14, 46.05);

\path[] (244.55, 41.78) --
	(244.55, 46.05);

\path[] (266.96, 41.78) --
	(266.96, 46.05);
\end{scope}
\begin{scope}
\path[clip] (  0.00,  0.00) rectangle (280.41,195.13);
\definecolor[named]{drawColor}{rgb}{0.00,0.00,0.00}

\node[text=drawColor,rotate= 90.00,anchor=base east,inner sep=0pt, outer sep=0pt, scale=  0.83] at ( 23.89, 38.94) {Enero};

\node[text=drawColor,rotate= 90.00,anchor=base east,inner sep=0pt, outer sep=0pt, scale=  0.83] at ( 46.30, 38.94) {Febrero};

\node[text=drawColor,rotate= 90.00,anchor=base east,inner sep=0pt, outer sep=0pt, scale=  0.83] at ( 68.71, 38.94) {Marzo};

\node[text=drawColor,rotate= 90.00,anchor=base east,inner sep=0pt, outer sep=0pt, scale=  0.83] at ( 91.12, 38.94) {Abril};

\node[text=drawColor,rotate= 90.00,anchor=base east,inner sep=0pt, outer sep=0pt, scale=  0.83] at (113.53, 38.94) {Mayo};

\node[text=drawColor,rotate= 90.00,anchor=base east,inner sep=0pt, outer sep=0pt, scale=  0.83] at (135.94, 38.94) {Junio};

\node[text=drawColor,rotate= 90.00,anchor=base east,inner sep=0pt, outer sep=0pt, scale=  0.83] at (158.35, 38.94) {Julio};

\node[text=drawColor,rotate= 90.00,anchor=base east,inner sep=0pt, outer sep=0pt, scale=  0.83] at (180.76, 38.94) {Agosto};

\node[text=drawColor,rotate= 90.00,anchor=base east,inner sep=0pt, outer sep=0pt, scale=  0.83] at (203.17, 38.94) {Septiembre};

\node[text=drawColor,rotate= 90.00,anchor=base east,inner sep=0pt, outer sep=0pt, scale=  0.83] at (225.58, 38.94) {Octubre};

\node[text=drawColor,rotate= 90.00,anchor=base east,inner sep=0pt, outer sep=0pt, scale=  0.83] at (247.99, 38.94) {Noviembre};

\node[text=drawColor,rotate= 90.00,anchor=base east,inner sep=0pt, outer sep=0pt, scale=  0.83] at (270.40, 38.94) {Diciembre};
\end{scope}
  \end{tikzpicture}}{INE, con datos del BANGUAT.}{\notitasin{Los datos del año 2014 se presentan como preliminares y serán ajustados por el registro tardío de los mismos.}}}


\end{document}