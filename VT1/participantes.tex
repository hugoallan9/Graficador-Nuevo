$\ $
\vspace{4.5cm}

\noindent\begin{tabular}{p{0.1cm}p{6.8cm}}
	& 2014.$\,$ Guatemala, América Central \\
	&\Bold Instituto Nacional de Estadística\\[-0.4cm]
	&\color{blue!50!black}\url{www.ine.gob.gt}\\[0.9cm]
\end{tabular}\\
\noindent\begin{tabular}{p{0.1cm}p{6.8cm}}
	& Está permitida la reproducción parcial o total de los contenidos de este documento con la mención de la fuente. \\[0.5cm]
	
	& Este documento fue elaborado empleando  {\Sans R}, Inkscape, Libre Office y {\Logos \XeLaTeX}.\\
\end{tabular} 


\clearpage




$\ $
\vspace{3.5cm}

\begin{center}
	\Bold \LARGE REPÚBLICA DE GUATEMALA\\
	ESTADÍSTICAS VITALES I-2014\
\end{center}
\cleardoublepage

\hoja{
	$\ $
	\vspace{0.3cm}
	
	\begin{center}
		{\Bold \LARGE AUTORIDADES}\\[0.7cm]
		
		
		{\Bold \large \color{color2} JUNTA  DIRECTIVA} \\[0.5cm]
		
		\begin{center}
			\begin{tabular}{x{6.0cm}x{6.0cm}}
				{\Bold Ministerio de Economía}   & 		{\Bold Ministerio de Finanzas}\\
				Lic. Sergio de la Torre, Titular & Lic. Dorval Carías, Titular \\
				Lic. Jacobo Rey Sigfrido Lee, Suplente & Lic. Edwin Oswaldo Martínez, Suplente \\
				& \\
				{\Bold Ministerio de Agricultura,} & {\Bold Ministerio de Energía y Minas}\\ 
				{\Bold Ganadería y Alimentación} & Lic. Erick Archila, Titular \\
				Ing. Elmer López, Titular & Licda. Ivanova Ancheta, Suplente \\
				Ing. Carlos Alfonso Anzueto, Suplente & \\
				
				& {\Bold Banco de Guatemala} \\
				{\Bold Secretaría de Planificación y} & Lic. Julio Roberto Suárez, Titular\\
				{\Bold Programación de la Presidencia} & Lic. Sergio Francisco Recinos Rivera, Suplente\\
				Licda. Ekaterina Arbolievna Parrilla, Titular & \\
				
				Licda. Dora Marina Coc, Suplente & {\Bold Universidad de San Carlos de Guatemala} \\
				
				& Ing. Murphy Olimpo Paiz, Titular \\
				{\Bold Comité Coordinador de } & Lic. Oscar René Paniagua Carrera, Suplente \\  
				{\Bold Asociaciones  Agrícolas, Comerciales, } & \\
				{\Bold Industriales y Financieras} & {\Bold Universidades Privadas} \\
				Lic. Juan Raúl Aguilar , Titular & Lic. Miguel Franco de León, Titular \\
				Lic. Oscar Augusto Sequeira, Suplente & Lic. Ariel Rivera Irías, Suplente \\[0.3cm]
			\end{tabular} 	
		\end{center}	
		
		
		
		
		
		{\Bold \large \color{color2} GERENCIA}\\[0.2cm]
		Lic. Rubén Darío Narciso Cruz, Gerente\\
		Ing. Orlando Monzón Girón, Subgerente Administrativo Financiero\\
		Lic. Jaime Mejía Salguero, Subgerente Técnico\\
		
		
	\end{center}
}{}
\clearpage

$\ $
\vspace{0.5cm}

\begin{center}
	{\Bold \LARGE EQUIPO RESPONSABLE}\\[1.5cm]
	
	{\Bold \large \color{color2} REVISIÓN GENERAL}\\[0.2cm]
	Rubén Narciso\\[0.8cm]
	
	
	{\Bold \large \color{color2} EQUIPO TÉCNICO}\\[0.2cm]
	Flor de María Hernández Soto\\
	Cristian Miguel Cabrera Ayala\\
	Blanca Angelica Ramirez González\\
	Marlon Humberto Pirir Garcia\\[0.8cm]
	
	{\Bold \large \color{color2} DIAGRAMACIÓN Y DISEÑO}\\[0.2cm]
	Hugo Allan García Monterrosa\\
	Fabiola Beatriz Ramírez Pinto\\
	José Carlos Bonilla Aldana\\[0.8cm]
	
	
	
\end{center}\setcounter{page}{0}\cleardoublepage



$\ $\\[0.7cm]

\tableofcontents

\cleardoublepage
\pagestyle{estandar}
\setcounter{page}{1}
\setlength{\arrayrulewidth}{1.0pt}


\cleardoublepage





$\ $\\[0.5cm]
\thispagestyle{empty}
\noindent {\Bold \LARGE Presentación}




$\ $\\



El Instituto Nacional de Estadística -INE-, consciente de la demanda de información demográfica y siendo el ente rector de la política estadística nacional en Guatemala, en cumplimiento a su Ley Orgánica, Decreto Ley 3-85, se complace en presentar el siguiente informe, que contiene las {\Bold Estadísticas Vitales}, con información correspondiente al {\Bold primer trimestre del 2014}, información esencial para la planificación del desarrollo humano.

La información presentada a continuación fue recolectada a través del Registro Nacional de las Personas  -RENAP- y consiste en los hechos ocurridos sobre nacimientos, defunciones, defunciones fetales, matrimonios y divorcios registrados en el primer trimestre del 2014. 

Sin embargo, los datos para el período {\Bold son preliminares}, sujetos a la adición de registros ingresados tardíamente.

Por lo tanto, el INE se complace en presentar este informe, con el propósito de brindar una herramienta más de análisis a la población guatemalteca, y a la vez agradece el aporte y colaboración del Registro Nacional de las Personas, a quien se insta  a continuar con el apoyo a este proceso.


\thispagestyle{empty}


\cleardoublepage
